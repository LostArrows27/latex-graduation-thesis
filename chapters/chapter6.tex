\chapter{Kết luận}

Khóa luận đã thực hiện nghiên cứu đánh giá các mẫu giao tiếp trong kiến trúc microservice thông qua triển khai thực tế và kiểm thử so sánh trên bốn kịch bản nghiệp vụ phổ biến. Kết quả cho thấy mỗi mẫu giao tiếp đều có những ưu và nhược điểm riêng, phù hợp với những bối cảnh cụ thể.

Nghiên cứu này đã đóng góp một số điểm quan trọng cho lĩnh vực kiến trúc microservice. Đầu tiên, đã cung cấp một khung đánh giá toàn diện cho các mẫu giao tiếp microservice, bao gồm các tiêu chí như hiệu suất, độ tin cậy, khả năng chịu lỗi và hiệu quả sử dụng tài nguyên. Thứ hai, nghiên cứu đã triển khai và đánh giá các mẫu giao tiếp trong các kịch bản thực tế phổ biến, cung cấp cái nhìn thực tiễn về ưu nhược điểm của từng mẫu trong các ngữ cảnh cụ thể. Thứ ba, nghiên cứu đã đề xuất các khuyến nghị về việc lựa chọn mẫu giao tiếp phù hợp dựa trên đặc điểm của kịch bản nghiệp vụ, giúp các nhà phát triển đưa ra quyết định thiết kế hiệu quả. Cuối cùng, nghiên cứu đã đề xuất một mô hình tích hợp các mẫu giao tiếp khác nhau, tận dụng ưu điểm của từng phương pháp để tối ưu hóa hiệu suất tổng thể của hệ thống.

Mặc dù nghiên cứu đã cung cấp những hiểu biết có giá trị, vẫn còn một số hạn chế cần ghi nhận. Các kịch bản kiểm thử được thực hiện trong môi trường có kiểm soát, có thể không hoàn toàn phản ánh các điều kiện hoạt động thực tế với các yếu tố như độ trễ mạng không đồng đều, sự cố phần cứng, và các vấn đề về bảo mật. Nghiên cứu tập trung chủ yếu vào hiệu suất và độ tin cậy, mà không đi sâu vào các khía cạnh khác như bảo mật, chi phí triển khai và vận hành, và khả năng tích hợp với các hệ thống hiện có. Việc đánh giá cũng chỉ bao gồm ba mẫu giao tiếp chính, trong khi có nhiều mẫu và biến thể khác cũng được sử dụng trong thực tế, như gRPC, GraphQL, và WebSockets. Cuối cùng, dự án không đánh giá hiệu suất của các mẫu giao tiếp trong thời gian dài, có thể làm thiếu sót các vấn đề như rò rỉ bộ nhớ, suy giảm hiệu suất theo thời gian, và khả năng phục hồi sau sự cố lớn.

Hướng nghiên cứu tiếp theo có thể mở rộng phạm vi đánh giá để bao gồm nhiều mẫu giao tiếp hơn, các kịch bản đa dạng hơn, và thời gian kiểm thử dài hơn. Việc đánh giá trong các môi trường đám mây thực tế với các yếu tố nhiễu như bị phát giả, tấn công DDoS, và sự cố phần cứng sẽ cung cấp những hiểu biết sâu sắc hơn về độ tin cậy của các mẫu giao tiếp trong điều kiện khắc nghiệt.

Một số xu hướng đáng chú ý cho nghiên cứu tương lai bao gồm server-less communication, nơi các hàm serverless được sử dụng để xử lý các sự kiện và thông điệp mà không cần quan tâm đến cơ sở hạ tầng bên dưới. Service Mesh, một lớp cơ sở hạ tầng chuyên dụng cho giao tiếp service-to-service, cung cấp các tính năng như service discovery, load balancing, encryption, observability, và authentication/authorization. Việc kết hợp các mẫu giao tiếp với các công nghệ stream processing như Apache Flink hoặc Kafka Streams cũng là một hướng nghiên cứu hứa hẹn. Cuối cùng, việc áp dụng các kỹ thuật học máy và trí tuệ nhân tạo để tự động hóa việc lựa chọn và tối ưu hóa các mẫu giao tiếp dựa trên đặc điểm của tải và yêu cầu nghiệp vụ là một hướng nghiên cứu đầy tiềm năng.

Tóm lại, việc lựa chọn và triển khai mẫu giao tiếp phù hợp là yếu tố then chốt quyết định hiệu suất, độ tin cậy và khả năng mở rộng của hệ thống microservice. Không có một mẫu giao tiếp nào là tối ưu cho mọi tình huống, mà cần phân tích kỹ yêu cầu nghiệp vụ và đặc điểm kỹ thuật của từng kịch bản để đưa ra lựa chọn phù hợp. Trong nhiều trường hợp, việc kết hợp các mẫu giao tiếp khác nhau trong cùng một hệ thống có thể mang lại kết quả tốt nhất, tận dụng ưu điểm của từng phương pháp. Cuối cùng, việc liên tục đánh giá và tối ưu hóa chiến lược giao tiếp là cần thiết khi hệ thống phát triển.