\section{Tổng quan về Microservice Architecture}

\subsection{Định nghĩa và đặc điểm}
Kiến trúc Microservice là một phương pháp phát triển phần mềm trong đó một ứng
dụng được cấu thành từ nhiều dịch vụ nhỏ, độc lập và có khả năng triển khai
riêng biệt. Mỗi dịch vụ này được thiết kế để thực hiện một chức năng cụ thể
trong phạm vi nghiệp vụ được định nghĩa rõ ràng, và giao tiếp với các dịch vụ
khác thông qua các cơ chế giao tiếp nhẹ, thường là API.

Theo Newman (2021), các đặc điểm chính của kiến trúc microservice bao gồm tính
tự trị cao, trong đó mỗi dịch vụ có thể được phát triển, triển khai và mở rộng
độc lập với các dịch vụ khác. Các dịch vụ được tổ chức xoay quanh các khả năng
nghiệp vụ thay vì các lớp công nghệ, thể hiện sự phân tách theo chức năng
nghiệp vụ. Quản lý dữ liệu trong microservice được thực hiện phi tập trung, với
mỗi dịch vụ quản lý dữ liệu riêng và chỉ có thể truy cập dữ liệu thông qua API
của dịch vụ sở hữu dữ liệu đó.

Thiết kế hướng lỗi là một đặc điểm quan trọng khác của microservice, trong đó
các dịch vụ được thiết kế để xử lý lỗi và khả năng các dịch vụ khác không khả
dụng. Cuối cùng, microservice cho phép tiến hóa độc lập, nghĩa là các dịch vụ
có thể thay đổi và phát triển theo thời gian mà không ảnh hưởng đến toàn bộ hệ
thống.

\subsection{So sánh với kiến trúc nguyên khối (Monolithic)}
Để hiểu rõ hơn về kiến trúc microservice, việc so sánh với kiến trúc nguyên khối là rất hữu ích. Trong kiến trúc nguyên khối, toàn bộ ứng dụng được xây dựng như một đơn vị duy nhất. Tất cả các chức năng nằm trong một codebase và được triển khai cùng nhau.

Về triển khai, kiến trúc nguyên khối đòi hỏi toàn bộ ứng dụng được triển khai
cùng một lúc, trong khi kiến trúc microservice cho phép các dịch vụ được triển
khai độc lập. Điều này có ý nghĩa quan trọng trong việc giảm thiểu rủi ro và
tăng tốc độ phát hành.

Khả năng mở rộng cũng khác biệt đáng kể giữa hai kiến trúc. Trong kiến trúc
nguyên khối, toàn bộ ứng dụng phải được mở rộng, ngay cả khi chỉ một phần cần
thêm tài nguyên. Ngược lại, kiến trúc microservice cho phép mở rộng từng dịch
vụ riêng biệt, tối ưu hóa việc sử dụng tài nguyên.

Về phát triển, kiến trúc nguyên khối thường có một nhóm phát triển làm việc
trên một codebase, dẫn đến các xung đột trong quá trình phát triển và triển
khai. Trong khi đó, kiến trúc microservice cho phép nhiều nhóm làm việc độc lập
trên các dịch vụ khác nhau, tăng tốc độ phát triển và giảm thiểu xung đột.

Công nghệ là một khía cạnh khác có sự khác biệt. Kiến trúc nguyên khối thường
bị giới hạn trong một stack công nghệ, trong khi mỗi microservice có thể sử
dụng công nghệ phù hợp nhất với yêu cầu của nó. Điều này tạo ra sự linh hoạt và
khả năng thích ứng với các công nghệ mới.

Khả năng chịu lỗi cũng là một điểm khác biệt quan trọng. Trong kiến trúc nguyên
khối, lỗi ở một phần có thể ảnh hưởng đến toàn bộ ứng dụng, trong khi trong
kiến trúc microservice, lỗi được cô lập trong một dịch vụ, giảm thiểu tác động
đến toàn bộ hệ thống.

Cuối cùng, về độ phức tạp, kiến trúc nguyên khối đơn giản hơn trong các ứng
dụng nhỏ, nhưng phức tạp hơn khi ứng dụng phát triển. Ngược lại, kiến trúc
microservice phức tạp hơn ngay từ đầu do tính phân tán, nhưng độ phức tạp này
được quản lý tốt hơn khi hệ thống phát triển.

\subsection{Lợi ích và thách thức của kiến trúc microservice}
Kiến trúc microservice mang lại nhiều lợi ích đáng kể cho việc phát triển và
vận hành phần mềm. Một trong những lợi ích chính là khả năng mở rộng có mục
tiêu. Các dịch vụ có thể được mở rộng độc lập dựa trên nhu cầu, tối ưu hóa việc
sử dụng tài nguyên. Điều này đặc biệt quan trọng trong môi trường cloud, nơi
chi phí tỷ lệ thuận với tài nguyên được sử dụng.

Phát triển nhanh hơn là một lợi ích khác của kiến trúc microservice. Các nhóm
nhỏ có thể làm việc trên các dịch vụ độc lập, cho phép phát triển song song và
chu kỳ phát hành nhanh hơn. Mỗi nhóm có thể tập trung vào một dịch vụ cụ thể,
hiểu rõ nó và phát triển nó một cách hiệu quả.

Tính linh hoạt công nghệ cũng là một lợi thế đáng kể. Mỗi dịch vụ có thể sử
dụng công nghệ phù hợp nhất với yêu cầu của nó. Ví dụ, một dịch vụ xử lý giao
dịch có thể sử dụng một ngôn ngữ chú trọng vào tính nhất quán, trong khi một
dịch vụ phân tích dữ liệu có thể sử dụng một ngôn ngữ tối ưu cho xử lý dữ liệu
lớn.

Khả năng chịu lỗi tốt hơn là một lợi ích khác của kiến trúc microservice. Lỗi
trong một dịch vụ không nhất thiết phải làm cho toàn bộ hệ thống không khả
dụng. Ví dụ, nếu dịch vụ gợi ý sản phẩm không hoạt động, người dùng vẫn có thể
duyệt và mua sản phẩm.

Khả năng bảo trì và hiểu biết tốt hơn cũng là một lợi thế của kiến trúc
microservice. Các dịch vụ nhỏ hơn dễ hiểu và bảo trì hơn các ứng dụng lớn. Mã
nguồn của mỗi dịch vụ nhỏ hơn và tập trung vào một chức năng cụ thể, giúp nhà
phát triển dễ dàng hiểu và thay đổi nó.

Tuy nhiên, kiến trúc microservice cũng đặt ra một số thách thức đáng kể. Độ
phức tạp phân tán là một thách thức lớn. Hệ thống phân tán vốn phức tạp hơn,
đòi hỏi kiến thức và công cụ chuyên biệt. Các vấn đề như latency mạng, xử lý
lỗi và đồng bộ hóa dữ liệu trở nên phức tạp hơn trong một hệ thống phân tán.

Giao tiếp giữa các dịch vụ là một thách thức khác. Thiết kế và quản lý giao
tiếp giữa các dịch vụ đòi hỏi cân nhắc kỹ lưỡng về hiệu suất, độ tin cậy và khả
năng mở rộng. Việc lựa chọn giao thức giao tiếp phù hợp và xử lý các trường hợp
lỗi trong giao tiếp là các vấn đề phức tạp.

Quản lý dữ liệu cũng là một thách thức đáng kể trong kiến trúc microservice.
Duy trì tính nhất quán dữ liệu giữa các dịch vụ có thể phức tạp, đặc biệt là
khi mỗi dịch vụ có cơ sở dữ liệu riêng. Các mẫu như Saga và Event Sourcing được
sử dụng để giải quyết vấn đề này, nhưng chúng cũng đưa ra sự phức tạp riêng.

Vận hành và giám sát là một thách thức khác của kiến trúc microservice. Triển
khai và giám sát nhiều dịch vụ đòi hỏi công cụ và quy trình tinh vi hơn. Các
công cụ như Kubernetes và Prometheus đã được phát triển để giải quyết vấn đề
này, nhưng chúng cũng đòi hỏi kiến thức và nỗ lực đáng kể để sử dụng hiệu quả.

Cuối cùng, kiểm thử cũng trở nên phức tạp hơn trong kiến trúc microservice.
Kiểm thử tích hợp đòi hỏi sự phối hợp giữa nhiều dịch vụ, có thể chạy trên các
máy khác nhau và sử dụng các công nghệ khác nhau. Các kỹ thuật như kiểm thử hợp
đồng và môi trường kiểm thử tích hợp được sử dụng để giải quyết vấn đề này.

\subsection{Các nguyên tắc thiết kế}
Để thiết kế một kiến trúc microservice hiệu quả, một số nguyên tắc thiết kế chính cần được tuân thủ. Nguyên tắc đầu tiên là Single Responsibility Principle (Nguyên tắc Trách nhiệm Đơn lẻ), theo đó mỗi dịch vụ nên chịu trách nhiệm cho một chức năng nghiệp vụ duy nhất. Điều này giúp giữ các dịch vụ đơn giản và tập trung, dễ hiểu và bảo trì.

Domain-Driven Design (DDD) là một phương pháp thiết kế hữu ích cho kiến trúc
microservice. DDD sử dụng các khái niệm như Bounded Context để định nghĩa ranh
giới giữa các dịch vụ. Bounded Context giúp xác định phạm vi trách nhiệm của
mỗi dịch vụ và cách chúng tương tác với nhau.

API First là một nguyên tắc khác, nhấn mạnh việc thiết kế API trước, xem nó như
một hợp đồng giữa các dịch vụ. Điều này giúp đảm bảo rằng các dịch vụ có thể
giao tiếp hiệu quả và rằng các thay đổi không phá vỡ tương thích ngược.

Tự động hóa là một phần quan trọng của kiến trúc microservice thành công. Tự
động hóa quá trình xây dựng, kiểm thử và triển khai giúp quản lý sự phức tạp
của việc phát triển và vận hành nhiều dịch vụ. Các công cụ CI/CD (Continuous
Integration/Continuous Deployment) là rất quan trọng trong môi trường
microservice.

Monitoring và Observability là các nguyên tắc quan trọng khác. Thiết kế hệ
thống để dễ dàng giám sát và hiểu được hoạt động nội bộ giúp phát hiện và giải
quyết vấn đề một cách nhanh chóng. Các công cụ như logging tập trung, theo dõi
phân tán và thu thập số liệu là rất quan trọng.

Cuối cùng, Fault Tolerance (Khả năng chịu lỗi) là một nguyên tắc thiết kế quan
trọng cho kiến trúc microservice. Các dịch vụ nên được thiết kế để xử lý lỗi
một cách thanh nhã, sử dụng các kỹ thuật như Circuit Breaker. Circuit Breaker
ngăn lỗi lan truyền bằng cách ngừng gửi yêu cầu đến các dịch vụ không phản hồi.