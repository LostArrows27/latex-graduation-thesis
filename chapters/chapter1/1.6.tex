\section{Cấu trúc khóa luận}
Khóa luận được tổ chức thành 5 chương chính, mỗi chương tập trung vào một khía cạnh cụ thể của nghiên cứu. 

Chương 1, Mở đầu, giới thiệu bối
cảnh và sự cần thiết của đề tài, xác định mục tiêu, phạm vi và phương pháp
nghiên cứu, và trình bày ý nghĩa khoa học và thực tiễn của nghiên cứu. Chương 2, Kiến thức nền tảng, cung cấp tổng quan về kiến trúc vi dịch vụ, khái niệm và
phân loại các mẫu giao tiếp, và giới thiệu các công nghệ và giao thức giao tiếp
phổ biến. Chương 3, Phân tích các mẫu giao tiếp, phân loại các mẫu giao tiếp
theo tiêu chí đồng bộ/bất đồng bộ và one-to-one/one-to-many, phân tích chi tiết
các mẫu giao tiếp đồng bộ (one-to-one), phân tích chi tiết các mẫu giao tiếp
bất đồng bộ (one-to-one), phân tích chi tiết các mẫu giao tiếp bất đồng bộ
(one-to-many), và so sánh và đánh giá các mẫu giao tiếp. Chương 4, Triển khai
thử nghiệm, mô tả bài toán và yêu cầu, thiết kế và cài đặt ứng dụng mẫu, cài
đặt và triển khai các mẫu giao tiếp, và kết quả triển khai và đánh giá hiệu
năng. Chương 5, Đánh giá và thảo luận, phân tích kết quả thực nghiệm, thảo luận
về các phát hiện chính, đề xuất các nguyên tắc lựa chọn mẫu giao tiếp, và kết
luận và hướng phát triển.