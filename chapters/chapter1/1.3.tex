\section{Phạm vi nghiên cứu}
Để đảm bảo tính khả thi và giá trị thực tiễn của khóa luận, phạm vi đánh giá được giới hạn như sau:

Về nội dung, khóa luận tập trung vào các mẫu giao tiếp phổ biến trong kiến trúc microservice. Các mẫu này được lựa chọn không chỉ vì tính phổ biến của chúng mà còn vì chúng đại diện cho các phương pháp tiếp cận khác nhau trong việc giải quyết vấn đề giao tiếp giữa các dịch vụ. Cụ thể, khóa luận sẽ tập trung vào giao tiếp đồng bộ thông qua REST API, giao tiếp bất đồng bộ one-to-one thông qua Message Queue (RabbitMQ), và giao tiếp bất đồng bộ one-to-many thông qua Pub/Sub (Kafka) \cite{aksakalli2021}.

Các khía cạnh được khóa luận phân tích bao gồm mô hình giao tiếp và luồng dữ liệu, mô tả cách thức các dịch vụ tương tác và trao đổi thông tin với nhau. Đo lường hiệu năng và tối ưu hóa là một khía cạnh quan trọng khác, bao gồm việc đánh giá hiệu suất của các mẫu giao tiếp và cách thức tối ưu hóa hiệu suất. Cuối cùng, khóa luận xem xét tính mở rộng và khả năng chịu tải của các mẫu giao tiếp, đánh giá khả năng của chúng trong việc hỗ trợ hệ thống mở rộng khi nhu cầu tăng lên.

Khóa luận không đi sâu vào chi tiết kỹ thuật triển khai của từng công nghệ cụ thể, vì điều này nằm ngoài phạm vi của một khóa luận tổng quan về các mẫu giao tiếp. Các vấn đề bảo mật và quản lý danh tính được đề cập nhưng không phải là trọng tâm, vì chúng đòi hỏi một phân tích chuyên sâu riêng biệt. Tương tự, các vấn đề liên quan đến cơ sở dữ liệu và quản lý trạng thái cũng nằm ngoài phạm vi chính của khóa luận này.

Về thực nghiệm, khóa luận sẽ xây dựng ứng dụng mẫu với 4 kịch bản thực tế để đánh giá hiệu quả của các mẫu giao tiếp khác nhau. Kịch bản đầu tiên là Order-Inventory, liên quan đến việc kiểm tra và cập nhật tồn kho. Đây là một tình huống phổ biến trong các hệ thống thương mại điện tử, đòi hỏi tính nhất quán dữ liệu cao. Kịch bản thứ hai là Order-Payment, liên quan đến việc xử lý thanh toán. Quá trình thanh toán thường đòi hỏi độ tin cậy cao và cần xử lý các tình huống lỗi một cách cẩn thận. Kịch bản thứ ba là Order-Notification, liên quan đến việc phân phối thông báo. Việc gửi thông báo thường yêu cầu một cơ chế giao tiếp một-đến-nhiều, phản ánh nhu cầu thông báo cho nhiều đối tượng khác nhau. Kịch bản cuối cùng là User Activity Logging, liên quan đến việc ghi nhận hoạt động người dùng. Đây là một tình huống cần khả năng xử lý lượng lớn các sự kiện, đòi hỏi một cơ chế giao tiếp có khả năng chịu tải cao.

Các tiêu chí đánh giá trong thực nghiệm bao gồm độ trễ trung bình và percentile thứ 95, phản ánh thời gian phản hồi trung bình và trong trường hợp xấu. Thông lượng là một chỉ số khác, đo lường số lượng giao dịch có thể xử lý trong một đơn vị thời gian. Tỷ lệ lỗi được theo dõi để đánh giá độ tin cậy của các mẫu giao tiếp. Tính nhất quán dữ liệu, khả năng phục hồi sau lỗi, và khả năng mở rộng theo chiều ngang cũng là những tiêu chí quan trọng được đánh giá. 