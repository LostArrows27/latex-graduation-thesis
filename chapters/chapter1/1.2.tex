\section{Mục tiêu nghiên cứu}
Khóa luận này hướng đến việc phân tích và đánh giá các mẫu giao tiếp trong kiến
trúc microservice nhằm cung cấp cơ sở lý thuyết và thực tiễn cho việc lựa chọn,
thiết kế và triển khai các cơ chế giao tiếp hiệu quả. Việc hiểu rõ và áp dụng
đúng các mẫu giao tiếp không chỉ giúp tối ưu hóa hiệu suất hệ thống mà còn đảm
bảo tính linh hoạt và khả năng mở rộng - những yếu tố then chốt trong thành
công của các hệ thống microservice hiện đại.

Cụ thể, khóa luận hướng đến việc phân loại và hệ thống hóa các mẫu giao tiếp
trong microservice theo tiêu chí giao tiếp đồng bộ/bất đồng bộ và mô hình
one-to-one/one-to-many. Việc phân loại này giúp tạo ra một khung tham chiếu rõ
ràng, từ đó người đọc có thể dễ dàng định vị và hiểu các mẫu giao tiếp phù hợp
với nhu cầu cụ thể của họ. Khung phân loại này cũng phản ánh các đặc tính cơ
bản của giao tiếp giữa các dịch vụ, bao gồm thời gian (đồng bộ hay bất đồng
bộ) và phạm vi (một đối một hay một đối nhiều).

Đồng thời, khóa luận sẽ phân tích chi tiết và so sánh các mẫu giao tiếp trên nhiều khía cạnh. Về cơ chế hoạt động, khóa luận sẽ mô tả chi tiết cách thức các mẫu giao tiếp hoạt động, bao gồm các thành phần, luồng dữ liệu và tương tác giữa các dịch vụ. Về hiệu suất và độ trễ, khóa luận sẽ đánh giá thời gian phản hồi, thông lượng và khả năng xử lý đồng thời của các mẫu giao tiếp. Về khả năng mở rộng, khóa luận sẽ phân tích khả năng của các mẫu giao tiếp trong việc hỗ trợ việc mở rộng số lượng dịch vụ và lưu lượng giao dịch. Về độ tin cậy và khả năng chịu lỗi, khóa luận sẽ đánh giá khả năng của các mẫu giao tiếp trong việc duy trì hoạt động ổn định khi gặp lỗi hoặc sự cố. Về độ phức tạp trong triển khai và bảo trì, khóa luận sẽ xem xét mức độ phức tạp và nguồn lực cần thiết để triển khai và duy trì các mẫu giao tiếp. Cuối cùng, về tính phù hợp với các tình huống cụ thể, khóa luận sẽ xác định các ngữ cảnh và yêu cầu mà mỗi mẫu giao tiếp phù hợp nhất.

Để đánh giá các mẫu giao tiếp một cách khách quan, khóa luận sẽ xây dựng và triển khai môi trường thử nghiệm mô phỏng các kịch bản thực tế. Các kịch bản này bao gồm kiểm tra và cập nhật tồn kho, xử lý thanh toán, phân phối thông báo, và ghi nhận hoạt động người dùng. Những kịch bản này được chọn vì chúng đại diện cho các tình huống phổ biến trong các ứng dụng thực tế và đòi hỏi các đặc tính giao tiếp khác nhau. Thông qua việc triển khai và đánh giá các mẫu giao tiếp trong các kịch bản này, khóa luận có thể cung cấp một cái nhìn thực tế về hiệu quả của từng mẫu giao tiếp.

Khóa luận sẽ đo lường và phân tích hiệu năng của các mẫu giao tiếp trong điều
kiện khác nhau về tải và độ trễ mạng. Việc đo lường này bao gồm các chỉ số như
thời gian phản hồi, thông lượng, tỷ lệ lỗi, tính nhất quán dữ liệu, và khả năng
phục hồi sau lỗi. Thông qua việc phân tích các chỉ số này, khóa luận có thể xác
định các điểm mạnh và điểm yếu của từng mẫu giao tiếp trong các điều kiện khác
nhau.

Kết quả cuối cùng của khóa luận là tổng hợp các nguyên tắc và hướng dẫn thực
tiễn cho việc lựa chọn mẫu giao tiếp phù hợp dựa trên yêu cầu cụ thể của ứng
dụng. Các hướng dẫn này sẽ giúp các nhà phát triển và kiến trúc sư hệ thống đưa
ra quyết định sáng suốt về cách thức các dịch vụ giao tiếp với nhau, từ đó tối
ưu hóa hiệu suất, độ tin cậy và khả năng mở rộng của hệ thống microservice.