\section{Bối cảnh và sự cần thiết của đề tài}
Trong bối cảnh phát triển phần mềm hiện đại, các hệ thống ngày càng trở nên
phức tạp và đòi hỏi khả năng mở rộng cao để đáp ứng nhu cầu kinh doanh không
ngừng thay đổi. Kiến trúc microservice đã nổi lên như một giải pháp hiệu quả,
cho phép các nhóm phát triển độc lập tạo ra, triển khai và mở rộng các dịch vụ
nhỏ, tập trung vào một chức năng cụ thể của hệ thống. Sự phân tách này tạo điều
kiện cho việc phát triển nhanh chóng và linh hoạt, giúp các tổ chức thích ứng
tốt hơn với các yêu cầu thay đổi của thị trường.

Theo một báo cáo của IDC, đến năm 2025, hơn 80\% các tổ chức doanh nghiệp sẽ
chuyển đổi sang kiến trúc phân tán như microservice để tăng tốc độ phát triển
và triển khai ứng dụng \cite{idc2021}. Xu hướng này phản ánh nhu cầu ngày càng tăng về khả
năng mở rộng, tính linh hoạt và tốc độ phát triển trong môi trường kinh doanh
cạnh tranh. Tuy nhiên, việc phân tách một hệ thống thành nhiều dịch vụ nhỏ cũng
đặt ra thách thức lớn về cách thức các dịch vụ này giao tiếp với nhau.

Trong kiến trúc microservice, giao tiếp giữa các dịch vụ là một yếu tố then
chốt quyết định đến hiệu suất tổng thể của hệ thống. Mô hình giao tiếp không
phù hợp có thể tạo ra các điểm nghẽn, làm tăng độ trễ và giảm khả năng đáp ứng
của hệ thống. Khi nhiều microservice phải tương tác với nhau để hoàn thành một
nhiệm vụ, hiệu suất của toàn bộ chuỗi dịch vụ có thể bị ảnh hưởng bởi thời gian
phản hồi của dịch vụ chậm nhất hoặc khả năng xử lý thấp nhất trong chuỗi.

Về độ tin cậy, trong một hệ thống phân tán, các lỗi giao tiếp có thể xảy ra ở
nhiều điểm, ảnh hưởng đến tính nhất quán và độ sẵn sàng của dịch vụ. Mạng không
ổn định, dịch vụ quá tải, hoặc các sự cố không lường trước đều có thể dẫn đến
lỗi giao tiếp. Các cơ chế phát hiện lỗi, xử lý lỗi và khôi phục là rất quan
trọng để duy trì tính khả dụng của hệ thống.

Các mẫu giao tiếp phải hỗ trợ việc mở rộng số lượng dịch vụ và lưu lượng giao
dịch mà không làm giảm hiệu suất, đồng thời đảm bảo hệ thống hoạt động ổn định
ngay cả khi một số dịch vụ gặp sự cố. Khả năng mở rộng là đặc biệt quan trọng
khi doanh nghiệp phát triển và nhu cầu về hệ thống tăng lên. Các mẫu giao tiếp
không chỉ phải xử lý được lưu lượng hiện tại mà còn phải có khả năng thích ứng
với sự tăng trưởng trong tương lai.

Thực tế cho thấy, việc lựa chọn cơ chế giao tiếp không phù hợp có thể dẫn đến
các vấn đề nghiêm trọng. Theo nghiên cứu của Gartner, hơn 70\% các dự án
microservice gặp khó khăn trong giai đoạn đầu triển khai do thiếu hiểu biết về
các mô hình giao tiếp và cách thức áp dụng chúng hiệu quả \cite{gartner2019}. Nhiều tổ chức đã
phải thiết kế lại kiến trúc của họ sau khi gặp phải các vấn đề về hiệu suất, độ
tin cậy và khả năng quản lý.

Các tổ chức phải đối mặt với sự đánh đổi giữa tính nhất quán và hiệu suất, giữa
độ tin cậy và độ phức tạp. Những quyết định về cơ chế giao tiếp có ảnh hưởng
sâu sắc đến kiến trúc tổng thể và thành công của hệ thống. Ví dụ, giao tiếp
đồng bộ có thể đơn giản và dễ triển khai, nhưng có thể ảnh hưởng đến hiệu suất
và khả năng chịu lỗi. Ngược lại, giao tiếp bất đồng bộ có thể cải thiện khả
năng mở rộng và chịu lỗi, nhưng làm tăng độ phức tạp trong việc quản lý trạng
thái và đảm bảo tính nhất quán của dữ liệu \cite{newman2015}.

Do đó, việc phân tích và đánh giá các mẫu giao tiếp (communication patterns)
trong kiến trúc microservice là vô cùng cần thiết, giúp các nhà phát triển và
kiến trúc sư hệ thống hiểu rõ các lựa chọn có sẵn và những đánh đổi liên quan.
Thông qua việc phân tích có hệ thống, các nhà phát triển có thể áp dụng các mẫu
phù hợp với bối cảnh cụ thể, tối ưu hóa hiệu suất và độ tin cậy của hệ thống,
đồng thời đảm bảo khả năng mở rộng trong tương lai.

Bài đánh giá này nhằm cung cấp một cái nhìn toàn diện về các mẫu giao tiếp
trong kiến trúc microservice, phân tích ưu nhược điểm của từng mẫu, và đưa ra
các hướng dẫn thực tiễn cho việc lựa chọn và triển khai các mẫu giao tiếp phù
hợp.
