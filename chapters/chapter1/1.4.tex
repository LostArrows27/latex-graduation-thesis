\section{Phương pháp nghiên cứu}
Khóa luận kết hợp phương pháp nghiên cứu lý thuyết và thực nghiệm để đánh giá toàn diện các mẫu giao tiếp trong kiến trúc vi dịch vụ.

Về phương pháp lý thuyết, khóa luận phân tích tài liệu học thuật, báo cáo kỹ thuật từ các nguồn uy tín như IEEE, ACM, và các tài liệu từ công ty công nghệ hàng đầu như Netflix và Uber. Khóa luận thực hiện phân tích so sánh các mẫu giao tiếp dựa trên các tiêu chí định lượng và định tính như hiệu suất, độ tin cậy, khả năng mở rộng, và phân tích các trường hợp thực tế về triển khai mẫu giao tiếp trong các tổ chức lớn.

Về phương pháp thực nghiệm, khóa luận phát triển ứng dụng vi dịch vụ mẫu theo kiến trúc tham chiếu, mô phỏng các tình huống thực tế như quản lý đơn hàng, thanh toán, gửi thông báo và ghi nhận hoạt động người dùng. Khóa luận thiết kế các kịch bản thử nghiệm mô phỏng các điều kiện tải khác nhau, thu thập dữ liệu về hiệu suất, độ tin cậy và khả năng mở rộng, và áp dụng phương pháp thống kê để phân tích dữ liệu.

Quy trình đánh giá gồm 5 giai đoạn: Giai đoạn 1 - nghiên cứu lý thuyết và tổng hợp tài liệu; Giai đoạn 2 - thiết kế và phát triển ứng dụng mẫu; Giai đoạn 3 - thiết lập môi trường và thực hiện thử nghiệm; Giai đoạn 4 - phân tích dữ liệu và đánh giá hiệu quả các mẫu giao tiếp; Giai đoạn 5 - tổng hợp kết quả và xây dựng hướng dẫn thực tiễn.

Phương pháp nghiên cứu này đảm bảo cung cấp cái nhìn toàn diện về hiệu quả của các mẫu giao tiếp trong kiến trúc vi dịch vụ, từ lý thuyết đến thực tiễn, từ định tính đến định lượng.