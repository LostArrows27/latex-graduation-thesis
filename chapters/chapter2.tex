\chapter{Kiến thức nền tảng}

Chương 2, Kiến thức nền tảng, cung cấp nền tảng kiến thức toàn diện về kiến trúc vi dịch vụ và các cơ chế giao tiếp trong môi trường phân tán. Chương này bắt đầu với việc giới thiệu tổng quan về kiến trúc vi dịch vụ, bao gồm định nghĩa, đặc điểm nổi bật và so sánh với kiến trúc truyền thống. Tiếp theo, chương trình bày các lợi ích và thách thức khi áp dụng kiến trúc vi dịch vụ, cùng với các nguyên tắc thiết kế quan trọng để xây dựng hệ thống vi dịch vụ hiệu quả. Phần thứ hai của chương tập trung vào khái niệm và phân loại các mẫu giao tiếp trong kiến trúc vi dịch vụ, phân tích các đặc điểm, ưu nhược điểm của từng loại. Chương kết thúc với việc giới thiệu các công nghệ và giao thức giao tiếp phổ biến, cung cấp cái nhìn thực tế về cách triển khai các mẫu giao tiếp trong các hệ thống thực tế.

\section{Tổng quan về Microservice Architecture}

\subsection{Định nghĩa và đặc điểm}
Kiến trúc Microservice là một phương pháp phát triển phần mềm trong đó một ứng
dụng được cấu thành từ nhiều dịch vụ nhỏ, độc lập và có khả năng triển khai
riêng biệt. Mỗi dịch vụ này được thiết kế để thực hiện một chức năng cụ thể
trong phạm vi nghiệp vụ được định nghĩa rõ ràng, và giao tiếp với các dịch vụ
khác thông qua các cơ chế giao tiếp nhẹ, thường là API.

Theo Newman (2021), các đặc điểm chính của kiến trúc microservice bao gồm tính
tự trị cao, trong đó mỗi dịch vụ có thể được phát triển, triển khai và mở rộng
độc lập với các dịch vụ khác. Các dịch vụ được tổ chức xoay quanh các khả năng
nghiệp vụ thay vì các lớp công nghệ, thể hiện sự phân tách theo chức năng
nghiệp vụ. Quản lý dữ liệu trong microservice được thực hiện phi tập trung, với
mỗi dịch vụ quản lý dữ liệu riêng và chỉ có thể truy cập dữ liệu thông qua API
của dịch vụ sở hữu dữ liệu đó.

Thiết kế hướng lỗi là một đặc điểm quan trọng khác của microservice, trong đó
các dịch vụ được thiết kế để xử lý lỗi và khả năng các dịch vụ khác không khả
dụng. Cuối cùng, microservice cho phép tiến hóa độc lập, nghĩa là các dịch vụ
có thể thay đổi và phát triển theo thời gian mà không ảnh hưởng đến toàn bộ hệ
thống.

\subsection{So sánh với kiến trúc nguyên khối (Monolithic)}
Để hiểu rõ hơn về kiến trúc microservice, việc so sánh với kiến trúc nguyên khối là rất hữu ích. Trong kiến trúc nguyên khối, toàn bộ ứng dụng được xây dựng như một đơn vị duy nhất. Tất cả các chức năng nằm trong một codebase và được triển khai cùng nhau.

Về triển khai, kiến trúc nguyên khối đòi hỏi toàn bộ ứng dụng được triển khai
cùng một lúc, trong khi kiến trúc microservice cho phép các dịch vụ được triển
khai độc lập. Điều này có ý nghĩa quan trọng trong việc giảm thiểu rủi ro và
tăng tốc độ phát hành.

Khả năng mở rộng cũng khác biệt đáng kể giữa hai kiến trúc. Trong kiến trúc
nguyên khối, toàn bộ ứng dụng phải được mở rộng, ngay cả khi chỉ một phần cần
thêm tài nguyên. Ngược lại, kiến trúc microservice cho phép mở rộng từng dịch
vụ riêng biệt, tối ưu hóa việc sử dụng tài nguyên.

Về phát triển, kiến trúc nguyên khối thường có một nhóm phát triển làm việc
trên một codebase, dẫn đến các xung đột trong quá trình phát triển và triển
khai. Trong khi đó, kiến trúc microservice cho phép nhiều nhóm làm việc độc lập
trên các dịch vụ khác nhau, tăng tốc độ phát triển và giảm thiểu xung đột.

Công nghệ là một khía cạnh khác có sự khác biệt. Kiến trúc nguyên khối thường
bị giới hạn trong một stack công nghệ, trong khi mỗi microservice có thể sử
dụng công nghệ phù hợp nhất với yêu cầu của nó. Điều này tạo ra sự linh hoạt và
khả năng thích ứng với các công nghệ mới.

Khả năng chịu lỗi cũng là một điểm khác biệt quan trọng. Trong kiến trúc nguyên
khối, lỗi ở một phần có thể ảnh hưởng đến toàn bộ ứng dụng, trong khi trong
kiến trúc microservice, lỗi được cô lập trong một dịch vụ, giảm thiểu tác động
đến toàn bộ hệ thống.

Cuối cùng, về độ phức tạp, kiến trúc nguyên khối đơn giản hơn trong các ứng
dụng nhỏ, nhưng phức tạp hơn khi ứng dụng phát triển. Ngược lại, kiến trúc
microservice phức tạp hơn ngay từ đầu do tính phân tán, nhưng độ phức tạp này
được quản lý tốt hơn khi hệ thống phát triển.

\subsection{Lợi ích và thách thức của kiến trúc microservice}
Kiến trúc microservice mang lại nhiều lợi ích đáng kể cho việc phát triển và
vận hành phần mềm. Một trong những lợi ích chính là khả năng mở rộng có mục
tiêu. Các dịch vụ có thể được mở rộng độc lập dựa trên nhu cầu, tối ưu hóa việc
sử dụng tài nguyên. Điều này đặc biệt quan trọng trong môi trường cloud, nơi
chi phí tỷ lệ thuận với tài nguyên được sử dụng.

Phát triển nhanh hơn là một lợi ích khác của kiến trúc microservice. Các nhóm
nhỏ có thể làm việc trên các dịch vụ độc lập, cho phép phát triển song song và
chu kỳ phát hành nhanh hơn. Mỗi nhóm có thể tập trung vào một dịch vụ cụ thể,
hiểu rõ nó và phát triển nó một cách hiệu quả.

Tính linh hoạt công nghệ cũng là một lợi thế đáng kể. Mỗi dịch vụ có thể sử
dụng công nghệ phù hợp nhất với yêu cầu của nó. Ví dụ, một dịch vụ xử lý giao
dịch có thể sử dụng một ngôn ngữ chú trọng vào tính nhất quán, trong khi một
dịch vụ phân tích dữ liệu có thể sử dụng một ngôn ngữ tối ưu cho xử lý dữ liệu
lớn.

Khả năng chịu lỗi tốt hơn là một lợi ích khác của kiến trúc microservice. Lỗi
trong một dịch vụ không nhất thiết phải làm cho toàn bộ hệ thống không khả
dụng. Ví dụ, nếu dịch vụ gợi ý sản phẩm không hoạt động, người dùng vẫn có thể
duyệt và mua sản phẩm.

Khả năng bảo trì và hiểu biết tốt hơn cũng là một lợi thế của kiến trúc
microservice. Các dịch vụ nhỏ hơn dễ hiểu và bảo trì hơn các ứng dụng lớn. Mã
nguồn của mỗi dịch vụ nhỏ hơn và tập trung vào một chức năng cụ thể, giúp nhà
phát triển dễ dàng hiểu và thay đổi nó.

Tuy nhiên, kiến trúc microservice cũng đặt ra một số thách thức đáng kể. Độ
phức tạp phân tán là một thách thức lớn. Hệ thống phân tán vốn phức tạp hơn,
đòi hỏi kiến thức và công cụ chuyên biệt. Các vấn đề như latency mạng, xử lý
lỗi và đồng bộ hóa dữ liệu trở nên phức tạp hơn trong một hệ thống phân tán.

Giao tiếp giữa các dịch vụ là một thách thức khác. Thiết kế và quản lý giao
tiếp giữa các dịch vụ đòi hỏi cân nhắc kỹ lưỡng về hiệu suất, độ tin cậy và khả
năng mở rộng. Việc lựa chọn giao thức giao tiếp phù hợp và xử lý các trường hợp
lỗi trong giao tiếp là các vấn đề phức tạp.

Quản lý dữ liệu cũng là một thách thức đáng kể trong kiến trúc microservice.
Duy trì tính nhất quán dữ liệu giữa các dịch vụ có thể phức tạp, đặc biệt là
khi mỗi dịch vụ có cơ sở dữ liệu riêng. Các mẫu như Saga và Event Sourcing được
sử dụng để giải quyết vấn đề này, nhưng chúng cũng đưa ra sự phức tạp riêng.

Vận hành và giám sát là một thách thức khác của kiến trúc microservice. Triển
khai và giám sát nhiều dịch vụ đòi hỏi công cụ và quy trình tinh vi hơn. Các
công cụ như Kubernetes và Prometheus đã được phát triển để giải quyết vấn đề
này, nhưng chúng cũng đòi hỏi kiến thức và nỗ lực đáng kể để sử dụng hiệu quả.

Cuối cùng, kiểm thử cũng trở nên phức tạp hơn trong kiến trúc microservice.
Kiểm thử tích hợp đòi hỏi sự phối hợp giữa nhiều dịch vụ, có thể chạy trên các
máy khác nhau và sử dụng các công nghệ khác nhau. Các kỹ thuật như kiểm thử hợp
đồng và môi trường kiểm thử tích hợp được sử dụng để giải quyết vấn đề này.

\subsection{Các nguyên tắc thiết kế}
Để thiết kế một kiến trúc microservice hiệu quả, một số nguyên tắc thiết kế chính cần được tuân thủ. Nguyên tắc đầu tiên là Single Responsibility Principle (Nguyên tắc Trách nhiệm Đơn lẻ), theo đó mỗi dịch vụ nên chịu trách nhiệm cho một chức năng nghiệp vụ duy nhất. Điều này giúp giữ các dịch vụ đơn giản và tập trung, dễ hiểu và bảo trì.

Domain-Driven Design (DDD) là một phương pháp thiết kế hữu ích cho kiến trúc
microservice. DDD sử dụng các khái niệm như Bounded Context để định nghĩa ranh
giới giữa các dịch vụ. Bounded Context giúp xác định phạm vi trách nhiệm của
mỗi dịch vụ và cách chúng tương tác với nhau.

API First là một nguyên tắc khác, nhấn mạnh việc thiết kế API trước, xem nó như
một hợp đồng giữa các dịch vụ. Điều này giúp đảm bảo rằng các dịch vụ có thể
giao tiếp hiệu quả và rằng các thay đổi không phá vỡ tương thích ngược.

Tự động hóa là một phần quan trọng của kiến trúc microservice thành công. Tự
động hóa quá trình xây dựng, kiểm thử và triển khai giúp quản lý sự phức tạp
của việc phát triển và vận hành nhiều dịch vụ. Các công cụ CI/CD (Continuous
Integration/Continuous Deployment) là rất quan trọng trong môi trường
microservice.

Monitoring và Observability là các nguyên tắc quan trọng khác. Thiết kế hệ
thống để dễ dàng giám sát và hiểu được hoạt động nội bộ giúp phát hiện và giải
quyết vấn đề một cách nhanh chóng. Các công cụ như logging tập trung, theo dõi
phân tán và thu thập số liệu là rất quan trọng.

Cuối cùng, Fault Tolerance (Khả năng chịu lỗi) là một nguyên tắc thiết kế quan
trọng cho kiến trúc microservice. Các dịch vụ nên được thiết kế để xử lý lỗi
một cách thanh nhã, sử dụng các kỹ thuật như Circuit Breaker. Circuit Breaker
ngăn lỗi lan truyền bằng cách ngừng gửi yêu cầu đến các dịch vụ không phản hồi.
\section{Giao tiếp trong kiến trúc vi dịch vụ}

\subsection{Vai trò của giao tiếp trong kiến trúc vi dịch vụ}
Giao tiếp đóng vai trò then chốt trong kiến trúc vi dịch vụ. Khác với các ứng
dụng nguyên khối, các vi dịch vụ giao tiếp qua mạng, thường thông qua HTTP, gRPC hoặc middleware messaging.

Giao tiếp không chỉ trao đổi dữ liệu, mà còn định hình toàn bộ kiến trúc và ảnh hưởng đến tính khả dụng, hiệu suất và khả năng mở rộng của hệ thống \cite{wolff2016}.

Giao tiếp tạo điều kiện cho các dịch vụ phối hợp hoàn thành các tác vụ nghiệp vụ phức tạp. Ví dụ, quy trình đặt hàng liên quan đến các dịch vụ quản lý đơn hàng, thanh toán, kho hàng và vận chuyển cần giao tiếp để đảm bảo xử lý chính xác.

Giao tiếp đóng vai trò quan trọng trong việc đảm bảo tính nhất quán dữ liệu. Trong hệ thống dữ liệu phân tán, mỗi dịch vụ quản lý một phần dữ liệu, cần giao tiếp khi dữ liệu thay đổi để duy trì nhất quán.

Giao tiếp hỗ trợ khả năng chịu lỗi thông qua các cơ chế như Circuit Breaker, giúp hệ thống phục hồi và tiếp tục hoạt động khi gặp lỗi.

Cuối cùng, thiết kế giao tiếp tốt cho phép mở rộng hệ thống, thêm dịch vụ mới hoặc phiên bản mới mà không ảnh hưởng đến dịch vụ khác.

\subsection{Các thuộc tính quan trọng của giao tiếp vi dịch vụ}
Khi thiết kế giao tiếp cho vi dịch vụ, cần xem xét một số thuộc tính quan trọng:

Độ tin cậy đảm bảo thông điệp được gửi và nhận thành công trong hệ thống phân tán. Các cơ chế như xác nhận, thử lại và hàng đợi bền vững giúp tăng độ tin cậy \cite{hohpe2004}.

Độ trễ là thời gian thông điệp di chuyển từ nguồn đến đích. Các yếu tố ảnh hưởng bao gồm khoảng cách vật lý giữa dịch vụ, phương pháp tuần tự hóa và tải mạng.

Khả năng mở rộng giúp xử lý khối lượng thông điệp tăng khi hệ thống phát triển, thông qua việc thêm nhiều phiên bản dịch vụ.

Cách ly lỗi ngăn lỗi lan truyền giữa các dịch vụ. Các mẫu như Circuit Breaker và Bulkhead hạn chế ảnh hưởng của lỗi từ một dịch vụ đến các dịch vụ khác.

Tính nhất quán liên quan đến cách đảm bảo dữ liệu nhất quán. Trong hệ thống phân tán, có sự đánh đổi giữa tính nhất quán, khả năng sẵn sàng và khả năng chịu đựng phân vùng.

Định dạng dữ liệu đề cập cách dữ liệu được cấu trúc và tuần tự hóa. Các định dạng phổ biến như JSON, XML và Protocol Buffers có ưu nhược điểm riêng.

Khả năng tương tác cho phép các dịch vụ sử dụng công nghệ khác nhau giao tiếp với nhau, quan trọng trong môi trường đa ngôn ngữ và đa nền tảng.

Bảo mật bảo vệ thông điệp khỏi truy cập trái phép qua mã hóa, xác thực và ủy quyền khi dịch vụ giao tiếp qua mạng.

\subsection{Các mô hình giao tiếp cơ bản}
Có hai mô hình giao tiếp cơ bản trong vi dịch vụ: đồng bộ và bất đồng bộ.

Trong giao tiếp đồng bộ, người gửi đợi phản hồi trước khi tiếp tục xử lý. Ví dụ, một dịch vụ gửi yêu cầu HTTP đợi phản hồi trước khi tiếp tục. Mô hình này đơn giản, dễ hiểu và cung cấp phản hồi tức thì \cite{newman2015}. Tuy nhiên, có thể dẫn đến hiệu suất kém vì phải đợi phản hồi, hiệu ứng xung hồi khi dịch vụ chậm, và vấn đề khả năng mở rộng khi số yêu cầu tăng.

Trong giao tiếp bất đồng bộ, người gửi không đợi phản hồi. Ví dụ, dịch vụ gửi thông điệp vào hàng đợi rồi tiếp tục xử lý. Mô hình này tạo coupling lỏng lẻo và khả năng đệm tốt \cite{hohpe2004}. Tuy nhiên, phức tạp hơn để triển khai, có thể có độ trễ cao, và gây khó khăn về tính nhất quán dữ liệu.

\subsection{Kiểu tương tác}
Ngoài mô hình giao tiếp đồng bộ và bất đồng bộ, các vi dịch vụ thường tương tác theo hai kiểu chính:

Kiểu one-to-one (một-một): Trong mô hình này, một dịch vụ giao tiếp trực tiếp với một dịch vụ khác. Đây là kiểu tương tác phổ biến nhất trong kiến trúc vi dịch vụ, thường được triển khai thông qua các REST API hoặc RPC. Ví dụ điển hình là dịch vụ đơn hàng gọi dịch vụ thanh toán để xử lý giao dịch, hoặc dịch vụ người dùng truy vấn dịch vụ xác thực để kiểm tra quyền truy cập. Mô hình này đơn giản, dễ triển khai và quản lý, nhưng có thể tạo ra sự phụ thuộc chặt chẽ giữa các dịch vụ.

Kiểu one-to-many (một-nhiều): Trong kiểu tương tác này, một dịch vụ gửi thông điệp hoặc sự kiện đến nhiều dịch vụ khác. Mô hình này thường được triển khai thông qua các cơ chế publish/subscribe (phát hành/đăng ký) sử dụng message broker hoặc event bus. Ví dụ tiêu biểu là khi dịch vụ đơn hàng phát hành sự kiện "đơn hàng đã tạo", các dịch vụ khác như kho hàng, vận chuyển, thông báo và phân tích đều nhận được thông báo này và xử lý tương ứng. Mô hình one-to-many tạo ra sự kết nối lỏng lẻo giữa các dịch vụ, cải thiện khả năng mở rộng và linh hoạt của hệ thống, nhưng cũng làm tăng độ phức tạp trong việc đảm bảo tính nhất quán và theo dõi luồng dữ liệu.

Sự kết hợp giữa mô hình giao tiếp (đồng bộ/bất đồng bộ) và kiểu tương tác (một-một/một-nhiều) tạo ra các mẫu giao tiếp khác nhau, mỗi mẫu phù hợp với các tình huống sử dụng cụ thể trong kiến trúc vi dịch vụ.

\subsection{Các công nghệ và giao thức phổ biến}
HTTP/REST là giao thức phổ biến nhất cho giao tiếp đồng bộ, sử dụng phương thức HTTP và tài nguyên đại diện. REST đơn giản, dễ hiểu và được hỗ trợ rộng rãi.

gRPC là framework RPC hiệu suất cao, sử dụng HTTP/2 và Protocol Buffers. Cung cấp hiệu suất tốt hơn REST nhờ multiplexing và định dạng tuần tự hóa hiệu quả, hỗ trợ streaming hai chiều.

Message Queue là mẫu bất đồng bộ với dịch vụ gửi và nhận thông điệp qua hàng đợi như RabbitMQ, ActiveMQ và AWS SQS. Cung cấp coupling lỏng lẻo và khả năng đệm tốt, nhưng thêm độ phức tạp và độ trễ.

Pub/Sub là mẫu bất đồng bộ với nhà xuất bản gửi thông điệp mà không biết người nhận, thường qua Apache Kafka, AWS SNS/SQS, Google Pub/Sub hoặc NATS. Cung cấp coupling lỏng lẻo cao và khả năng mở rộng tốt.

GraphQL là ngôn ngữ truy vấn và thao tác dữ liệu hiệu quả, cho phép chỉ định chính xác dữ liệu cần thiết, tránh over-fetching và under-fetching \cite{richardson2019}. Hữu ích cho ứng dụng di động với băng thông hạn chế.
 
\subsection{Thách thức trong giao tiếp vi dịch vụ}
Giao tiếp vi dịch vụ đặt ra nhiều thách thức:

Network Reliability là thách thức quan trọng khi mạng không đáng tin cậy, gây mất thông điệp hoặc độ trễ cao. Giải pháp bao gồm retry, timeout và circuit breaker.

Service Discovery giải quyết việc các dịch vụ tìm thấy nhau trong môi trường động. Các giải pháp gồm Client-side Discovery với service registry và Server-side Discovery qua load balancer hoặc API gateway.

Load Balancing phân phối tải giữa các phiên bản dịch vụ, đảm bảo hiệu suất và độ tin cậy. Các cơ chế gồm Round Robin, Least Connections và Hash-based.

Data Consistency là thách thức duy trì tính nhất quán dữ liệu trong giao tiếp vi dịch vụ. Các mẫu như Saga, Event Sourcing và CQRS \cite{richardson2019} giúp quản lý giao dịch phân tán, lưu trạng thái và tách biệt đọc/ghi.

Versioning quản lý thay đổi API giữa các dịch vụ. Các giải pháp gồm Semantic Versioning với hệ thống đánh số (MAJOR.MINOR.PATCH), API Versioning duy trì nhiều phiên bản API, và Backward Compatibility đảm bảo tương thích ngược.

Error Handling xử lý lỗi trong hệ thống phân tán. Các cơ chế gồm Retry thử lại yêu cầu thất bại, Circuit Breaker ngăn yêu cầu đến dịch vụ không phản hồi, và Fallback cung cấp phản hồi thay thế.

Monitoring and Debugging giám sát và gỡ lỗi trong hệ thống phức tạp. Các công cụ gồm Logging tập trung, Distributed Tracing theo dõi yêu cầu qua nhiều dịch vụ, và Metrics Collection thu thập chỉ số hiệu suất.

\subsection{Các mẫu giao tiếp}
Trong kiến trúc vi dịch vụ, các dịch vụ cần trao đổi thông tin để phối hợp và hoàn thành chức năng. Dưới đây là năm mẫu giao tiếp chính:

Request-Response là mẫu đồng bộ phổ biến nhất, dịch vụ gửi yêu cầu đến dịch vụ khác và đợi phản hồi. Dịch vụ gửi thiết lập kết nối HTTP/REST hoặc gRPC, chờ đợi phản hồi từ dịch vụ nhận. Mẫu này đơn giản, dễ hiểu và đảm bảo tính nhất quán dữ liệu cao. Tuy nhiên, tạo coupling chặt chẽ, hiệu suất kém khi độ trễ cao, và có nguy cơ lỗi cascade.

Event-Driven là mẫu các dịch vụ giao tiếp qua phát và lắng nghe sự kiện thông qua message broker. Dịch vụ phát hành không cần biết ai xử lý sự kiện, tạo decoupling cao và khả năng mở rộng tốt. Tuy nhiên, việc theo dõi luồng thực thi và gỡ lỗi phức tạp hơn, khó duy trì tính nhất quán dữ liệu.

Publish-Subscribe là dạng cụ thể của Event-Driven, cho phép phân phối thông tin từ một nguồn đến nhiều người nhận. Publisher gửi thông điệp đến kênh, nhiều subscribers nhận từ kênh đó. Triển khai qua Apache Kafka, RabbitMQ hoặc NATS. Phù hợp cho truyền thông tin một-đến-nhiều, dễ mở rộng, nhưng phức tạp trong quản lý tính nhất quán và có thể xử lý trùng lặp.

Point-to-Point Messaging gửi thông điệp từ nguồn đến đích cụ thể qua hàng đợi. Producer gửi thông điệp vào hàng đợi, chỉ một consumer xử lý mỗi thông điệp. Đảm bảo tin cậy cao, phù hợp cho phân phối tác vụ và cân bằng tải. Tuy nhiên, có thể nghẽn hàng đợi nếu xử lý chậm và không phù hợp khi nhiều dịch vụ cần nhận cùng thông tin.

Asynchronous Request-Response là biến thể bất đồng bộ của Request-Response. Dịch vụ gửi yêu cầu và tiếp tục xử lý, dịch vụ nhận xử lý và gửi phản hồi qua hàng đợi. Dịch vụ gửi được thông báo qua callback, webhook hoặc long polling. Tránh blocking, cải thiện hiệu suất, nhưng phức tạp hơn trong triển khai và quản lý.

Mỗi mẫu có ưu nhược điểm riêng, lựa chọn phù hợp phụ thuộc vào yêu cầu về tính nhất quán, hiệu suất, khả năng mở rộng và độ tin cậy. Thực tế, hệ thống vi dịch vụ thường kết hợp nhiều mẫu để giải quyết các tình huống khác nhau hiệu quả.
\section{Công nghệ và phương pháp đo lường hiệu năng}

\subsection{Các công nghệ triển khai trong dự án}
Trong triển khai kiến trúc microservices, việc lựa chọn công nghệ phù hợp đóng vai trò quan trọng, ảnh hưởng trực tiếp đến hiệu suất, khả năng mở rộng và bảo trì của hệ thống \cite{newman2015}. Khóa luận sử dụng NestJS làm framework chính cho việc phát triển microservices, một framework Node.js tiến bộ dựa trên TypeScript, cung cấp kiến trúc ứng dụng lấy cảm hứng từ Angular với các nguyên tắc SOLID và mô hình MVC. Framework này mang lại lợi ích như hỗ trợ dependency injection, kiến trúc mô-đun hóa cao và tích hợp sẵn với nhiều công nghệ khác.

Mỗi microservice được triển khai như một ứng dụng NestJS độc lập, với cấu trúc gồm controllers (xử lý yêu cầu HTTP), services (chứa logic nghiệp vụ), modules (đóng gói thành phần liên quan) và entities (đại diện đối tượng dữ liệu). NestJS cung cấp module microservices chuyên dụng hỗ trợ các giao thức như TCP, Redis, MQTT, gRPC, và Kafka, giúp đơn giản hóa việc triển khai các mẫu giao tiếp.

TypeScript được chọn làm ngôn ngữ lập trình chính với ưu điểm hệ thống kiểu dữ liệu tĩnh, giúp phát hiện lỗi sớm, tăng cường khả năng đọc hiểu và bảo trì mã nguồn. Trong môi trường microservices, TypeScript giúp đảm bảo tính nhất quán của dữ liệu được truyền giữa các dịch vụ thông qua các contract rõ ràng.

Về lưu trữ dữ liệu, nguyên tắc "mỗi dịch vụ có cơ sở dữ liệu riêng" được tuân thủ. TypeORM được sử dụng để tương tác với cơ sở dữ liệu, hỗ trợ nhiều hệ quản trị và cung cấp tính năng như quan hệ, kế thừa, migrations. TypeORM sử dụng cả Active Record và Data Mapper, hỗ trợ lazy/eager loading, transactions và query builder để tối ưu hóa hiệu suất truy vấn.

PostgreSQL được chọn làm hệ quản trị cơ sở dữ liệu chính do tính ổn định, hiệu suất cao và hỗ trợ dữ liệu phức tạp (JSON, JSONB, arrays). Khả năng xử lý đồng thời và transaction của PostgreSQL đảm bảo tính nhất quán dữ liệu trong môi trường phân tán.

Về giao tiếp giữa microservices, khóa luận sử dụng nhiều công nghệ cho các mẫu giao tiếp khác nhau. HTTP/REST API là nền tảng cho giao tiếp đồng bộ, với Axios làm HTTP client. RabbitMQ được triển khai cho mẫu Point-to-Point và Asynchronous Request-Response, cung cấp cơ chế tin cậy cao với xác nhận tin nhắn và hàng đợi bền vững. Apache Kafka được sử dụng cho Publish/Subscribe và Event-Driven, nổi bật với khả năng xử lý hàng triệu sự kiện mỗi giây, độ trễ thấp và lưu trữ sự kiện lâu dài.

\subsection{Các thông số đo lường chính}
Để đánh giá hiệu năng của các mẫu giao tiếp, khóa luận xem xét một tập hợp thông số toàn diện. Latency (Độ trễ) là thông số quan trọng nhất, đại diện cho thời gian cần thiết để hoàn thành một yêu cầu, từ khi gửi đến khi nhận phản hồi \cite{jun2018}. Độ trễ được phân tích theo nhiều khía cạnh: độ trễ đầu cuối (tổng thời gian từ client đến phản hồi), độ trễ dịch vụ (thời gian xử lý trong một microservice) và độ trễ mạng (thời gian di chuyển dữ liệu giữa dịch vụ).

Throughput (Thông lượng) đo lường số lượng yêu cầu hệ thống xử lý trong một đơn vị thời gian, biểu thị bằng yêu cầu/giây (RPS) hoặc giao dịch/giây (TPS) \cite{jun2018}. Thông lượng được đo ở nhiều cấp độ: hệ thống, dịch vụ và endpoint. Các mẫu giao tiếp khác nhau ảnh hưởng đáng kể đến thông lượng - mẫu đồng bộ thường có thông lượng thấp hơn, mẫu bất đồng bộ có thể đạt thông lượng cao hơn nhờ xử lý song song.

Error Rate (Tỷ lệ lỗi) là tỷ lệ phần trăm yêu cầu thất bại so với tổng số yêu cầu \cite{newman2015}. Tỷ lệ lỗi bị ảnh hưởng bởi lỗi mạng, lỗi dịch vụ, timeout hoặc lỗi logic nghiệp vụ \cite{richardson2019}. Mỗi loại lỗi (mạng, timeout, dịch vụ, logic) cần được phân loại và xử lý riêng biệt. Các mẫu giao tiếp khác nhau có cách tiếp cận khác nhau đối với xử lý lỗi, từ HTTP status codes đến dead-letter queues.

Resource Utilization (Sử dụng tài nguyên) đề cập đến lượng tài nguyên hệ thống (CPU, bộ nhớ, băng thông mạng) được sử dụng. Khóa luận giám sát sử dụng tài nguyên cho từng microservice và toàn hệ thống. Các mẫu giao tiếp đồng bộ thường có yêu cầu CPU/bộ nhớ thấp hơn nhưng nhiều kết nối mạng, mẫu bất đồng bộ có thể yêu cầu CPU/bộ nhớ cao hơn nhưng sử dụng mạng hiệu quả hơn.

Scalability (Khả năng mở rộng) đo lường khả năng xử lý tải tăng khi thêm tài nguyên. Khả năng mở rộng theo chiều ngang (thêm instance) thường được ưu tiên hơn chiều dọc (thêm tài nguyên cho instance hiện có). Các mẫu bất đồng bộ thường có khả năng mở rộng tốt hơn do tạo ít phụ thuộc trực tiếp giữa dịch vụ.

Consistency (Tính nhất quán) là khả năng duy trì trạng thái dữ liệu đồng bộ giữa các dịch vụ. Khóa luận đánh giá mức độ nhất quán dữ liệu đạt được bởi các mẫu giao tiếp khác nhau, từ tính nhất quán mạnh (strong consistency) đến nhất quán cuối cùng (eventual consistency).

\subsection{Phương pháp đo lường}
Khóa luận áp dụng nhiều phương pháp bổ sung nhau để thu thập dữ liệu hiệu năng \cite{newman2015}. Load Testing (Kiểm thử tải) mô phỏng điều kiện tải thực tế và đánh giá hiệu năng dưới áp lực \cite{jun2018}. Các kịch bản kiểm thử như kiểm tra tăng dần, chịu tải, phá vỡ và độ bền được thiết kế để mô phỏng trường hợp thực tế (tạo đơn hàng, kiểm tra tồn kho, xử lý thanh toán, gửi thông báo).

Benchmarking (Đánh giá) so sánh hiệu năng của các cấu hình hệ thống khác nhau trong điều kiện tiêu chuẩn \cite{richardson2019}. Benchmark được tiến hành cho mỗi mẫu giao tiếp với các trường hợp thử nghiệm giống nhau, từ 10 đến 100 người dùng đồng thời. Các metric thu thập bao gồm thời gian phản hồi, thông lượng, tỷ lệ lỗi và sử dụng tài nguyên.

Profiling (Lập hồ sơ) phân tích chi tiết tài nguyên và thời gian thực thi của các thành phần. Trong Node.js, profiling thực hiện bằng công cụ như Node.js Profiler hoặc clinic.js. Khóa luận sử dụng profiling để phân tích thời gian cho serialization/deserialization, xử lý mạng, logic nghiệp vụ và tương tác database.

Distributed Tracing (Theo dõi phân tán) theo dõi yêu cầu qua nhiều dịch vụ, xác định điểm nghẽn và mối quan hệ phụ thuộc. OpenTelemetry được tích hợp với NestJS thông qua interceptors và middleware. Mỗi trace đại diện cho một yêu cầu và gồm nhiều spans (hoạt động đơn lẻ như HTTP request, database query).

Metrics Collection (Thu thập số liệu) thu thập và phân tích chỉ số hiệu năng theo thời gian. Khóa luận thu thập HTTP metrics, microservice metrics, database metrics, message broker metrics và system metrics, lưu trữ trong time-series database để phân tích xu hướng và thiết lập cảnh báo.

\subsection{Công cụ đo lường hiệu năng}
Để thực hiện các phương pháp trên, khóa luận triển khai bộ công cụ toàn diện \cite{aksakalli2021}. K6, công cụ kiểm thử tải mã nguồn mở dựa trên JavaScript, tạo tải và đo lường hiệu năng \cite{jun2018}. K6 cho phép viết kịch bản phức tạp mô phỏng hành vi thực tế, hỗ trợ HTTP, WebSocket và gRPC, với khả năng mở rộng và tùy chỉnh cao.

Prometheus, hệ thống giám sát mã nguồn mở, thu thập và lưu trữ số liệu hiệu năng từ microservices \cite{richardson2019}. Prometheus sử dụng mô hình pull để truy vấn định kỳ các mục tiêu được cấu hình \cite{newman2015}, cung cấp ngôn ngữ truy vấn PromQL và hệ thống cảnh báo mạnh mẽ. Các microservices được cấu hình để hiển thị endpoint metrics (/metrics) mà Prometheus truy vấn mỗi 15 giây.

Kết hợp các công nghệ triển khai và công cụ đo lường này tạo môi trường toàn diện để đánh giá và so sánh các mẫu giao tiếp. Thông qua thu thập và phân tích dữ liệu từ nhiều góc độ, khóa luận cung cấp cái nhìn sâu sắc về ưu nhược điểm của mỗi mẫu và đưa ra khuyến nghị dựa trên bằng chứng cho việc lựa chọn mẫu phù hợp trong từng tình huống.
\section{Tổng kết}

Chương này đã cung cấp một cái nhìn tổng quan về kiến trúc microservice và vai trò quan trọng của giao tiếp trong kiến trúc này. Chúng ta đã thảo luận về các đặc điểm chính của microservices, so sánh với kiến trúc nguyên khối, và xem xét các lợi ích cũng như thách thức.

Về mặt đặc điểm, microservice là một kiến trúc phân tán, trong đó mỗi dịch vụ tự trị, tập trung vào một chức năng nghiệp vụ cụ thể, quản lý dữ liệu riêng của nó, được thiết kế để xử lý lỗi, và có thể phát triển độc lập. So với kiến trúc nguyên khối, microservice cung cấp khả năng mở rộng có mục tiêu, phát triển nhanh hơn, tính linh hoạt công nghệ, khả năng chịu lỗi tốt hơn, và khả năng bảo trì và hiểu biết tốt hơn.

Tuy nhiên, microservice cũng đặt ra một số thách thức, bao gồm độ phức tạp phân tán, giao tiếp giữa các dịch vụ, quản lý dữ liệu, vận hành và giám sát, và kiểm thử. Để giải quyết những thách thức này, một số nguyên tắc thiết kế nên được tuân thủ, bao gồm Single Responsibility Principle, Domain-Driven Design, API First, tự động hóa, Monitoring và Observability, và Fault Tolerance.

Trong phần về giao tiếp, chúng ta đã khám phá vai trò quan trọng của giao tiếp trong kiến trúc microservice, bao gồm tạo điều kiện cho sự hợp tác giữa các dịch vụ, đảm bảo tính nhất quán dữ liệu, hỗ trợ khả năng chịu lỗi, và cho phép tính mở rộng. Chúng ta cũng đã thảo luận về các thuộc tính quan trọng của giao tiếp microservice, bao gồm độ tin cậy, độ trễ, khả năng mở rộng, cách ly lỗi, tính nhất quán, định dạng dữ liệu, khả năng tương tác, và bảo mật.

Hai mô hình giao tiếp cơ bản trong microservices là đồng bộ và bất đồng bộ. Trong giao tiếp đồng bộ, người gửi đợi phản hồi từ người nhận, trong khi trong giao tiếp bất đồng bộ, người gửi không đợi phản hồi. Ngoài ra, các microservice cũng giao tiếp theo các kiểu tương tác khác nhau, bao gồm one-to-one, one-to-many, many-to-one, và many-to-many.

Có nhiều công nghệ và giao thức được sử dụng cho giao tiếp microservice, bao gồm HTTP/REST, gRPC, Message Queue, Pub/Sub, WebSockets và GraphQL. Mỗi công nghệ có ưu và nhược điểm riêng và phù hợp với các tình huống khác nhau.

Giao tiếp microservice đặt ra một số thách thức, bao gồm Network Reliability, Service Discovery, Load Balancing, Data Consistency, Versioning, Error Handling, và Monitoring and Debugging. Để giải quyết những thách thức này, một số mẫu giao tiếp đã được phát triển, bao gồm API Gateway, Circuit Breaker, Bulkhead, Retry, Timeout, Saga, Event Sourcing, và CQRS.

Trong phần về đo lường hiệu năng, chúng ta đã thảo luận về các thông số đo lường chính, bao gồm Latency, Throughput, Error Rate, Resource Utilization, và Scalability. Chúng ta cũng đã khám phá các phương pháp đo lường, bao gồm Load Testing, Benchmarking, Profiling, Distributed Tracing, và Metrics Collection. Cuối cùng, chúng ta đã giới thiệu một số công cụ phổ biến để đo lường hiệu năng của microservices, bao gồm K6, Prometheus, Grafana, Jaeger/Zipkin, và ELK Stack.

Các khái niệm và hiểu biết từ chương này sẽ làm nền tảng cho các chương tiếp theo, nơi chúng ta sẽ đi sâu vào việc phân tích chi tiết các mẫu giao tiếp cụ thể trong kiến trúc microservice. Chúng ta sẽ phân loại các mẫu này theo tiêu chí đồng bộ/bất đồng bộ và one-to-one/one-to-many, phân tích ưu và nhược điểm của từng mẫu, và cung cấp hướng dẫn cho việc lựa chọn mẫu phù hợp cho các tình huống cụ thể.

Việc hiểu rõ các khái niệm cơ bản và thách thức của giao tiếp microservice sẽ giúp chúng ta đánh giá tốt hơn hiệu quả của các mẫu giao tiếp trong các kịch bản thực tế. Đồng thời, các phương pháp và công cụ đo lường hiệu năng đã được giới thiệu sẽ được áp dụng trong các phần tiếp theo để đánh giá hiệu suất của các mẫu giao tiếp và đưa ra các khuyến nghị dựa trên dữ liệu.

Tóm lại, chương này đã cung cấp một cái nhìn toàn diện về kiến trúc microservice và vai trò quan trọng của giao tiếp trong kiến trúc này. Chúng ta đã hiểu được các đặc điểm, lợi ích và thách thức của microservice, cũng như các mô hình giao tiếp, công nghệ và mẫu thiết kế phổ biến. Những kiến thức này sẽ là nền tảng vững chắc cho các phân tích chi tiết hơn trong các chương tiếp theo.  