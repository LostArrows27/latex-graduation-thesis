\chapter{Mở đầu}

\section{Bối cảnh và sự cần thiết của đề tài}
Trong những năm gần đây, kiến trúc microservice đã trở thành một xu hướng quan trọng trong phát triển phần mềm hiện đại. Sự phát triển này đặt ra nhiều thách thức trong việc quản lý giao tiếp giữa các dịch vụ. Việc nghiên cứu và hiểu rõ các cơ chế giao tiếp trong kiến trúc microservice là cần thiết để:

\begin{itemize}
    \item Tối ưu hóa hiệu suất của hệ thống
    \item Đảm bảo độ tin cậy và khả năng mở rộng
    \item Giảm thiểu độ phức tạp trong phát triển và bảo trì
    \item Nâng cao khả năng chịu lỗi của hệ thống
\end{itemize}

\section{Mục tiêu nghiên cứu}
Khóa luận này nhằm đạt được các mục tiêu sau:

\begin{itemize}
    \item Phân tích và đánh giá các cơ chế giao tiếp trong kiến trúc microservice
    \item So sánh ưu nhược điểm của các phương pháp giao tiếp khác nhau
    \item Đề xuất các giải pháp tối ưu cho các tình huống cụ thể
    \item Thực nghiệm và đánh giá hiệu quả của các cơ chế giao tiếp
\end{itemize}

\section{Phạm vi nghiên cứu}
\begin{itemize}
    \item Tập trung vào các cơ chế giao tiếp phổ biến trong microservice
    \item Đánh giá trên các tiêu chí: hiệu suất, độ tin cậy, khả năng mở rộng
    \item Thực nghiệm trên các nền tảng và công nghệ phổ biến
\end{itemize}

\section{Phương pháp nghiên cứu}
\begin{itemize}
    \item \textbf{Phương pháp nghiên cứu lý thuyết:}
    \begin{itemize}
        \item Tổng hợp và phân tích tài liệu
        \item So sánh các phương pháp tiếp cận
        \item Đánh giá ưu nhược điểm
    \end{itemize}
    
    \item \textbf{Phương pháp nghiên cứu thực nghiệm:}
    \begin{itemize}
        \item Xây dựng môi trường thử nghiệm
        \item Triển khai các cơ chế giao tiếp
        \item Đo lường và đánh giá kết quả
    \end{itemize}
\end{itemize}

\section{Ý nghĩa khoa học và thực tiễn}
\begin{itemize}
    \item \textbf{Ý nghĩa khoa học:}
    \begin{itemize}
        \item Đóng góp vào việc nghiên cứu và phát triển các phương pháp giao tiếp trong kiến trúc microservice
        \item Cung cấp cơ sở lý thuyết cho việc lựa chọn và triển khai các cơ chế giao tiếp
        \item Đề xuất các hướng nghiên cứu mới trong lĩnh vực này
    \end{itemize}
    
    \item \textbf{Ý nghĩa thực tiễn:}
    \begin{itemize}
        \item Cung cấp hướng dẫn thực tế cho việc triển khai các hệ thống microservice
        \item Giúp các nhà phát triển đưa ra quyết định phù hợp về cơ chế giao tiếp
        \item Tối ưu hóa hiệu suất và độ tin cậy của hệ thống
    \end{itemize}
\end{itemize}

\section{Đối tượng và phạm vi nghiên cứu}
\subsection{Đối tượng nghiên cứu}
\begin{itemize}
    \item Các cơ chế giao tiếp trong kiến trúc microservice
    \item Các công nghệ và giao thức giao tiếp phổ biến
    \item Các mô hình triển khai và quản lý giao tiếp
\end{itemize}

\subsection{Phạm vi nghiên cứu}
\begin{itemize}
    \item Tập trung vào các cơ chế giao tiếp phổ biến trong microservice
    \item Đánh giá trên các tiêu chí: hiệu suất, độ tin cậy, khả năng mở rộng
    \item Thực nghiệm trên các nền tảng và công nghệ phổ biến
\end{itemize}

\section{Cấu trúc khóa luận}
Khóa luận được tổ chức thành 5 chương:

\begin{itemize}
    \item \textbf{Chương 1: Mở đầu} - Giới thiệu tổng quan về đề tài
    \item \textbf{Chương 2: Cơ sở lý thuyết} - Trình bày các khái niệm cơ bản
    \item \textbf{Chương 3: Phân tích các cơ chế giao tiếp} - Chi tiết về các phương pháp giao tiếp
    \item \textbf{Chương 4: Đánh giá và thực nghiệm} - Kết quả thực nghiệm và phân tích
    \item \textbf{Chương 5: Kết luận và hướng phát triển} - Tổng kết và đề xuất
\end{itemize} 