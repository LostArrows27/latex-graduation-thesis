\section{Cách phân loại các Communication pattern}
Trong kiến trúc microservices, giao tiếp giữa các dịch vụ đóng vai trò quan trọng trong việc đảm bảo hiệu suất, khả năng mở rộng và độ tin cậy của hệ thống \cite{richardson2019}. Việc thiết kế giao tiếp giữa các microservices là một trong những thách thức quan trọng nhất, ảnh hưởng trực tiếp đến kiến trúc tổng thể và hiệu quả hoạt động của hệ thống. Để hiểu rõ và áp dụng hiệu quả các mẫu giao tiếp, việc phân loại chúng một cách có hệ thống là cần thiết.

Các mẫu giao tiếp trong microservices có thể được phân loại dựa trên hai tiêu chí chính: phương thức giao tiếp (communication mode) và phạm vi giao tiếp (communication scope). Phương thức giao tiếp xác định cách thức tương tác giữa các dịch vụ, có thể là đồng bộ (synchronous) hoặc bất đồng bộ (asynchronous). Phạm vi giao tiếp xác định số lượng người nhận trong một lần giao tiếp, phổ biến nhất là one-to-one và one-to-many. Bảng 3.1 dưới đây tổng hợp các mẫu giao tiếp phổ biến dựa trên sự kết hợp của hai tiêu chí này.

\begin{table}[h]{So sánh các kiểu tương tác trong giao tiếp microservice}
    \centering
    {\setlength{\arrayrulewidth}{1pt}
    \renewcommand{\arraystretch}{1.5}
    \setlength{\tabcolsep}{12pt}
    \begin{tabular}{|l|c|c|}
        \hline
        \textbf{} & \textbf{one-to-one} & \textbf{one-to-many} \\
        \hline
        \textbf{Synchronous} & Request/Response & --- \\
        \hline
        \textbf{Asynchronous} & \begin{tabular}{@{}c@{}}Asynchronous Request/Response\\ One-way Notifications\end{tabular} & \begin{tabular}{@{}c@{}}Publish/Subscribe\\ Publish/Async responses\end{tabular} \\
        \hline
    \end{tabular}}
\end{table}

\subsection{Tiêu chí phân loại theo communication mode (sync/async)}
Tiêu chí đầu tiên và cơ bản nhất trong việc phân loại các mẫu giao tiếp là dựa trên mode giao tiếp: đồng bộ (synchronous) và bất đồng bộ (asynchronous). Sự phân biệt này liên quan đến cách các dịch vụ tương tác và đợi phản hồi từ nhau.

Trong giao tiếp đồng bộ, dịch vụ gửi yêu cầu chặn (block) quá trình xử lý của nó và đợi cho đến khi nhận được phản hồi từ dịch vụ nhận. Mô hình này tạo ra một sự phụ thuộc trực tiếp về thời gian giữa người gửi và người nhận, đồng thời đặt ra yêu cầu cả hai dịch vụ phải đồng thời hoạt động để hoàn thành giao tiếp \cite{newman2015}. Giao tiếp đồng bộ giống như một "cuộc gọi điện thoại" - người gọi phải đợi người nhận trả lời và hoàn thành cuộc trò chuyện trước khi có thể tiếp tục các hoạt động khác.

Các giao thức đồng bộ phổ biến trong microservices bao gồm HTTP/REST, gRPC và SOAP. Giao tiếp đồng bộ thường được triển khai thông qua các API endpoint, với dịch vụ gọi gửi yêu cầu HTTP và đợi phản hồi. Mặc dù giao tiếp đồng bộ có ưu điểm về tính đơn giản và dễ hiểu, nó cũng tạo ra coupling chặt chẽ giữa các dịch vụ và có thể dẫn đến hiệu suất kém trong môi trường phân tán.

Ngược lại, trong giao tiếp bất đồng bộ, dịch vụ gửi không chặn quá trình xử lý của nó khi đợi phản hồi. Thay vào đó, nó tiếp tục thực hiện các tác vụ khác và xử lý phản hồi (nếu cần) khi phản hồi đó đến. Giao tiếp bất đồng bộ giống như "gửi email" - người gửi không cần đợi người nhận đọc và trả lời ngay lập tức.

Các công nghệ bất đồng bộ phổ biến bao gồm message brokers như RabbitMQ, Apache Kafka và Amazon SQS. Các công nghệ này cho phép các dịch vụ giao tiếp thông qua việc gửi và nhận tin nhắn mà không cần đồng thời hoạt động. Giao tiếp bất đồng bộ thúc đẩy sự tách rời (decoupling) giữa các dịch vụ, cải thiện khả năng chịu lỗi và mở rộng, nhưng cũng làm tăng độ phức tạp của hệ thống và khó khăn trong việc theo dõi luồng yêu cầu.

Trong một đánh giá hệ thống về các mẫu giao tiếp và triển khai trong kiến trúc microservices, các mẫu đồng bộ như Request-Response thường có thông lượng thấp hơn do tính chất tuần tự của chúng, trong khi các mẫu bất đồng bộ như Publish/Subscribe và Event-Driven có thể đạt thông lượng cao hơn do khả năng xử lý song song \cite{aksakalli2021}.

\subsection{Tiêu chí phân loại theo communication scope (one-to-one/one-to-many)}
Tiêu chí phân loại thứ hai là dựa trên phạm vi giao tiếp, tức là số lượng người nhận tham gia vào quá trình nhận thông điệp. Trong phạm vi khóa luận này, chúng ta tập trung vào hai loại chính là one-to-one và one-to-many, đây là hai mô hình giao tiếp phổ biến nhất trong kiến trúc microservices.

Giao tiếp one-to-one là mô hình trong đó một dịch vụ gửi thông điệp đến chính xác một dịch vụ khác và chỉ dịch vụ đó nhận được thông điệp. Mô hình này phù hợp cho các tương tác yêu cầu-phản hồi trực tiếp giữa hai dịch vụ, chẳng hạn như khi một dịch vụ cần truy vấn dữ liệu từ dịch vụ khác \cite{richardson2019}. Trong mô hình này, người gửi biết rõ người nhận và thường mong đợi một phản hồi. Các mẫu giao tiếp one-to-one phổ biến bao gồm Request-Response (đồng bộ), Asynchronous Request-Response và One-way Notifications (bất đồng bộ).

Giao tiếp one-to-many là mô hình trong đó một dịch vụ gửi thông điệp đến nhiều dịch vụ khác cùng một lúc. Mô hình này thích hợp cho việc phát tán thông tin hoặc thông báo sự kiện trong hệ thống. Ví dụ, khi một dịch vụ muốn thông báo về sự thay đổi trạng thái mà có thể ảnh hưởng đến nhiều dịch vụ khác. Trong mô hình này, người gửi thường không biết hoặc không quan tâm đến việc ai sẽ nhận thông điệp của mình. Các mẫu giao tiếp one-to-many điển hình bao gồm Publish/Subscribe và Publish/Async responses, tất cả đều là các mẫu bất đồng bộ.

Sự kết hợp giữa hai tiêu chí phân loại - communication mode (sync/async) và communication scope (one-to-one/one-to-many) - tạo ra một ma trận phân loại các mẫu giao tiếp. Trong ma trận này, có một điểm đáng chú ý là trong thực tế, hầu như không có mẫu giao tiếp đồng bộ one-to-many được sử dụng rộng rãi. Điều này hoàn toàn hợp lý vì việc một dịch vụ đồng thời gửi yêu cầu đến nhiều dịch vụ khác và đợi tất cả phản hồi sẽ tạo ra bottleneck về hiệu suất và tăng khả năng lỗi cascade. Do đó, khi cần giao tiếp one-to-many, các kiến trúc microservices hiện đại gần như luôn ưu tiên sử dụng các mẫu bất đồng bộ.

Phạm vi giao tiếp có ảnh hưởng lớn đến sự phức tạp, khả năng mở rộng và khả năng bảo trì của hệ thống. Các mẫu giao tiếp one-to-many thường phức tạp hơn để triển khai và quản lý so với one-to-one, nhưng cung cấp khả năng tách rời (decoupling) tốt hơn và khả năng mở rộng cao hơn. Ví dụ, với mẫu Publish/Subscribe, việc thêm người đăng ký (subscriber) mới không đòi hỏi bất kỳ thay đổi nào từ phía nhà xuất bản (publisher), điều này tạo ra sự linh hoạt cao trong việc mở rộng hệ thống.

\subsection{Các yếu tố ảnh hưởng đến việc lựa chọn pattern}
Việc lựa chọn mẫu giao tiếp phù hợp cho một hệ thống microservices phụ thuộc vào nhiều yếu tố. Hiểu rõ các yếu tố này sẽ giúp kiến trúc sư và nhà phát triển đưa ra quyết định đúng đắn, cân bằng giữa hiệu suất, độ tin cậy, độ phức tạp và khả năng mở rộng.

Yêu cầu về độ trễ (latency) là một trong những yếu tố quan trọng nhất ảnh hưởng đến việc lựa chọn mẫu giao tiếp. Các ứng dụng đòi hỏi thời gian phản hồi nhanh, như các ứng dụng giao diện người dùng, thường ưu tiên giao tiếp đồng bộ \cite{richardson2019}. Trong khi đó, các quy trình xử lý nền không cần phản hồi tức thì có thể hưởng lợi từ giao tiếp bất đồng bộ. Giao tiếp đồng bộ đơn giản hơn để hiểu và triển khai, nhưng có thể dẫn đến độ trễ cao khi có nhiều cuộc gọi nối tiếp. Giao tiếp bất đồng bộ giảm thiểu độ trễ cảm nhận được bằng cách cho phép máy khách tiếp tục hoạt động trong khi yêu cầu được xử lý, nhưng làm tăng độ phức tạp của hệ thống \cite{newman2015}.

Yêu cầu về khả năng mở rộng (scalability) cũng là một yếu tố quyết định. Các hệ thống cần xử lý khối lượng lớn yêu cầu thường ưu tiên các mẫu giao tiếp bất đồng bộ, đặc biệt là các mẫu như Publish/Subscribe và Event-Driven. Các mẫu bất đồng bộ như Publish/Subscribe và Event-Driven thường có khả năng mở rộng tốt hơn do chúng tách rời các producer và consumer, cho phép chúng mở rộng độc lập \cite{aksakalli2021}.

Độ tin cậy và khả năng chịu lỗi là các yếu tố không thể bỏ qua. Các mẫu giao tiếp bất đồng bộ thường cung cấp khả năng chịu lỗi tốt hơn so với các mẫu đồng bộ, vì chúng có thể xử lý các tình huống như dịch vụ không khả dụng tạm thời \cite{fowler2014}. Các mẫu như Point-to-Point Messaging với đảm bảo giao hàng và Asynchronous Request-Response với cơ chế retry tích hợp có thể cải thiện đáng kể độ tin cậy của hệ thống.

Độ phức tạp triển khai và bảo trì cũng là một yếu tố quan trọng. Giao tiếp đồng bộ thường đơn giản hơn để triển khai và gỡ lỗi, trong khi giao tiếp bất đồng bộ đòi hỏi cơ sở hạ tầng phức tạp hơn (như message brokers) và logic xử lý phức tạp hơn để đảm bảo tin cậy \cite{newman2015}.

Tính nhất quán dữ liệu là một yếu tố đặc biệt quan trọng trong các hệ thống phân tán. Việc duy trì tính nhất quán dữ liệu giữa các microservices có thể là một thách thức, đặc biệt với giao tiếp bất đồng bộ \cite{richardson2019}. Các mẫu như Saga và Event Sourcing đã được phát triển để giải quyết vấn đề này bằng cách cung cấp các cơ chế để quản lý giao dịch phân tán và đảm bảo tính nhất quán cuối cùng.

Yêu cầu nghiệp vụ cụ thể cũng đóng vai trò quan trọng trong việc lựa chọn mẫu giao tiếp. Ví dụ, các quy trình nghiệp vụ đòi hỏi phản hồi ngay lập tức, như xử lý thanh toán trực tuyến, có thể yêu cầu giao tiếp đồng bộ. Trong khi đó, các quy trình như phân tích dữ liệu hoặc gửi email thông báo có thể phù hợp hơn với giao tiếp bất đồng bộ.

Cuối cùng, tính linh hoạt của thiết kế và khả năng phát triển trong tương lai cũng cần được cân nhắc. Tầm quan trọng của việc thiết kế hệ thống có thể thích nghi với các yêu cầu thay đổi không thể phủ nhận \cite{newman2015}. Các mẫu giao tiếp bất đồng bộ, đặc biệt là Event-Driven và Publish/Subscribe, thường cung cấp tính linh hoạt cao hơn bằng cách cho phép thêm dịch vụ mới mà không cần thay đổi các dịch vụ hiện có.

Việc hiểu và cân nhắc các yếu tố này là rất quan trọng để lựa chọn mẫu giao tiếp phù hợp cho một hệ thống microservices cụ thể. Thông thường, một hệ thống microservices hiệu quả sẽ kết hợp nhiều mẫu giao tiếp khác nhau để đáp ứng các yêu cầu khác nhau của các phần khác nhau trong hệ thống.