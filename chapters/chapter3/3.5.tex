\section{Tổng kết}
Trong chương này, chúng ta đã tìm hiểu chi tiết về các mẫu giao tiếp trong kiến trúc microservices. Đầu tiên, chúng ta đã phân loại các mẫu giao tiếp theo hai tiêu chí chính: communication mode (synchronous/asynchronous) và communication scope (one-to-one/one-to-many).

Với giao tiếp đồng bộ (synchronous) one-to-one, chúng ta tập trung vào mẫu Request/Response, được triển khai thông qua các công nghệ phổ biến như REST, gRPC và GraphQL. Mẫu này có ưu điểm là đơn giản, trực quan và phản hồi tức thì, nhưng hạn chế ở khả năng mở rộng và độ tin cậy.

Trong giao tiếp bất đồng bộ (asynchronous) one-to-one, chúng ta đã xem xét các mẫu như One-way Notifications và Message Queue. Các mẫu này cung cấp decoupling tốt hơn, khả năng chịu lỗi cao và khả năng mở rộng tốt, nhưng lại phức tạp hơn trong triển khai và debug.

Đối với giao tiếp bất đồng bộ one-to-many, chúng ta đã tìm hiểu các mẫu như Publish/Subscribe, Event Sourcing và Message Broker với Exchange Routing. Các mẫu này cung cấp decoupling cao nhất và khả năng mở rộng tốt nhất, đặc biệt phù hợp cho việc phát tán thông tin và xử lý sự kiện.

Mỗi mẫu giao tiếp đều có những use cases phù hợp riêng và việc lựa chọn mẫu giao tiếp phù hợp phụ thuộc vào nhiều yếu tố như yêu cầu về độ trễ, tính nhất quán dữ liệu, khả năng mở rộng và độ phức tạp triển khai.

Trong thực tế, một hệ thống microservices hiệu quả thường kết hợp nhiều mẫu giao tiếp khác nhau, sử dụng mỗi mẫu cho những trường hợp phù hợp nhất. Hiểu rõ các mẫu giao tiếp và tiêu chí lựa chọn là nền tảng quan trọng để thiết kế một kiến trúc microservices thành công.

Ở chương tiếp theo, chúng ta sẽ tiến hành triển khai thử nghiệm các mẫu giao tiếp đã học trong một hệ thống microservices thực tế. Chúng ta sẽ xây dựng một ứng dụng mẫu với nhiều microservices khác nhau, triển khai các mẫu giao tiếp khác nhau, và đánh giá hiệu suất cũng như tính phù hợp của từng mẫu trong các tình huống cụ thể.