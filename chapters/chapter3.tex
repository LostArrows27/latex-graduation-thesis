\chapter{Phân tích các mẫu giao tiếp}

Chương 3, Phân tích các mẫu giao tiếp, trình bày cách thức phân loại và đánh giá chi tiết các mẫu giao tiếp trong kiến trúc vi dịch vụ. Chương này bắt đầu với việc phân loại các mẫu giao tiếp theo hai tiêu chí chính: phương thức giao tiếp (đồng bộ/bất đồng bộ) và phạm vi giao tiếp (một-một/một-nhiều), cũng như phân tích các yếu tố ảnh hưởng đến việc lựa chọn mẫu giao tiếp phù hợp. Tiếp theo, chương đi sâu vào phân tích chi tiết ba nhóm mẫu giao tiếp chính: mẫu giao tiếp đồng bộ (một-một), mẫu giao tiếp bất đồng bộ (một-một), và mẫu giao tiếp bất đồng bộ (một-nhiều). Với mỗi nhóm mẫu, chương trình bày cơ chế hoạt động, các pattern cụ thể, ưu điểm và hạn chế, các trường hợp sử dụng phù hợp, và các ví dụ thực tế.

\section{Cách phân loại các mẫu giao tiếp}
Trong kiến trúc microservices, giao tiếp giữa các dịch vụ đóng vai trò quan trọng trong việc đảm bảo hiệu suất, khả năng mở rộng và độ tin cậy của hệ thống \cite{richardson2019}. Việc thiết kế giao tiếp giữa các microservices là một trong những thách thức quan trọng nhất, ảnh hưởng trực tiếp đến kiến trúc tổng thể và hiệu quả hoạt động của hệ thống. Để hiểu rõ và áp dụng hiệu quả các mẫu giao tiếp, việc phân loại chúng một cách có hệ thống là cần thiết.

Các mẫu giao tiếp trong microservices có thể được phân loại dựa trên hai tiêu chí chính: phương thức giao tiếp (communication mode) và phạm vi giao tiếp (communication scope). Phương thức giao tiếp xác định cách thức tương tác giữa các dịch vụ, có thể là đồng bộ (synchronous) hoặc bất đồng bộ (asynchronous). Phạm vi giao tiếp xác định số lượng người nhận trong một lần giao tiếp, phổ biến nhất là one-to-one và one-to-many. Bảng 3.1 dưới đây tổng hợp các mẫu giao tiếp phổ biến dựa trên sự kết hợp của hai tiêu chí này.

\begin{table}[h]{So sánh các kiểu tương tác trong giao tiếp microservice}
    \centering
    {\setlength{\arrayrulewidth}{1pt}
    \renewcommand{\arraystretch}{1.5}
    \setlength{\tabcolsep}{12pt}
    \begin{tabular}{|l|c|c|}
        \hline
        \textbf{} & \textbf{one-to-one} & \textbf{one-to-many} \\
        \hline
        \textbf{Synchronous} & Request/Response & --- \\
        \hline
        \textbf{Asynchronous} & \begin{tabular}{@{}c@{}}Asynchronous Request/Response\\ One-way Notifications\end{tabular} & \begin{tabular}{@{}c@{}}Publish/Subscribe\\ Publish/Async responses\end{tabular} \\
        \hline
    \end{tabular}}
\end{table}

\subsection{Tiêu chí phân loại theo communication mode (sync/async)}
Tiêu chí đầu tiên và cơ bản nhất trong việc phân loại các mẫu giao tiếp là dựa trên mode giao tiếp: đồng bộ (synchronous) và bất đồng bộ (asynchronous). Sự phân biệt này liên quan đến cách các dịch vụ tương tác và đợi phản hồi từ nhau.

Trong giao tiếp đồng bộ, dịch vụ gửi yêu cầu chặn (block) quá trình xử lý của nó và đợi cho đến khi nhận được phản hồi từ dịch vụ nhận. Mô hình này tạo ra một sự phụ thuộc trực tiếp về thời gian giữa người gửi và người nhận, đồng thời đặt ra yêu cầu cả hai dịch vụ phải đồng thời hoạt động để hoàn thành giao tiếp \cite{newman2015}. Giao tiếp đồng bộ giống như một "cuộc gọi điện thoại" - người gọi phải đợi người nhận trả lời và hoàn thành cuộc trò chuyện trước khi có thể tiếp tục các hoạt động khác.

Các giao thức đồng bộ phổ biến trong microservices bao gồm HTTP/REST, gRPC và SOAP. Giao tiếp đồng bộ thường được triển khai thông qua các API endpoint, với dịch vụ gọi gửi yêu cầu HTTP và đợi phản hồi. Mặc dù giao tiếp đồng bộ có ưu điểm về tính đơn giản và dễ hiểu, nó cũng tạo ra coupling chặt chẽ giữa các dịch vụ và có thể dẫn đến hiệu suất kém trong môi trường phân tán.

Ngược lại, trong giao tiếp bất đồng bộ, dịch vụ gửi không chặn quá trình xử lý của nó khi đợi phản hồi. Thay vào đó, nó tiếp tục thực hiện các tác vụ khác và xử lý phản hồi (nếu cần) khi phản hồi đó đến. Giao tiếp bất đồng bộ giống như "gửi email" - người gửi không cần đợi người nhận đọc và trả lời ngay lập tức.

Các công nghệ bất đồng bộ phổ biến bao gồm message brokers như RabbitMQ, Apache Kafka và Amazon SQS. Các công nghệ này cho phép các dịch vụ giao tiếp thông qua việc gửi và nhận tin nhắn mà không cần đồng thời hoạt động. Giao tiếp bất đồng bộ thúc đẩy sự tách rời (decoupling) giữa các dịch vụ, cải thiện khả năng chịu lỗi và mở rộng, nhưng cũng làm tăng độ phức tạp của hệ thống và khó khăn trong việc theo dõi luồng yêu cầu.

Trong một đánh giá hệ thống về các mẫu giao tiếp và triển khai trong kiến trúc microservices, các mẫu đồng bộ như Request-Response thường có thông lượng thấp hơn do tính chất tuần tự của chúng, trong khi các mẫu bất đồng bộ như Publish/Subscribe và Event-Driven có thể đạt thông lượng cao hơn do khả năng xử lý song song \cite{aksakalli2021}.

\subsection{Tiêu chí phân loại theo communication scope (one-to-one/one-to-many)}
Tiêu chí phân loại thứ hai là dựa trên phạm vi giao tiếp, tức là số lượng người nhận tham gia vào quá trình nhận thông điệp. Trong phạm vi khóa luận này, chúng ta tập trung vào hai loại chính là one-to-one và one-to-many, đây là hai mô hình giao tiếp phổ biến nhất trong kiến trúc microservices.

Giao tiếp one-to-one là mô hình trong đó một dịch vụ gửi thông điệp đến chính xác một dịch vụ khác và chỉ dịch vụ đó nhận được thông điệp. Mô hình này phù hợp cho các tương tác yêu cầu-phản hồi trực tiếp giữa hai dịch vụ, chẳng hạn như khi một dịch vụ cần truy vấn dữ liệu từ dịch vụ khác \cite{richardson2019}. Trong mô hình này, người gửi biết rõ người nhận và thường mong đợi một phản hồi. Các mẫu giao tiếp one-to-one phổ biến bao gồm Request-Response (đồng bộ), Asynchronous Request-Response và One-way Notifications (bất đồng bộ).

Giao tiếp one-to-many là mô hình trong đó một dịch vụ gửi thông điệp đến nhiều dịch vụ khác cùng một lúc. Mô hình này thích hợp cho việc phát tán thông tin hoặc thông báo sự kiện trong hệ thống. Ví dụ, khi một dịch vụ muốn thông báo về sự thay đổi trạng thái mà có thể ảnh hưởng đến nhiều dịch vụ khác. Trong mô hình này, người gửi thường không biết hoặc không quan tâm đến việc ai sẽ nhận thông điệp của mình. Các mẫu giao tiếp one-to-many điển hình bao gồm Publish/Subscribe và Publish/Async responses, tất cả đều là các mẫu bất đồng bộ.

Sự kết hợp giữa hai tiêu chí phân loại - communication mode (sync/async) và communication scope (one-to-one/one-to-many) - tạo ra một ma trận phân loại các mẫu giao tiếp. Trong ma trận này, có một điểm đáng chú ý là trong thực tế, hầu như không có mẫu giao tiếp đồng bộ one-to-many được sử dụng rộng rãi. Điều này hoàn toàn hợp lý vì việc một dịch vụ đồng thời gửi yêu cầu đến nhiều dịch vụ khác và đợi tất cả phản hồi sẽ tạo ra bottleneck về hiệu suất và tăng khả năng lỗi cascade. Do đó, khi cần giao tiếp one-to-many, các kiến trúc microservices hiện đại gần như luôn ưu tiên sử dụng các mẫu bất đồng bộ.

Phạm vi giao tiếp có ảnh hưởng lớn đến sự phức tạp, khả năng mở rộng và khả năng bảo trì của hệ thống. Các mẫu giao tiếp one-to-many thường phức tạp hơn để triển khai và quản lý so với one-to-one, nhưng cung cấp khả năng tách rời (decoupling) tốt hơn và khả năng mở rộng cao hơn. Ví dụ, với mẫu Publish/Subscribe, việc thêm người đăng ký (subscriber) mới không đòi hỏi bất kỳ thay đổi nào từ phía nhà xuất bản (publisher), điều này tạo ra sự linh hoạt cao trong việc mở rộng hệ thống.

\subsection{Các yếu tố ảnh hưởng đến việc lựa chọn pattern}
Việc lựa chọn mẫu giao tiếp phù hợp cho một hệ thống microservices phụ thuộc vào nhiều yếu tố. Hiểu rõ các yếu tố này sẽ giúp kiến trúc sư và nhà phát triển đưa ra quyết định đúng đắn, cân bằng giữa hiệu suất, độ tin cậy, độ phức tạp và khả năng mở rộng.

Yêu cầu về độ trễ (latency) là một trong những yếu tố quan trọng nhất ảnh hưởng đến việc lựa chọn mẫu giao tiếp. Các ứng dụng đòi hỏi thời gian phản hồi nhanh, như các ứng dụng giao diện người dùng, thường ưu tiên giao tiếp đồng bộ \cite{richardson2019}. Trong khi đó, các quy trình xử lý nền không cần phản hồi tức thì có thể hưởng lợi từ giao tiếp bất đồng bộ. Giao tiếp đồng bộ đơn giản hơn để hiểu và triển khai, nhưng có thể dẫn đến độ trễ cao khi có nhiều cuộc gọi nối tiếp. Giao tiếp bất đồng bộ giảm thiểu độ trễ cảm nhận được bằng cách cho phép máy khách tiếp tục hoạt động trong khi yêu cầu được xử lý, nhưng làm tăng độ phức tạp của hệ thống \cite{newman2015}.

Yêu cầu về khả năng mở rộng (scalability) cũng là một yếu tố quyết định. Các hệ thống cần xử lý khối lượng lớn yêu cầu thường ưu tiên các mẫu giao tiếp bất đồng bộ, đặc biệt là các mẫu như Publish/Subscribe và Event-Driven. Các mẫu bất đồng bộ như Publish/Subscribe và Event-Driven thường có khả năng mở rộng tốt hơn do chúng tách rời các producer và consumer, cho phép chúng mở rộng độc lập \cite{aksakalli2021}.

Độ tin cậy và khả năng chịu lỗi là các yếu tố không thể bỏ qua. Các mẫu giao tiếp bất đồng bộ thường cung cấp khả năng chịu lỗi tốt hơn so với các mẫu đồng bộ, vì chúng có thể xử lý các tình huống như dịch vụ không khả dụng tạm thời \cite{fowler2014}. Các mẫu như Point-to-Point Messaging với đảm bảo giao hàng và Asynchronous Request-Response với cơ chế retry tích hợp có thể cải thiện đáng kể độ tin cậy của hệ thống.

Độ phức tạp triển khai và bảo trì cũng là một yếu tố quan trọng. Giao tiếp đồng bộ thường đơn giản hơn để triển khai và gỡ lỗi, trong khi giao tiếp bất đồng bộ đòi hỏi cơ sở hạ tầng phức tạp hơn (như message brokers) và logic xử lý phức tạp hơn để đảm bảo tin cậy \cite{newman2015}.

Tính nhất quán dữ liệu là một yếu tố đặc biệt quan trọng trong các hệ thống phân tán. Việc duy trì tính nhất quán dữ liệu giữa các microservices có thể là một thách thức, đặc biệt với giao tiếp bất đồng bộ \cite{richardson2019}. Các mẫu như Saga và Event Sourcing đã được phát triển để giải quyết vấn đề này bằng cách cung cấp các cơ chế để quản lý giao dịch phân tán và đảm bảo tính nhất quán cuối cùng.

Yêu cầu nghiệp vụ cụ thể cũng đóng vai trò quan trọng trong việc lựa chọn mẫu giao tiếp. Ví dụ, các quy trình nghiệp vụ đòi hỏi phản hồi ngay lập tức, như xử lý thanh toán trực tuyến, có thể yêu cầu giao tiếp đồng bộ. Trong khi đó, các quy trình như phân tích dữ liệu hoặc gửi email thông báo có thể phù hợp hơn với giao tiếp bất đồng bộ.

Cuối cùng, tính linh hoạt của thiết kế và khả năng phát triển trong tương lai cũng cần được cân nhắc. Tầm quan trọng của việc thiết kế hệ thống có thể thích nghi với các yêu cầu thay đổi không thể phủ nhận \cite{newman2015}. Các mẫu giao tiếp bất đồng bộ, đặc biệt là Event-Driven và Publish/Subscribe, thường cung cấp tính linh hoạt cao hơn bằng cách cho phép thêm dịch vụ mới mà không cần thay đổi các dịch vụ hiện có.

Việc hiểu và cân nhắc các yếu tố này là rất quan trọng để lựa chọn mẫu giao tiếp phù hợp cho một hệ thống microservices cụ thể. Thông thường, một hệ thống microservices hiệu quả sẽ kết hợp nhiều mẫu giao tiếp khác nhau để đáp ứng các yêu cầu khác nhau của các phần khác nhau trong hệ thống.
\section{Synchronous Communication Patterns (one-to-one)}
Giao tiếp đồng bộ là hình thức tương tác phổ biến nhất giữa các microservice, đặc biệt trong các mô hình one-to-one. Đây là mô hình được ứng dụng rộng rãi nhờ tính đơn giản và dễ triển khai của nó. Trong phần này, chúng ta sẽ tìm hiểu sâu về Request/Response - mẫu giao tiếp đồng bộ cơ bản nhất trong kiến trúc microservices, cùng các công nghệ triển khai phổ biến và những ứng dụng thực tế của nó.

\subsection{Request/Response Pattern}
Request/Response là mẫu giao tiếp đồng bộ cơ bản nhất, trong đó một microservice (client) gửi yêu cầu đến một microservice khác (server) và chờ đợi cho đến khi nhận được phản hồi. Quá trình giao tiếp này tạo ra một sự ràng buộc tạm thời giữa hai dịch vụ, buộc client phải chờ đợi server xử lý và phản hồi trước khi tiếp tục các hoạt động khác.

Mẫu Request/Response tuân theo một quy trình cơ bản: client microservice xác định yêu cầu cần thực hiện và endpoint của server microservice; gửi yêu cầu đến server microservice, thường thông qua một giao thức như HTTP; server microservice nhận yêu cầu, xử lý nó và chuẩn bị phản hồi; server gửi phản hồi trở lại client microservice; và cuối cùng client nhận phản hồi và tiếp tục quá trình xử lý của mình. Trong suốt quá trình này, luồng thực thi của client microservice bị chặn cho đến khi nhận được phản hồi từ server microservice, đây chính là đặc điểm định nghĩa tính đồng bộ của mẫu giao tiếp này [1].

Các công nghệ triển khai Request/Response phổ biến nhất bao gồm REST, gRPC, GraphQL, và SOAP. Trong đó, REST (Representational State Transfer) là phương pháp phổ biến nhất do tính đơn giản và khả năng tương thích rộng rãi của nó.

REST (Representational State Transfer)

REST là một kiến trúc giao tiếp dựa trên HTTP, sử dụng các phương thức HTTP chuẩn (GET, POST, PUT, DELETE) để thực hiện các thao tác CRUD (Create, Read, Update, Delete) trên tài nguyên. REST đã trở thành tiêu chuẩn de facto cho việc xây dựng API web và giao tiếp giữa các dịch vụ trên Internet [2].

Trong kiến trúc microservices, REST API thường được sử dụng để triển khai mẫu Request/Response bởi những ưu điểm về tính đơn giản và dễ hiểu, tính stateless (mỗi yêu cầu chứa đầy đủ thông tin cần thiết, không phụ thuộc vào yêu cầu trước đó), khả năng mở rộng tốt (nhờ tính chất stateless, REST API có thể mở rộng dễ dàng bằng cách thêm nhiều instance server), và tính tương thích cao (REST có thể được triển khai trên nhiều nền tảng và ngôn ngữ lập trình khác nhau).

Một tương tác REST điển hình trong kiến trúc microservices có thể bao gồm một yêu cầu GET đến một endpoint như /api/products/123 để lấy thông tin về một sản phẩm cụ thể. Server sẽ phản hồi với mã trạng thái HTTP (như 200 OK cho thành công) và dữ liệu sản phẩm, thường ở định dạng JSON.

Mặc dù REST là phương pháp triển khai phổ biến nhất cho Request/Response, nó cũng có một số hạn chế như vấn đề overloading nghiêm trọng (REST thường dẫn đến overfetching hoặc underfetching), thiếu kiểm soát contract (REST không có cơ chế tích hợp để định nghĩa và kiểm tra contract giữa client và server), và hiệu suất không tối ưu (HTTP là một giao thức text-based, không hiệu quả bằng các giao thức binary).

gRPC (Google Remote Procedure Call)

gRPC là một framework RPC hiệu suất cao được phát triển bởi Google, sử dụng Protocol Buffers làm ngôn ngữ định nghĩa interface và HTTP/2 làm giao thức truyền tải. gRPC cung cấp một cách hiệu quả hơn để triển khai mẫu Request/Response trong kiến trúc microservices [3].

So với REST, gRPC có nhiều ưu điểm đáng kể về hiệu suất (gRPC sử dụng Protocol Buffers, một định dạng nhị phân hiệu quả hơn JSON hoặc XML), service contract rõ ràng (gRPC sử dụng .proto files để định nghĩa service interface, giúp tạo ra client và server code tự động), hỗ trợ streaming (gRPC hỗ trợ cả streaming đơn hướng và hai chiều), và hỗ trợ đa ngôn ngữ (gRPC tạo code cho nhiều ngôn ngữ lập trình từ cùng một định nghĩa).

Trong gRPC, dịch vụ được định nghĩa trong file .proto với các messages (cấu trúc dữ liệu) và services (các phương thức RPC). Từ file này, gRPC sẽ tự động tạo ra code client và server cho nhiều ngôn ngữ lập trình khác nhau. Quá trình giao tiếp diễn ra thông qua HTTP/2, cung cấp nhiều tính năng như multiplexing, header compression, và full-duplex communication.

Mặc dù gRPC mang lại nhiều lợi ích, nó cũng có một số hạn chế như độ phức tạp cao hơn (gRPC đòi hỏi hiểu biết về Protocol Buffers và HTTP/2), thiếu hỗ trợ trình duyệt (hầu hết các trình duyệt web không hỗ trợ trực tiếp gRPC), và khó khăn trong debug (do sử dụng định dạng nhị phân, việc debug gRPC phức tạp hơn so với REST).

GraphQL

GraphQL là một ngôn ngữ truy vấn và thao tác dữ liệu được phát triển bởi Facebook. Trong kiến trúc microservices, GraphQL thường được sử dụng như một lớp API gateway để tổng hợp dữ liệu từ nhiều microservices và cho phép client chỉ định chính xác dữ liệu họ cần [4].

GraphQL có một số ưu điểm đáng chú ý so với REST và gRPC như tránh được overfetching và underfetching (client có thể chỉ định chính xác dữ liệu cần thiết), sử dụng một endpoint duy nhất cho tất cả các truy vấn, khả năng introspection (GraphQL API tự mô tả, cho phép tạo tài liệu tự động), và hỗ trợ tốt cho frontend (các thư viện client như Apollo và Relay cung cấp tích hợp mạnh mẽ với các framework frontend).

Trong GraphQL, client gửi một truy vấn (query) xác định chính xác dữ liệu mà họ cần. Ví dụ, thay vì gọi nhiều endpoint REST khác nhau để lấy thông tin về sản phẩm, đánh giá và người dùng, client có thể gửi một truy vấn GraphQL duy nhất chỉ định tất cả thông tin cần thiết. Server GraphQL chịu trách nhiệm thu thập dữ liệu từ các nguồn khác nhau (có thể là các microservices khác) và trả về chính xác những gì client yêu cầu.

Mặc dù GraphQL mang lại nhiều lợi ích, nó cũng có những thách thức như độ phức tạp cao (triển khai GraphQL API đòi hỏi kiến thức chuyên sâu và nhiều boilerplate code), caching phức tạp hơn (do sử dụng một endpoint duy nhất, caching với GraphQL phức tạp hơn so với REST), và vấn đề N+1 truy vấn (GraphQL có thể dẫn đến vấn đề N+1 truy vấn nếu không được triển khai cẩn thận).

\subsection{Ưu điểm và Hạn chế của Synchronous Communication}
Giao tiếp đồng bộ trong kiến trúc microservices có nhiều ưu điểm và hạn chế đáng chú ý. Hiểu rõ những điểm mạnh và điểm yếu này là chìa khóa để lựa chọn mẫu giao tiếp phù hợp cho từng trường hợp sử dụng cụ thể.

Về ưu điểm, giao tiếp đồng bộ mang lại tính đơn giản và trực quan, làm cho nó dễ hiểu và triển khai, đặc biệt là với REST. Client gửi yêu cầu và nhận phản hồi theo một cách rõ ràng và trực tiếp, không đòi hỏi hiểu biết về các khái niệm phức tạp như message broker hay event handling. Ưu điểm thứ hai là phản hồi tức thì, client nhận được phản hồi ngay lập tức sau khi server xử lý yêu cầu, điều này quan trọng cho các tương tác yêu cầu tốc độ như truy vấn dữ liệu trong giao diện người dùng. Thứ ba, giao tiếp đồng bộ cho phép xử lý lỗi một cách dễ dàng, lỗi được phát hiện và xử lý ngay lập tức trong quy trình giao tiếp, giúp client có thể phản hồi lập tức cho người dùng hoặc thử lại yêu cầu. Ngoài ra, giao tiếp đồng bộ còn mang lại tính nhất quán cao, client biết ngay lập tức kết quả của yêu cầu, tạo điều kiện cho tính nhất quán dữ liệu giữa các dịch vụ. Cuối cùng, giao tiếp đồng bộ dễ dàng debug, luồng yêu cầu-phản hồi rõ ràng và dễ theo dõi, giúp nhà phát triển nhanh chóng xác định và sửa lỗi.

Tuy nhiên, giao tiếp đồng bộ cũng có những hạn chế đáng kể. Đầu tiên là coupling chặt chẽ, client và server phải đồng thời hoạt động để giao tiếp thành công, nếu server không khả dụng, client sẽ không thể tiếp tục hoạt động. Hạn chế thứ hai là hiệu suất kém trong môi trường phân tán, độ trễ mạng có thể ảnh hưởng đáng kể đến hiệu suất, đặc biệt là khi có nhiều cuộc gọi nối tiếp. Giao tiếp đồng bộ cũng thiếu khả năng mở rộng, nó có thể tạo thành bottleneck khi số lượng yêu cầu tăng cao, hạn chế khả năng mở rộng của hệ thống. Một hạn chế nghiêm trọng khác là khả năng gây ra lỗi cascade, khi một service gặp sự cố, các service gọi đến nó sẽ bị chặn, và lỗi có thể lan truyền trong toàn hệ thống, gây ra hiệu ứng domino. Cuối cùng, giao tiếp đồng bộ khiến tài nguyên bị chặn, client phải chờ đợi phản hồi trước khi tiếp tục, dẫn đến lãng phí tài nguyên và giảm throughput.

Để giảm thiểu những hạn chế này, nhiều mẫu thiết kế và kỹ thuật đã được phát triển và áp dụng trong giao tiếp đồng bộ microservices. Ví dụ, để giải quyết vấn đề lỗi cascade, các kỹ thuật như circuit breaker pattern có thể được sử dụng. Để cải thiện hiệu suất, các kỹ thuật như caching, connection pooling, và batch processing có thể được áp dụng.

\subsection{Use cases phù hợp cho Synchronous Communication}
Giao tiếp đồng bộ đặc biệt phù hợp cho một số trường hợp sử dụng cụ thể trong kiến trúc microservices. Hiểu rõ những trường hợp này giúp kiến trúc sư và nhà phát triển đưa ra quyết định đúng đắn về việc khi nào nên sử dụng giao tiếp đồng bộ.

Trường hợp sử dụng đầu tiên và quan trọng nhất là khi yêu cầu phản hồi tức thì. Trong nhiều tình huống, client cần phản hồi ngay lập tức từ server để tiếp tục xử lý hoặc cung cấp phản hồi cho người dùng. Ví dụ, khi người dùng thực hiện thanh toán, hệ thống cần kiểm tra ngay lập tức tính khả dụng của sản phẩm và xác thực thanh toán trước khi hoàn tất đơn hàng. Giao tiếp đồng bộ là lựa chọn tự nhiên cho những tình huống này do khả năng cung cấp phản hồi tức thì.

Trường hợp thứ hai là truy vấn dữ liệu đơn giản. Các hoạt động CRUD (Create, Read, Update, Delete) cơ bản mà không yêu cầu xử lý phức tạp thường phù hợp với giao tiếp đồng bộ. Ví dụ, khi ứng dụng cần lấy thông tin chi tiết về một sản phẩm để hiển thị cho người dùng, một cuộc gọi API REST đơn giản là đủ và hiệu quả.

Trường hợp thứ ba là khi tính nhất quán dữ liệu quan trọng. Trong một số quy trình nghiệp vụ, client cần biết ngay lập tức liệu hoạt động đã thành công hay không để đảm bảo tính nhất quán dữ liệu. Ví dụ, khi cập nhật thông tin quan trọng như số dư tài khoản hoặc trạng thái đặt phòng, hệ thống cần đảm bảo cập nhật đã được thực hiện thành công trước khi tiếp tục các hoạt động khác.

Trường hợp thứ tư là tương tác người dùng trực tiếp. Khi hành động của người dùng trực tiếp kích hoạt yêu cầu, và người dùng đang chờ kết quả, giao tiếp đồng bộ thường là lựa chọn tốt nhất. Điều này đảm bảo người dùng nhận được phản hồi ngay lập tức, cải thiện trải nghiệm người dùng. Ví dụ, khi người dùng nhấp vào nút "Thêm vào giỏ hàng", hệ thống cần xác nhận ngay lập tức rằng sản phẩm đã được thêm thành công.

Trường hợp cuối cùng là quy trình nghiệp vụ đơn giản. Các quy trình không yêu cầu xử lý phức tạp hoặc nhiều bước thường phù hợp với giao tiếp đồng bộ. Quy trình càng đơn giản, lợi ích của giao tiếp bất đồng bộ càng ít, và chi phí phức tạp bổ sung càng khó được biện minh.

Fowler [5] lưu ý rằng giao tiếp đồng bộ thường được ưu tiên trong các hệ thống ban đầu và những hệ thống có quy mô nhỏ, sau đó có thể phát triển thành mô hình bất đồng bộ khi hệ thống phát triển hoặc yêu cầu về khả năng mở rộng tăng lên. Điều này phản ánh sự đánh đổi giữa tính đơn giản của giao tiếp đồng bộ và khả năng mở rộng của giao tiếp bất đồng bộ.

\subsection{Case studies}
Để hiểu rõ hơn về cách giao tiếp đồng bộ được triển khai và sử dụng trong thế giới thực, chúng ta sẽ xem xét một số trường hợp nghiên cứu từ các công ty nổi tiếng đã thành công với kiến trúc microservices.

Netflix là một ví dụ điển hình về việc sử dụng giao tiếp đồng bộ trong kiến trúc microservices quy mô lớn. Netflix sử dụng Zuul làm API Gateway để quản lý các yêu cầu từ nhiều client khác nhau đến hàng trăm microservices backend [6]. Zuul cung cấp một điểm vào duy nhất cho tất cả các client, thực hiện các chức năng như routing, filtering, và load balancing. Bên cạnh Zuul, Netflix cũng sử dụng Hystrix để triển khai Circuit Breaker pattern, bảo vệ hệ thống khỏi lỗi cascade và cải thiện khả năng phục hồi. Netflix đã phát triển một hệ sinh thái các công cụ để hỗ trợ kiến trúc microservices của họ, bao gồm Eureka (service discovery), Ribbon (client-side load balancing), và Feign (declarative HTTP client). Tất cả các công cụ này đều được thiết kế để hoạt động hiệu quả trong môi trường đám mây và xử lý các thách thức của giao tiếp đồng bộ giữa các microservices.

Amazon là một ví dụ khác về việc sử dụng giao tiếp đồng bộ trong kiến trúc microservices quy mô lớn. Amazon sử dụng AWS API Gateway để quản lý và định tuyến các yêu cầu API đến các microservices backend khác nhau [7]. AWS API Gateway cung cấp nhiều tính năng như authentication, authorization, rate limiting, và caching. Nó cũng tích hợp với các dịch vụ AWS khác như Lambda, DynamoDB, và SNS, cho phép xây dựng kiến trúc serverless kết hợp với microservices. Bên cạnh đó, Amazon cũng sử dụng các service mesh như AWS App Mesh để quản lý giao tiếp giữa các microservices, cung cấp các tính năng như service discovery, traffic routing, và observation.

Uber là một ví dụ về công ty đã chuyển đổi từ REST sang gRPC để cải thiện hiệu suất giao tiếp giữa các microservices [8]. Kiến trúc của Uber bao gồm hàng nghìn microservices với hàng triệu yêu cầu mỗi giây. Uber đã phát triển một framework gọi là Ringpop, sử dụng gRPC để giao tiếp giữa các dịch vụ. Ringpop sử dụng giao thức gossip để phân phối thông tin về trạng thái của các node trong hệ thống và đưa ra quyết định về việc định tuyến yêu cầu. Uber cũng đã phát triển một thư viện gọi là TChannel, một giao thức RPC có độ tin cậy cao dựa trên Thrift, trước khi chuyển sang gRPC. Việc chuyển đổi từ REST sang gRPC đã giúp Uber cải thiện đáng kể hiệu suất và giảm kích thước của các tin nhắn trao đổi giữa các dịch vụ.

Những trường hợp nghiên cứu này minh họa cách các công ty lớn triển khai và điều chỉnh giao tiếp đồng bộ để đáp ứng nhu cầu cụ thể của họ. Chúng cũng cho thấy rằng không có một giải pháp "one-size-fits-all" cho giao tiếp microservices - mỗi công ty cần đánh giá nhu cầu của mình và lựa chọn công nghệ phù hợp nhất.

\subsection{Tương lai của Synchronous Communication trong Microservices}
Dù giao tiếp bất đồng bộ ngày càng phổ biến trong kiến trúc microservices, giao tiếp đồng bộ vẫn sẽ tiếp tục đóng vai trò quan trọng, đặc biệt là đối với các tương tác yêu cầu phản hồi tức thì.

Một xu hướng đáng chú ý trong tương lai của giao tiếp đồng bộ là Service Mesh. Công nghệ service mesh như Istio, Linkerd, và Consul Connect đang ngày càng được áp dụng để quản lý giao tiếp giữa các microservices, cung cấp các tính năng như mã hóa TLS, authentication, authorization, và observability. Service mesh hoạt động bằng cách triển khai một proxy bên cạnh mỗi instance của microservice, tạo thành một lớp infrastructure layer quản lý tất cả giao tiếp giữa các dịch vụ. Điều này cho phép tách biệt logic nghiệp vụ khỏi logic giao tiếp mạng, làm cho hệ thống dễ quản lý và bảo trì hơn.

WebAssembly (Wasm) cũng đang trở thành một công nghệ quan trọng trong tương lai của giao tiếp đồng bộ. WebAssembly đang được áp dụng trong API Gateway và service mesh để cải thiện hiệu suất và linh hoạt. Ví dụ, Envoy proxy (được sử dụng trong Istio) hỗ trợ các filter được viết bằng WebAssembly, cho phép mở rộng proxy một cách an toàn và hiệu quả mà không cần triển khai lại.

Serverless API Gateway là một xu hướng khác, với các giải pháp như AWS Lambda + API Gateway hoặc Azure Functions + API Management đang trở nên phổ biến, giúp giảm chi phí và cải thiện khả năng mở rộng. Trong mô hình này, API Gateway định tuyến yêu cầu đến các hàm serverless, chỉ được kích hoạt khi cần thiết, giúp giảm chi phí khi không có lưu lượng truy cập.

GraphQL Federation cũng đang ngày càng được áp dụng, với Apollo Federation và những công nghệ tương tự cho phép chia GraphQL schema thành nhiều microservices, mỗi dịch vụ sở hữu một phần của schema tổng thể. Điều này cho phép các đội phát triển độc lập quản lý các phần khác nhau của API mà không ảnh hưởng đến nhau.

Cuối cùng, các giao thức mới như HTTP/3 và QUIC hứa hẹn cải thiện hiệu suất giao tiếp đồng bộ, đặc biệt trong điều kiện mạng không ổn định. HTTP/3 dựa trên QUIC, một giao thức truyền tải mới phát triển bởi Google, cung cấp những cải tiến đáng kể về độ trễ và độ tin cậy so với TCP.

\subsection{Kết luận}
Giao tiếp đồng bộ one-to-one, đặc biệt là mẫu Request/Response, vẫn là một phần không thể thiếu trong kiến trúc microservices hiện đại. Mặc dù có những hạn chế về coupling và khả năng mở rộng, giao tiếp đồng bộ vẫn cung cấp nhiều lợi ích quan trọng như đơn giản, trực quan, và phản hồi tức thì.

Việc lựa chọn công nghệ triển khai phù hợp (REST, gRPC, GraphQL) phụ thuộc vào nhu cầu cụ thể của dự án, bao gồm yêu cầu về hiệu suất, khả năng mở rộng, và độ phức tạp. REST vẫn là lựa chọn phổ biến nhất do tính đơn giản và khả năng tương thích rộng rãi, nhưng gRPC và GraphQL đang ngày càng được áp dụng cho các trường hợp sử dụng cụ thể yêu cầu hiệu suất cao hoặc truy vấn dữ liệu phức tạp.

Trong thực tế, một hệ thống microservices hiệu quả thường kết hợp cả giao tiếp đồng bộ và bất đồng bộ, sử dụng mẫu phù hợp cho từng trường hợp sử dụng cụ thể. Hiểu rõ ưu điểm và hạn chế của mỗi mẫu giao tiếp là chìa khóa để thiết kế một kiến trúc microservices thành công.

Với sự phát triển liên tục của công nghệ và sự xuất hiện của các mẫu thiết kế mới, giao tiếp đồng bộ trong kiến trúc microservices sẽ tiếp tục phát triển và cải thiện, cung cấp nhiều tùy chọn hơn cho các kiến trúc sư và nhà phát triển để xây dựng các hệ thống phân tán mạnh mẽ và hiệu quả.
\section{Asynchronous Communication Patterns (one-to-one)}
Giao tiếp bất đồng bộ one-to-one là một phương pháp quan trọng trong kiến trúc vi dịch vụ, tạo nên sự tách rời (decoupling) giữa các dịch vụ và cải thiện khả năng mở rộng của hệ thống. Không giống như giao tiếp đồng bộ, trong giao tiếp bất đồng bộ one-to-one, vi dịch vụ gửi tin nhắn không cần đợi phản hồi tức thì từ vi dịch vụ nhận. Phần này sẽ tìm hiểu chi tiết về các mẫu giao tiếp bất đồng bộ one-to-one phổ biến, cùng với các công nghệ triển khai và các trường hợp sử dụng thực tế.

\subsection{Cơ chế hoạt động}
Giao tiếp bất đồng bộ one-to-one hoạt động theo mô hình khác biệt so với giao tiếp đồng bộ, với đặc điểm chính là sự phi chặn (non-blocking) trong quá trình truyền tải tin nhắn. Thay vì chờ đợi phản hồi, sender tiếp tục thực hiện các hoạt động khác sau khi gửi tin nhắn, và receiver xử lý tin nhắn khi có khả năng. Cơ chế này thường được triển khai thông qua một thành phần trung gian, có thể là message broker hoặc queue, đóng vai trò lưu trữ tạm thời tin nhắn khi receiver chưa sẵn sàng nhận \cite{hohpe2004}.

Quá trình giao tiếp bất đồng bộ one-to-one thường tuân theo các bước sau: sender chuẩn bị tin nhắn với đầy đủ thông tin cần thiết; sender gửi tin nhắn đến hệ thống trung gian (message broker hoặc queue) mà không đợi phản hồi; sender tiếp tục xử lý các tác vụ khác; tin nhắn được lưu trữ trong hệ thống trung gian cho đến khi receiver sẵn sàng; receiver lấy tin nhắn từ hệ thống trung gian khi có thể; receiver xử lý tin nhắn theo logic nghiệp vụ của nó; tùy thuộc vào mẫu cụ thể, receiver có thể hoặc không gửi phản hồi.

Một khía cạnh quan trọng của giao tiếp bất đồng bộ là cơ chế delivery guarantee. Có ba mức độ đảm bảo phổ biến: at-most-once (tin nhắn có thể bị mất nhưng không bao giờ được gửi hơn một lần), at-least-once (tin nhắn được đảm bảo gửi ít nhất một lần, nhưng có thể trùng lặp), và exactly-once (tin nhắn được đảm bảo gửi chính xác một lần). Mức độ đảm bảo nào được sử dụng phụ thuộc vào yêu cầu nghiệp vụ và khả năng của hệ thống messaging \cite{newman2015}.

Một cơ chế khác cần xem xét là message ordering. Trong một số trường hợp, thứ tự xử lý tin nhắn là quan trọng (ví dụ, tin nhắn cập nhật giá trị phải được xử lý sau tin nhắn tạo giá trị). Một số message broker đảm bảo thứ tự tin nhắn trong một partition hoặc queue cụ thể, trong khi những broker khác có thể không cung cấp đảm bảo này.

\subsection{One-way Notifications Pattern}
Mẫu One-way Notifications là hình thức đơn giản nhất của giao tiếp bất đồng bộ one-to-one. Trong mẫu này, một vi dịch vụ gửi tin nhắn tới một vi dịch vụ khác mà không mong đợi phản hồi. Đây là mẫu hoàn toàn "fire-and-forget", trong đó sender không quan tâm đến việc tin nhắn được xử lý như thế nào hoặc khi nào \cite{hohpe2004}.

Mẫu One-way Notifications đặc biệt hữu ích cho các thông báo sự kiện không yêu cầu phản hồi, chẳng hạn như cập nhật trạng thái, ghi nhật ký, hoặc theo dõi hoạt động. Chẳng hạn, khi một đơn hàng được cập nhật, dịch vụ đơn hàng có thể gửi thông báo cho dịch vụ thông báo mà không cần đợi phản hồi.

Triển khai mẫu One-way Notifications thường đơn giản hơn so với các mẫu bất đồng bộ khác, vì không cần cơ chế tương quan giữa yêu cầu và phản hồi. Sender chỉ cần đặt tin nhắn vào hàng đợi hoặc chủ đề, và receiver xử lý nó khi sẵn sàng.

Các công nghệ phổ biến cho One-way Notifications bao gồm RabbitMQ, Apache Kafka, Amazon SQS, và Google Cloud Pub/Sub. Ví dụ, trong RabbitMQ, sender có thể xuất bản tin nhắn tới một exchange, và RabbitMQ định tuyến tin nhắn tới hàng đợi phù hợp dựa trên routing key. Receiver đăng ký hàng đợi và xử lý tin nhắn khi chúng đến. Apache Kafka hoạt động tương tự, với sender xuất bản tin nhắn tới chủ đề, và receiver đăng ký chủ đề để nhận tin nhắn.

Theo Newman \cite{newman2015}, một khía cạnh quan trọng của One-way Notifications là xử lý lỗi. Vì sender không đợi phản hồi, việc xử lý lỗi phải được xử lý khác với giao tiếp đồng bộ. Các tiếp cận phổ biến bao gồm sử dụng Dead Letter Queues (DLQ) để lưu trữ tin nhắn không thể xử lý, các cơ chế retry để thử lại tin nhắn lỗi, và các cơ chế logging và monitoring để phát hiện và giải quyết vấn đề.

Richardson \cite{richardson2019} đề xuất một số best practices khi triển khai One-way Notifications, bao gồm sử dụng message versioning để hỗ trợ backward và forward compatibility, triển khai idempotent message handling để tránh xử lý lặp lại tin nhắn, và sử dụng message acknowledgements để đảm bảo tin nhắn được xử lý thành công.

\subsection{Message Queue Pattern}
Message Queue là một mẫu cơ bản trong giao tiếp bất đồng bộ, trong đó các tin nhắn được gửi tới một hàng đợi trung gian, nơi chúng được lưu trữ cho đến khi được xử lý bởi consumer. Mẫu này hỗ trợ cả One-way Notifications và các mẫu bất đồng bộ khác, và cung cấp nhiều lợi ích như decoupling, buffering, và reliable delivery \cite{hohpe2004}.

Trong kiến trúc Message Queue, thường có ba thành phần chính: Producer đặt tin nhắn vào hàng đợi; Queue lưu trữ tin nhắn cho đến khi chúng được xử lý; và Consumer lấy tin nhắn từ hàng đợi và xử lý chúng. Một tính năng quan trọng của Message Queue là nó cho phép asynchronous consumption, có nghĩa là tin nhắn không cần được xử lý ngay lập tức khi chúng đến. Điều này giúp xử lý các peak load và bảo vệ các service khi chúng đang quá tải.

Các công nghệ Message Queue phổ biến bao gồm RabbitMQ, Apache ActiveMQ, Amazon SQS, và Microsoft Azure Service Bus. Mỗi công nghệ có những đặc điểm và ưu điểm riêng. Ví dụ, RabbitMQ hỗ trợ nhiều mẫu messaging như work queues, publish/subscribe, routing, và topics. Nó cũng cung cấp các tính năng như message acknowledgement, durability, và fair dispatch. Amazon SQS là một dịch vụ hàng đợi tin nhắn được quản lý, cung cấp các tính năng như at-least-once delivery, message retention, và visibility timeout.

Một khía cạnh quan trọng của Mecassage Queue là khả năng mở rộng. Có hai cách chính để mở rộng hệ thống Message Queue: horizontal scaling (thêm nhiều consumer để xử lý nhiều tin nhắn hơn song song) và partitioning (chia hàng đợi thành nhiều partition, mỗi partition được xử lý bởi một consumer).

Hohpe và Woolf \cite{hohpe2004} mô tả nhiều mẫu messaging chi tiết hơn liên quan đến Message Queue, bao gồm Competing Consumers (nhiều consumer cạnh tranh để xử lý tin nhắn từ một hàng đợi), Message Dispatcher (định tuyến tin nhắn tới consumer cụ thể dựa trên một số tiêu chí), và Priority Queue (tin nhắn ưu tiên cao được xử lý trước).

Trong kiến trúc vi dịch vụ, Message Queue thường được sử dụng để xử lý các tác vụ nặng hoặc tốn thời gian bất đồng bộ. Ví dụ, khi một người dùng đăng ký, service đăng ký có thể đặt một tin nhắn vào hàng đợi để gửi email xác nhận, trong khi vẫn phản hồi ngay lập tức cho người dùng.

\subsection{Ưu điểm và Hạn chế của Asynchronous Communication (one-to-one)}
Giao tiếp bất đồng bộ one-to-one có nhiều ưu điểm và hạn chế đáng chú ý so với giao tiếp đồng bộ, hiểu rõ những điểm mạnh và điểm yếu này là chìa khóa để lựa chọn mẫu giao tiếp phù hợp.

Về ưu điểm, decoupling là lợi ích lớn nhất của giao tiếp bất đồng bộ. Các service giao tiếp không cần biết về nhau hoặc thậm chí hoạt động đồng thời. Sender có thể gửi tin nhắn ngay cả khi receiver đang ngoại tuyến, và receiver có thể xử lý tin nhắn khi sẵn sàng. Điều này làm giảm đáng kể coupling giữa các service. Ưu điểm thứ hai là khả năng phục hồi cải thiện. Giao tiếp bất đồng bộ có thể chịu được lỗi service tạm thời, vì tin nhắn có thể lưu trữ trong hàng đợi cho đến khi receiver khả dụng. Điều này giúp hệ thống duy trì hoạt động ngay cả khi một số thành phần gặp sự cố. Thứ ba, giao tiếp bất đồng bộ cung cấp khả năng mở rộng tốt hơn, cho phép mở rộng linh hoạt hơn bằng cách thêm nhiều receiver để xử lý nhiều tin nhắn hơn song song, mà không cần thay đổi sender. Ngoài ra, bất đồng bộ giúp cải thiện khả năng phản hồi và hiệu suất, vì sender không cần đợi receiver xử lý yêu cầu, do đó giảm thời gian chờ và cải thiện trải nghiệm người dùng. Giao tiếp bất đồng bộ cũng cung cấp buffering và smoothing, hàng đợi tin nhắn có thể hấp thụ đỉnh tải và bảo vệ service khỏi quá tải. Cuối cùng, giao tiếp bất đồng bộ cải thiện sử dụng tài nguyên, vì service không chặn tài nguyên chờ phản hồi, cho phép sử dụng tài nguyên hiệu quả hơn.

Tuy nhiên, giao tiếp bất đồng bộ cũng có những hạn chế đáng kể. Đầu tiên là độ phức tạp gia tăng. Giao tiếp bất đồng bộ thường đòi hỏi cơ sở hạ tầng bổ sung (như message broker) và logic phức tạp hơn để xử lý tin nhắn, quản lý lỗi, và đảm bảo tin cậy, làm tăng độ phức tạp của hệ thống. Hạn chế thứ hai là tính nhất quán yếu hơn. Bất đồng bộ thường dẫn đến eventual consistency thay vì strong consistency, nghĩa là có độ trễ trước khi tất cả các service phản ánh cùng một trạng thái. Điều này có thể là một thách thức đối với các hệ thống yêu cầu tính nhất quán cao. Thứ ba, debugging và tracing có thể phức tạp hơn trong giao tiếp bất đồng bộ, vì luồng tin nhắn không trực quan như yêu cầu-phản hồi đồng bộ. Theo dõi tin nhắn qua nhiều service và hàng đợi có thể là một thách thức. Giao tiếp bất đồng bộ cũng gặp phải vấn đề về độ tin cậy và mất tin nhắn. Mặc dù nhiều message broker cung cấp các cơ chế đảm bảo độ tin cậy như message acknowledgement và persistent messaging, nhưng vẫn có nguy cơ mất tin nhắn hoặc trùng lặp tin nhắn. Cuối cùng, độ trễ end-to-end cao hơn. Mặc dù giao tiếp bất đồng bộ cải thiện độ phản hồi của sender, độ trễ end-to-end (từ khi gửi tin nhắn đến khi hoàn thành xử lý) thường cao hơn so với giao tiếp đồng bộ.

Để giảm thiểu những hạn chế này, nhiều kỹ thuật và mẫu thiết kế đã được phát triển, Ví dụ, để giải quyết vấn đề tính nhất quán yếu, mẫu thiết kế như Saga và Event Sourcing có thể được sử dụng để quản lý giao dịch phân tán và duy trì tính nhất quán trong hệ thống bất đồng bộ. Để cải thiện khả năng debugging và tracing, các công cụ như Zipkin và Jaeger có thể được sử dụng để theo dõi tin nhắn qua hệ thống phân tán. Và để cải thiện độ tin cậy, các message broker cung cấp các tính năng như message acknowledgement, dead letter queues, và persistent messaging.

\subsection{Use cases phù hợp cho Asynchronous Communication (one-to-one)}
Giao tiếp bất đồng bộ one-to-one đặc biệt phù hợp cho một số trường hợp sử dụng cụ thể trong kiến trúc vi dịch vụ. Hiểu rõ những trường hợp này giúp kiến trúc sư và nhà phát triển đưa ra quyết định đúng đắn về việc khi nào nên sử dụng giao tiếp bất đồng bộ.

Trường hợp sử dụng đầu tiên và quan trọng nhất là xử lý background. Các tác vụ tốn thời gian hoặc tài nguyên không cần phản hồi ngay lập tức là ứng viên lý tưởng cho giao tiếp bất đồng bộ. Ví dụ, khi người dùng tải lên video, việc xử lý video (như chuyển đổi định dạng, tạo thumbnail) có thể được thực hiện bất đồng bộ trong background, trong khi người dùng nhận được phản hồi ngay lập tức rằng quá trình tải lên đã hoàn tất.

Trường hợp thứ hai là xử lý số lượng lớn. Khi một service cần xử lý số lượng lớn yêu cầu, giao tiếp bất đồng bộ cho phép điều tiết tải thông qua buffering và sử dụng nhiều receiver để xử lý song song. Ví dụ, hệ thống xử lý log hoặc các ứng dụng phân tích dữ liệu lớn có thể sử dụng giao tiếp bất đồng bộ để thu thập và xử lý dữ liệu từ nhiều nguồn.

Trường hợp thứ ba là cải thiện phản hồi người dùng. Trong các tình huống mà phản hồi ngay lập tức cho người dùng là quan trọng, nhưng xử lý hoàn chỉnh có thể hoàn thành sau, giao tiếp bất đồng bộ cho phép hệ thống phản hồi ngay lập tức trong khi tiếp tục xử lý trong background. Ví dụ, khi đặt đơn hàng trực tuyến, hệ thống có thể xác nhận đơn hàng ngay lập tức và xử lý thanh toán, gửi email xác nhận, và cập nhật kho bất đồng bộ.

Trường hợp thứ tư là long-running processes. Các quy trình nghiệp vụ kéo dài, liên quan đến nhiều bước hoặc tương tác với nhiều service hay hệ thống bên ngoài, phù hợp với giao tiếp bất đồng bộ. Ví dụ, quy trình onboarding khách hàng có thể yêu cầu nhiều kiểm tra (như xác minh danh tính, kiểm tra tín dụng) có thể mất nhiều thời gian, trong khi khách hàng không nên phải đợi quá trình này hoàn tất.

Trường hợp cuối cùng là integration với hệ thống bên ngoài. Khi giao tiếp với hệ thống bên ngoài không trong quyền kiểm soát của bạn, giao tiếp bất đồng bộ cung cấp sự cách ly và khả năng phục hồi. Nếu hệ thống bên ngoài chậm hoặc không khả dụng, hệ thống của bạn có thể tiếp tục hoạt động và xử lý yêu cầu khi hệ thống bên ngoài trở lại.

Bên cạnh đó, các hệ thống cần khả năng chịu lỗi cao và lỏng lẻo về tính nhất quán dữ liệu thường phù hợp với giao tiếp bất đồng bộ \cite{fowler2002}. Tiếp cận bất đồng bộ cho phép hệ thống tiếp tục hoạt động ngay cả khi một số thành phần gặp sự cố, mặc dù có thể dẫn đến tính nhất quán yếu hơn (eventual consistency).

\subsection{Case studies}
Để hiểu rõ hơn về cách giao tiếp bất đồng bộ one-to-one được triển khai và sử dụng trong thế giới thực, chúng ta sẽ xem xét một số trường hợp nghiên cứu từ các công ty nổi tiếng.

LinkedIn là một ví dụ nổi bật về việc sử dụng giao tiếp bất đồng bộ one-to-one trong kiến trúc vi dịch vụ quy mô lớn. Họ đã phát triển và sử dụng Apache Kafka, một nền tảng streaming phân tán, để xử lý hàng nghìn tỉ tin nhắn mỗi ngày \cite{goodhope2012}. LinkedIn sử dụng Kafka cho nhiều mục đích, bao gồm activity tracking, xử lý sự kiện real-time, và đồng bộ hóa dữ liệu giữa các hệ thống khác nhau. Ví dụ, khi một người dùng thực hiện hành động như cập nhật profile hoặc kết nối với người dùng khác, hành động này được ghi lại như một sự kiện và được gửi tới Kafka. Các service khác nhau, như service đề xuất kết nối hoặc service thông báo, sau đó xử lý sự kiện này để cập nhật dữ liệu và cung cấp chức năng của chúng.

PayPal cũng là một ví dụ đáng chú ý về việc sử dụng giao tiếp bất đồng bộ one-to-one trong một hệ thống xử lý thanh toán cần độ tin cậy cao. PayPal đã chuyển đổi từ một kiến trúc monolithic sang vi dịch vụ, sử dụng Apache Kafka làm platform tin nhắn chính \cite{raman2016}. Trong kiến trúc này, các service như xử lý thanh toán, phát hiện gian lận, và thông báo người dùng giao tiếp bất đồng bộ thông qua Kafka. Ví dụ, khi một giao dịch được khởi tạo, một sự kiện được gửi tới Kafka và được xử lý bởi nhiều service khác nhau, bao gồm service xác thực, service phát hiện gian lận, và service xử lý giao dịch. Mỗi service thực hiện chức năng cụ thể của nó và có thể tạo ra các sự kiện khác, tạo thành một luồng xử lý transaction.

Uber cũng sử dụng giao tiếp bất đồng bộ one-to-one rộng rãi trong kiến trúc vi dịch vụ của họ. Uber đã xây dựng một platform gọi là Cadence, một service orchestration engine cho phép viết workflow code như code đồng bộ, nhưng thực thi chúng bất đồng bộ và phân tán \cite{fateev2017}. Cadence được sử dụng cho nhiều use-case trong Uber, bao gồm matching riders với drivers, thanh toán, và onboarding drivers. Ví dụ, khi một yêu cầu chuyến đi được tạo, Cadence khởi tạo một workflow để tìm driver phù hợp, đặt lịch chuyến đi, xử lý thanh toán, và cập nhật trạng thái của rider và driver. Mỗi bước trong workflow này được thực hiện bất đồng bộ, cho phép quá trình tiếp tục ngay cả khi một số service tạm thời không khả dụng.

Những trường hợp nghiên cứu này minh họa cách các công ty lớn triển khai giao tiếp bất đồng bộ one-to-one để xây dựng hệ thống phân tán có khả năng mở rộng, linh hoạt, và đáng tin cậy. Chúng cũng cho thấy rằng giao tiếp bất đồng bộ đặc biệt hữu ích cho các hệ thống xử lý số lượng lớn giao dịch hoặc sự kiện, và cho các quy trình nghiệp vụ phức tạp liên quan đến nhiều service.
\section{Mẫu giao tiếp bất đồng bộ (một đối nhiều)}
Giao tiếp bất đồng bộ một đối nhiều là mô hình quan trọng trong kiến trúc vi dịch vụ, cho phép một dịch vụ gửi thông điệp đến nhiều dịch vụ nhận khác nhau cùng một lúc. Không giống như mô hình một đối một, trong giao tiếp một đối nhiều, một thông điệp được phân phối đến nhiều người nhận mà không cần biết trước danh tính hoặc số lượng người nhận. Mô hình này đặc biệt hữu ích trong các hệ thống phân tán quy mô lớn, nơi cần thông báo nhiều thành phần về các sự kiện quan trọng \cite{newman2015}. Phần này sẽ tìm hiểu chi tiết về các mẫu giao tiếp bất đồng bộ một đối nhiều phổ biến, cùng với các công nghệ triển khai và các trường hợp sử dụng thực tế.
\subsection{Cơ chế hoạt động}
Giao tiếp bất đồng bộ một đối nhiều hoạt động theo nguyên tắc phân phối thông điệp từ một nguồn đến nhiều đích mà không yêu cầu sự tương tác trực tiếp giữa chúng. Thay vì thiết lập kết nối trực tiếp với từng người nhận, người gửi phát hành thông điệp đến một kênh trung gian, và thông điệp này được phân phối đến tất cả các dịch vụ đã đăng ký với kênh đó \cite{hohpe2004}.

Quá trình giao tiếp bất đồng bộ một đối nhiều thường tuân theo các bước sau: nhà xuất bản (publisher) chuẩn bị thông điệp; nhà xuất bản gửi thông điệp đến kênh hoặc topic trung gian; hệ thống trung gian nhận thông điệp và lưu trữ tạm thời; hệ thống trung gian xác định tất cả người đăng ký (subscribers) cho kênh hoặc topic đó; hệ thống trung gian phân phối thông điệp đến tất cả người đăng ký phù hợp; các người đăng ký nhận và xử lý thông điệp độc lập với nhau.

Một yếu tố quan trọng trong giao tiếp một đối nhiều là khả năng mở rộng động số lượng người nhận. Các dịch vụ mới có thể đăng ký nhận thông điệp mà không cần thay đổi logic của nhà xuất bản, tạo nên sự linh hoạt cao cho hệ thống. Đồng thời, nhiều hệ thống trung gian còn hỗ trợ các cơ chế lọc thông điệp, cho phép người đăng ký chỉ nhận những thông điệp phù hợp với tiêu chí cụ thể \cite{richardson2019}.

Cũng như trong giao tiếp một đối một, các cơ chế đảm bảo độ tin cậy của thông điệp như at-most-once, at-least-once hoặc exactly-once đều có thể được áp dụng tùy theo yêu cầu của hệ thống. Tuy nhiên, do tính chất phân phối đến nhiều người nhận, việc đảm bảo thông điệp được nhận bởi tất cả người đăng ký có thể phức tạp hơn \cite{aksakalli2021}.

\subsection{Mẫu Publish/Subscribe}
Publish/Subscribe (Pub/Sub) là mẫu giao tiếp không đồng bộ một đối nhiều phổ biến nhất trong kiến trúc vi dịch vụ. Trong mô hình này, nhà xuất bản (publisher) không gửi thông điệp trực tiếp đến người nhận cụ thể, mà thay vào đó phát hành (publish) thông điệp đến một kênh (topic hoặc channel). Các dịch vụ quan tâm đến loại thông điệp này đăng ký (subscribe) vào kênh đó và nhận tất cả thông điệp được phát hành \cite{hohpe2004}.

Wolff \cite{wolff2016} phân biệt hai biến thể chính của mô hình Pub/Sub: Topic-based Pub/Sub, trong đó người đăng ký đăng ký vào các kênh cụ thể và nhận tất cả thông điệp được phát hành đến kênh đó; và Content-based Pub/Sub, trong đó người đăng ký định nghĩa các tiêu chí lọc và chỉ nhận thông điệp phù hợp với tiêu chí đã định.

Các công nghệ phổ biến để triển khai Pub/Sub bao gồm Apache Kafka, RabbitMQ, Google Cloud Pub/Sub, Amazon SNS và Redis Pub/Sub. Ví dụ, Apache Kafka tổ chức thông điệp thành các topic và lưu trữ chúng trong log phân tán, cho phép người đăng ký xử lý thông điệp theo tốc độ riêng của họ và thậm chí phát lại thông điệp cũ. RabbitMQ sử dụng mô hình exchange-queue, trong đó các exchange nhận thông điệp từ nhà xuất bản và định tuyến chúng đến các queue dựa trên các quy tắc khác nhau, và người đăng ký nhận thông điệp từ queue \cite{jun2018}.

Pub/Sub đặc biệt hữu ích cho các trường hợp như cập nhật trạng thái và sự kiện hệ thống, phân tích và giám sát, đồng bộ hóa dữ liệu giữa các dịch vụ, và IoT. Chẳng hạn, khi một đơn hàng được cập nhật, dịch vụ đơn hàng có thể phát hành sự kiện "order\_updated" đến một topic, và nhiều dịch vụ khác nhau như thanh toán, thông báo, và kho hàng có thể đăng ký topic này để cập nhật trạng thái tương ứng của chúng \cite{newman2015}.

Uber sử dụng mô hình Pub/Sub để xây dựng nền tảng sự kiện quy mô lớn, xử lý hàng trăm tỷ sự kiện mỗi ngày. Kiến trúc của họ sử dụng Apache Kafka làm nền tảng chính, cho phép các dịch vụ phát hành và đăng ký các sự kiện mà không cần biết về nhau. Khi một chuyến đi được tạo, sự kiện "trip\_created" được phát hành và nhiều dịch vụ đăng ký sự kiện này để thực hiện các chức năng khác nhau như thông báo cho tài xế, tính toán giá thành, và thu thập dữ liệu phân tích \cite{beyer2018}.

\subsection{Mẫu Event Sourcing}
Event Sourcing là một mẫu kiến trúc trong đó thay vì lưu trữ trạng thái hiện tại của đối tượng, hệ thống lưu trữ chuỗi các sự kiện mô tả những thay đổi đã xảy ra với đối tượng đó theo thời gian. Trạng thái hiện tại được tạo ra bằng cách phát lại các sự kiện này \cite{fowler2002}.

Khi kết hợp với kiến trúc vi dịch vụ, Event Sourcing thường được triển khai như một mô hình giao tiếp không đồng bộ một đối nhiều, trong đó các sự kiện được lưu trữ trong Event Store và nhiều dịch vụ có thể đăng ký để nhận thông báo về các sự kiện mới. Richardson \cite{richardson2019} chỉ ra rằng Event Sourcing thường được sử dụng kết hợp với Command Query Responsibility Segregation (CQRS) để tách biệt hoạt động đọc và ghi dữ liệu.

Event Sourcing có nhiều ưu điểm trong kiến trúc vi dịch vụ, bao gồm khả năng lưu trữ lịch sử hoàn chỉnh của tất cả thay đổi (hữu ích cho kiểm toán và phân tích), khả năng phục hồi trạng thái hệ thống tại bất kỳ thời điểm nào, và khả năng phát triển song song các dịch vụ dựa trên luồng sự kiện chung.

Tuy nhiên, mẫu này cũng có những thách thức như độ phức tạp cao trong triển khai, vấn đề về schema evolution khi cấu trúc sự kiện thay đổi, và chi phí lưu trữ lớn do phải lưu trữ tất cả sự kiện. Event Sourcing phù hợp cho các hệ thống tài chính, quản lý đơn hàng, hệ thống quản lý nội dung, và các ứng dụng yêu cầu khả năng kiểm toán cao \cite{richardson2019}.

PayPal đã áp dụng mô hình Event Sourcing kết hợp với kiến trúc vi dịch vụ để xây dựng nền tảng thanh toán mới. Mỗi giao dịch thanh toán được mô hình hóa như một chuỗi các sự kiện (payment\_initiated, payment\_authorized, payment\_captured, v.v.), và nhiều dịch vụ khác nhau đăng ký để nhận thông báo về các sự kiện mới để thực hiện các chức năng như phát hiện gian lận, thông báo cho người dùng, và cập nhật báo cáo tài chính \cite{raman2016}.

\subsection{Ưu điểm và Hạn chế của giao tiếp bất đồng bộ (một đối nhiều)}
Giao tiếp bất đồng bộ một đối nhiều mang lại nhiều ưu điểm đáng kể so với các mô hình giao tiếp khác trong kiến trúc vi dịch vụ, nhưng cũng đi kèm với một số hạn chế cần được xem xét kỹ lưỡng.

Về ưu điểm, đầu tiên phải kể đến sự tách rời cao giữa nhà xuất bản và người đăng ký. Nhà xuất bản không cần biết danh tính hoặc số lượng người đăng ký, cho phép thêm hoặc xóa người đăng ký mà không ảnh hưởng đến nhà xuất bản. Điều này tạo ra hệ thống linh hoạt, có thể mở rộng dễ dàng \cite{newman2015}. Thứ hai, mô hình này hỗ trợ khả năng mở rộng động (dynamic scalability). Có thể thêm người đăng ký mới bất kỳ lúc nào để xử lý tải tăng lên hoặc cung cấp chức năng mới mà không cần thay đổi nhà xuất bản. Ưu điểm thứ ba là khả năng xử lý đồng thời (parallel processing). Nhiều người đăng ký có thể xử lý cùng một thông điệp song song, tăng hiệu suất tổng thể của hệ thống \cite{wolff2016}. Giao tiếp một đối nhiều cũng mang lại khả năng chịu lỗi cao (fault tolerance). Lỗi trong một người đăng ký không ảnh hưởng đến nhà xuất bản hoặc các người đăng ký khác, tăng tính ổn định của hệ thống \cite{richardson2019}. Cuối cùng, mô hình này rất phù hợp cho xử lý sự kiện phân tán (distributed event processing). Cho phép xây dựng hệ thống phản ứng với các sự kiện một cách linh hoạt và mở rộng \cite{hohpe2004}.

Tuy nhiên, giao tiếp bất đồng bộ một đối nhiều cũng có những hạn chế đáng kể. Đầu tiên là đồ phức tạp cao hơn. Triển khai mô hình này thường đòi hỏi cơ sở hạ tầng bổ sung (như message broker hoặc hệ thống streaming) và logic phức tạp hơn để quản lý việc phát hành và đăng ký \cite{newman2015}. Thứ hai, debugging và tracing khó khăn hơn. Theo dõi luồng thông điệp qua nhiều dịch vụ và trong môi trường phân tán có thể rất phức tạp, đặc biệt khi xử lý các lỗi \cite{wolff2016}. Giao tiếp một đối nhiều cũng dẫn đến tính nhất quán yếu hơn (eventual consistency). Các dịch vụ khác nhau có thể xử lý thông điệp với tốc độ khác nhau, dẫn đến trạng thái hệ thống không đồng nhất trong một khoảng thời gian \cite{richardson2019}. Ngoài ra, mô hình này có thể gặp vấn đề về quản lý thứ tự thông điệp. Đảm bảo thông điệp được xử lý theo đúng thứ tự có thể khó khăn, đặc biệt trong môi trường phân tán \cite{aksakalli2021}. Cuối cùng, giao tiếp một đối nhiều có thể dẫn đến độ trễ đầu cuối cao hơn. Mặc dù nhà xuất bản có thể phản hồi nhanh, thời gian để tất cả người đăng ký hoàn thành xử lý có thể đáng kể \cite{jun2018}.

Để giảm thiểu những hạn chế này, nhiều kỹ thuật và mẫu thiết kế đã được phát triển. Chẳng hạn, để cải thiện khả năng debugging và tracing, các công cụ như Zipkin và Jaeger có thể được sử dụng để theo dõi thông điệp qua hệ thống phân tán. Để quản lý thứ tự thông điệp, các nền tảng như Kafka hỗ trợ partitioning và ordering guarantees. Và để giải quyết vấn đề tính nhất quán yếu, các kỹ thuật như snapshot, event versioning, và materialized views có thể được áp dụng.

\subsection{Use cases phù hợp cho giao tiếp bất đồng bộ (một đối nhiều)}
Giao tiếp bất đồng bộ một đối nhiều đặc biệt phù hợp cho một số trường hợp sử dụng cụ thể trong kiến trúc vi dịch vụ. Hiểu rõ những trường hợp này giúp kiến trúc sư và nhà phát triển đưa ra quyết định đúng đắn về việc khi nào nên sử dụng mô hình giao tiếp này.

Trường hợp sử dụng đầu tiên và phổ biến nhất là thông báo sự kiện hệ thống (system event notification). Khi một sự kiện quan trọng xảy ra (như tạo đơn hàng, cập nhật hồ sơ người dùng, hoặc thay đổi trạng thái), nhiều dịch vụ khác nhau có thể cần biết về sự kiện đó. Ví dụ, khi một đơn hàng được tạo, dịch vụ thanh toán, dịch vụ thông báo, dịch vụ kho hàng, và dịch vụ phân tích đều cần được thông báo \cite{newman2015}.

Trường hợp thứ hai là tích hợp dữ liệu cross-service. Khi dữ liệu được tạo hoặc cập nhật trong một dịch vụ, các dịch vụ khác có thể cần sao chép hoặc chuyển đổi dữ liệu đó cho mục đích riêng của chúng. Ví dụ, khi dịch vụ người dùng cập nhật thông tin người dùng, dịch vụ tìm kiếm và dịch vụ đề xuất cần cập nhật chỉ mục của chúng \cite{richardson2019}.

Trường hợp thứ ba là xử lý dữ liệu thời gian thực (real-time data processing). Đối với các hệ thống cần phân tích hoặc phản ứng với dữ liệu thời gian thực, mô hình một đối nhiều cho phép nhiều dịch vụ xử lý song song cùng một luồng dữ liệu. Ví dụ, dữ liệu click stream của người dùng có thể được phân tích bởi nhiều dịch vụ để phát hiện hành vi bất thường, cập nhật mô hình ML, và cung cấp phân tích theo thời gian thực \cite{goodhope2012}.

Trường hợp thứ tư là logging và monitoring phân tán. Các hệ thống phân tán cần thu thập log và metric từ nhiều dịch vụ cho mục đích giám sát và phân tích. Mô hình một đối nhiều cho phép các dịch vụ phát hành log và metric đến một kênh trung tâm, nơi chúng có thể được tiêu thụ bởi nhiều dịch vụ như lưu trữ log, cảnh báo, và phân tích \cite{aksakalli2021}.

Trường hợp thứ năm là cung cấp feed dữ liệu cho các hệ thống bên ngoài. Khi nhiều hệ thống bên ngoài (như ứng dụng đối tác, dịch vụ phân tích, hoặc hệ thống báo cáo) cần truy cập vào cùng một luồng dữ liệu, mô hình một đối nhiều cho phép phân phối dữ liệu một cách hiệu quả mà không tạo thêm gánh nặng cho hệ thống nguồn \cite{beyer2018}.

Ngoài ra, các use case liên quan đến Internet of Things (IoT) cũng rất phù hợp với giao tiếp một đối nhiều. Đối với các hệ thống IoT, nơi dữ liệu từ nhiều thiết bị cần được thu thập và xử lý bởi nhiều dịch vụ, mô hình một đối nhiều cung cấp cách hiệu quả để phân phối dữ liệu đến tất cả các dịch vụ quan tâm \cite{indrasiri2020}.

Cuối cùng, các hệ thống cần khả năng chịu lỗi cao và tách rời mạnh mẽ giữa các thành phần cũng rất phù hợp với mô hình một đối nhiều. Tiếp cận này cho phép các dịch vụ hoạt động độc lập với nhau, làm tăng tính ổn định và độ tin cậy của hệ thống tổng thể \cite{fowler2002}.

\subsection{Kết luận}
Giao tiếp bất đồng bộ một đối nhiều đóng vai trò quan trọng trong các kiến trúc vi dịch vụ hiện đại, cung cấp khả năng phân phối thông điệp từ một nguồn đến nhiều đích một cách hiệu quả. Các mẫu như Publish/Subscribe, Event Sourcing, Message Broker với Exchange Routing, và Streaming Platform cung cấp các cách khác nhau để triển khai giao tiếp một đối nhiều, mỗi mẫu với những ưu điểm và trường hợp sử dụng phù hợp riêng.

Mô hình này mang lại nhiều lợi ích như sự tách rời cao, khả năng mở rộng động, xử lý đồng thời, và khả năng chịu lỗi tốt. Tuy nhiên, nó cũng đặt ra những thách thức về độ phức tạp, debugging, tính nhất quán dữ liệu, và quản lý thứ tự thông điệp. Với các công nghệ và mẫu thiết kế phù hợp, những thách thức này có thể được giải quyết, cho phép xây dựng các hệ thống vi dịch vụ mạnh mẽ và có khả năng mở rộng.

Các trường hợp sử dụng điển hình cho giao tiếp một đối nhiều bao gồm thông báo sự kiện hệ thống, tích hợp dữ liệu cross-service, xử lý dữ liệu thời gian thực, logging và monitoring phân tán, và IoT. Nhiều công ty công nghệ hàng đầu như Uber, LinkedIn, PayPal, và Netflix đã triển khai thành công các mô hình giao tiếp một đối nhiều trong kiến trúc vi dịch vụ của họ để xử lý khối lượng dữ liệu và sự kiện khổng lồ.

Trong thực tế, một hệ thống vi dịch vụ hiệu quả thường kết hợp cả giao tiếp đồng bộ, bất đồng bộ một đối một, và bất đồng bộ một đối nhiều, sử dụng mỗi loại cho những trường hợp sử dụng phù hợp. Hiểu rõ các mẫu giao tiếp khác nhau và khi nào nên sử dụng chúng là chìa khóa để thiết kế một kiến trúc vi dịch vụ thành công và có khả năng mở rộng.

Với sự phát triển của các công nghệ như serverless computing, edge computing, và 5G, tương lai của giao tiếp bất đồng bộ một đối nhiều trong vi dịch vụ còn nhiều hứa hẹn. Những tiến bộ này sẽ tiếp tục mở rộng khả năng và hiệu quả của giao tiếp một đối nhiều, cho phép xây dựng các hệ thống phân tán ngày càng phức tạp và có khả năng mở rộng.

Với sự phát triển của các công nghệ như serverless computing, edge computing, và 5G, tương lai của giao tiếp bất đồng bộ một đối nhiều trong vi dịch vụ còn nhiều hứa hẹn. Những tiến bộ này sẽ tiếp tục mở rộng khả năng và hiệu quả của giao tiếp một đối nhiều, cho phép xây dựng các hệ thống phân tán ngày càng phức tạp và có khả năng mở rộng.
\section{Tổng kết}
Trong chương này, chúng ta đã tìm hiểu chi tiết về các mẫu giao tiếp trong kiến trúc vi dịch vụ. Đầu tiên, chúng ta đã phân loại các mẫu giao tiếp theo hai tiêu chí chính: communication mode (synchronous/asynchronous) và communication scope (one-to-one/one-to-many).

Với giao tiếp đồng bộ (synchronous) one-to-one, chúng ta tập trung vào mẫu Request/Response, được triển khai thông qua các công nghệ phổ biến như REST, gRPC và GraphQL. Mẫu này có ưu điểm là đơn giản, trực quan và phản hồi tức thì, nhưng hạn chế ở khả năng mở rộng và độ tin cậy.

Trong giao tiếp bất đồng bộ (asynchronous) one-to-one, chúng ta đã xem xét các mẫu như One-way Notifications và Message Queue. Các mẫu này cung cấp decoupling tốt hơn, khả năng chịu lỗi cao và khả năng mở rộng tốt, nhưng lại phức tạp hơn trong triển khai và debug.

Đối với giao tiếp bất đồng bộ one-to-many, chúng ta đã tìm hiểu các mẫu như Publish/Subscribe, Event Sourcing và Message Broker với Exchange Routing. Các mẫu này cung cấp decoupling cao nhất và khả năng mở rộng tốt nhất, đặc biệt phù hợp cho việc phát tán thông tin và xử lý sự kiện.

Mỗi mẫu giao tiếp đều có những use cases phù hợp riêng và việc lựa chọn mẫu giao tiếp phù hợp phụ thuộc vào nhiều yếu tố như yêu cầu về độ trễ, tính nhất quán dữ liệu, khả năng mở rộng và độ phức tạp triển khai.

Trong thực tế, một hệ thống vi dịch vụ hiệu quả thường kết hợp nhiều mẫu giao tiếp khác nhau, sử dụng mỗi mẫu cho những trường hợp phù hợp nhất. Hiểu rõ các mẫu giao tiếp và tiêu chí lựa chọn là nền tảng quan trọng để thiết kế một kiến trúc vi dịch vụ thành công.

Ở chương tiếp theo, chúng ta sẽ tiến hành triển khai thử nghiệm các mẫu giao tiếp đã học trong một hệ thống vi dịch vụ thực tế. Chúng ta sẽ xây dựng một ứng dụng mẫu với nhiều vi dịch vụ khác nhau, triển khai các mẫu giao tiếp khác nhau, và đánh giá hiệu suất cũng như tính phù hợp của từng mẫu trong các tình huống cụ thể.