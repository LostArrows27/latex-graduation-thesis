\documentclass{uetgraduation}

% Document metadata
\studentname{Your Full Name}
\title{Nghiên cứu cơ chế giao tiếp trong kiến trúc microservice}
\documenttype{Khóa luận tốt nghiệp đại học hệ chính quy}
\major{Công nghệ thông tin}
\year{2024}
\supervisor{Your Supervisor's Name}
% \cosupervisor{Your Co-supervisor's Name} % Uncomment if you have a co-supervisor

% English metadata (for CLC students)
\englishtitle{Study of Communication Mechanisms in Microservice Architecture}
\englishmajor{Information Technology}
\englishsupervisor{Your Supervisor's Name in English}
% \englishcosupervisor{Your Co-supervisor's Name in English} % Uncomment if you have a co-supervisor

\begin{document}

% Cover pages
\makecovers

% Abstract
\begin{preamble}{Tóm tắt}
    \textbf{Tóm tắt:} Kiến trúc microservice đã trở thành một xu hướng quan trọng trong phát triển phần mềm hiện đại, cho phép xây dựng các hệ thống phức tạp từ các dịch vụ nhỏ, độc lập. Một trong những thách thức chính trong kiến trúc này là việc quản lý giao tiếp giữa các microservice. Khóa luận này tập trung nghiên cứu các cơ chế giao tiếp trong kiến trúc microservice, bao gồm các mô hình đồng bộ và bất đồng bộ, các giao thức và công nghệ được sử dụng, cũng như các thách thức và giải pháp trong việc triển khai. Nghiên cứu cũng đánh giá hiệu quả của các phương pháp giao tiếp khác nhau thông qua các trường hợp sử dụng thực tế và đề xuất các hướng tiếp cận tối ưu cho các tình huống cụ thể.

    \textit{\textbf{Từ khóa:} Microservice, Kiến trúc phần mềm, Giao tiếp dịch vụ, API, Message Queue.}
\end{preamble}

% Table of contents and lists
\begin{contentlisting}
    \tableofcontents
    \listoffigures
    \listoftables

    \begin{contentlistingsection}{Danh sách từ viết tắt}
        API: Application Programming Interface -- Giao diện lập trình ứng dụng

        RPC: Remote Procedure Call -- Gọi thủ tục từ xa

        MQ: Message Queue -- Hàng đợi tin nhắn

        REST: Representational State Transfer -- Chuyển giao trạng thái biểu diễn

        SOA: Service-Oriented Architecture -- Kiến trúc hướng dịch vụ
    \end{contentlistingsection}
\end{contentlisting}

% Main content
\chapter{Mở đầu}

\section{Bối cảnh và sự cần thiết của đề tài}
Trong những năm gần đây, kiến trúc microservice đã trở thành một xu hướng quan trọng trong phát triển phần mềm hiện đại. Sự phát triển này đặt ra nhiều thách thức trong việc quản lý giao tiếp giữa các dịch vụ. Việc nghiên cứu và hiểu rõ các cơ chế giao tiếp trong kiến trúc microservice là cần thiết để:

\begin{itemize}
    \item Tối ưu hóa hiệu suất của hệ thống
    \item Đảm bảo độ tin cậy và khả năng mở rộng
    \item Giảm thiểu độ phức tạp trong phát triển và bảo trì
    \item Nâng cao khả năng chịu lỗi của hệ thống
\end{itemize}

\section{Mục tiêu nghiên cứu}
Khóa luận này nhằm đạt được các mục tiêu sau:

\begin{itemize}
    \item Phân tích và đánh giá các cơ chế giao tiếp trong kiến trúc microservice
    \item So sánh ưu nhược điểm của các phương pháp giao tiếp khác nhau
    \item Đề xuất các giải pháp tối ưu cho các tình huống cụ thể
    \item Thực nghiệm và đánh giá hiệu quả của các cơ chế giao tiếp
\end{itemize}

\section{Phạm vi nghiên cứu}
\begin{itemize}
    \item Tập trung vào các cơ chế giao tiếp phổ biến trong microservice
    \item Đánh giá trên các tiêu chí: hiệu suất, độ tin cậy, khả năng mở rộng
    \item Thực nghiệm trên các nền tảng và công nghệ phổ biến
\end{itemize}

\section{Phương pháp nghiên cứu}
\begin{itemize}
    \item \textbf{Phương pháp nghiên cứu lý thuyết:}
    \begin{itemize}
        \item Tổng hợp và phân tích tài liệu
        \item So sánh các phương pháp tiếp cận
        \item Đánh giá ưu nhược điểm
    \end{itemize}
    
    \item \textbf{Phương pháp nghiên cứu thực nghiệm:}
    \begin{itemize}
        \item Xây dựng môi trường thử nghiệm
        \item Triển khai các cơ chế giao tiếp
        \item Đo lường và đánh giá kết quả
    \end{itemize}
\end{itemize}

\section{Ý nghĩa khoa học và thực tiễn}
\begin{itemize}
    \item \textbf{Ý nghĩa khoa học:}
    \begin{itemize}
        \item Đóng góp vào việc nghiên cứu và phát triển các phương pháp giao tiếp trong kiến trúc microservice
        \item Cung cấp cơ sở lý thuyết cho việc lựa chọn và triển khai các cơ chế giao tiếp
        \item Đề xuất các hướng nghiên cứu mới trong lĩnh vực này
    \end{itemize}
    
    \item \textbf{Ý nghĩa thực tiễn:}
    \begin{itemize}
        \item Cung cấp hướng dẫn thực tế cho việc triển khai các hệ thống microservice
        \item Giúp các nhà phát triển đưa ra quyết định phù hợp về cơ chế giao tiếp
        \item Tối ưu hóa hiệu suất và độ tin cậy của hệ thống
    \end{itemize}
\end{itemize}

\section{Đối tượng và phạm vi nghiên cứu}
\subsection{Đối tượng nghiên cứu}
\begin{itemize}
    \item Các cơ chế giao tiếp trong kiến trúc microservice
    \item Các công nghệ và giao thức giao tiếp phổ biến
    \item Các mô hình triển khai và quản lý giao tiếp
\end{itemize}

\subsection{Phạm vi nghiên cứu}
\begin{itemize}
    \item Tập trung vào các cơ chế giao tiếp phổ biến trong microservice
    \item Đánh giá trên các tiêu chí: hiệu suất, độ tin cậy, khả năng mở rộng
    \item Thực nghiệm trên các nền tảng và công nghệ phổ biến
\end{itemize}

\section{Cấu trúc khóa luận}
Khóa luận được tổ chức thành 5 chương:

\begin{itemize}
    \item \textbf{Chương 1: Mở đầu} - Giới thiệu tổng quan về đề tài
    \item \textbf{Chương 2: Cơ sở lý thuyết} - Trình bày các khái niệm cơ bản
    \item \textbf{Chương 3: Phân tích các cơ chế giao tiếp} - Chi tiết về các phương pháp giao tiếp
    \item \textbf{Chương 4: Đánh giá và thực nghiệm} - Kết quả thực nghiệm và phân tích
    \item \textbf{Chương 5: Kết luận và hướng phát triển} - Tổng kết và đề xuất
\end{itemize} 
\chapter{Cơ sở lý thuyết}

\section{Tổng quan về Microservice Architecture}
\subsection{Khái niệm và đặc điểm}
Microservice Architecture là một kiến trúc phần mềm trong đó các ứng dụng được phát triển như một tập hợp các dịch vụ nhỏ, độc lập, mỗi dịch vụ chạy trong một quy trình riêng và giao tiếp với nhau thông qua các cơ chế nhẹ, thường là HTTP resource API.

\subsection{Lợi ích và thách thức}
\begin{itemize}
    \item Lợi ích:
    \begin{itemize}
        \item Khả năng mở rộng độc lập
        \item Dễ dàng triển khai và bảo trì
        \item Sử dụng công nghệ đa dạng
        \item Khả năng chịu lỗi cao
    \end{itemize}
    \item Thách thức:
    \begin{itemize}
        \item Quản lý giao tiếp giữa các dịch vụ
        \item Đảm bảo tính nhất quán dữ liệu
        \item Giám sát và debug phức tạp
        \item Quản lý phiên bản
    \end{itemize}
\end{itemize}

\section{Communication trong Microservices}
\subsection{Tầm quan trọng của giao tiếp}
Giao tiếp giữa các microservice là yếu tố quan trọng quyết định hiệu suất và độ tin cậy của toàn bộ hệ thống. Việc lựa chọn cơ chế giao tiếp phù hợp ảnh hưởng trực tiếp đến:
\begin{itemize}
    \item Hiệu suất của hệ thống
    \item Khả năng mở rộng
    \item Độ tin cậy
    \item Tính nhất quán dữ liệu
\end{itemize}

\subsection{Các yếu tố ảnh hưởng đến giao tiếp}
\begin{itemize}
    \item Yêu cầu về độ trễ
    \item Tính nhất quán dữ liệu
    \item Khối lượng giao tiếp
    \item Mô hình giao tiếp (đồng bộ/bất đồng bộ)
    \item Phạm vi giao tiếp (one-to-one/one-to-many)
\end{itemize} 
\chapter{Phân tích các Communication Patterns}

\section{Cách phân loại các pattern}
\subsection{Tiêu chí phân loại theo communication mode}
\begin{itemize}
    \item Synchronous Communication
    \begin{itemize}
        \item REST API
        \item gRPC
        \item GraphQL
    \end{itemize}
    \item Asynchronous Communication
    \begin{itemize}
        \item Message Queue
        \item Event Bus
        \item Pub/Sub
    \end{itemize}
\end{itemize}

\subsection{Tiêu chí phân loại theo communication scope}
\begin{itemize}
    \item One-to-One Communication
    \begin{itemize}
        \item Direct API calls
        \item Point-to-point messaging
    \end{itemize}
    \item One-to-Many Communication
    \begin{itemize}
        \item Event broadcasting
        \item Pub/Sub messaging
    \end{itemize}
\end{itemize}

\subsection{Các yếu tố ảnh hưởng đến việc lựa chọn pattern}
\begin{itemize}
    \item Performance requirements
    \item Data consistency needs
    \item System scalability
    \item Error handling requirements
    \item Development complexity
\end{itemize}

\section{Synchronous Communication Patterns}
\subsection{REST API Pattern}
\begin{itemize}
    \item Request-Response model
    \item HTTP methods (GET, POST, PUT, DELETE)
    \item Stateless communication
\end{itemize}

\begin{itemize}
    \item Order-Inventory check
    \item Payment processing
    \item Simple CRUD operations
\end{itemize}

\begin{itemize}
    \item Ưu điểm:
    \begin{itemize}
        \item Simple implementation
        \item Immediate feedback
        \item Standard protocol
    \end{itemize}
    \item Nhược điểm:
    \begin{itemize}
        \item High latency
        \item Resource blocking
        \item Tight coupling
    \end{itemize}
\end{itemize}

\section{Asynchronous Communication (one-to-one)}
\subsection{Message Queue Pattern}

\begin{itemize}
    \item Producer-Consumer model
    \item Message persistence
    \item Guaranteed delivery
\end{itemize}

\begin{itemize}
    \item Long-running payment processing
    \item Background tasks
    \item Batch processing
\end{itemize}

\begin{itemize}
    \item Ưu điểm:
    \begin{itemize}
        \item Better resource utilization
        \item Loose coupling
        \item Reliable delivery
    \end{itemize}
    \item Nhược điểm:
    \begin{itemize}
        \item Eventual consistency
        \item Complex workflow
        \item Message ordering
    \end{itemize}
\end{itemize}

\section{Asynchronous Communication (one-to-many)}
\subsection{Pub/Sub Pattern}
\begin{itemize}
    \item Publisher-Subscriber model
    \item Topic-based routing
    \item Event-driven architecture
\end{itemize}

\begin{itemize}
    \item Order notifications
    \item User activity logging
    \item Real-time updates
\end{itemize}

\begin{itemize}
    \item Ưu điểm:
    \begin{itemize}
        \item High scalability
        \item Decoupled services
        \item Efficient broadcasting
    \end{itemize}
    \item Nhược điểm:
    \begin{itemize}
        \item Message ordering
        \item Delivery guarantees
        \item Complex setup
    \end{itemize}
\end{itemize}

\section{So sánh và đánh giá các patterns}
\subsection{Performance comparison}
\begin{itemize}
    \item Latency metrics
    \item Throughput capabilities
    \item Resource utilization
\end{itemize}

\subsection{Error handling capabilities}
\begin{itemize}
    \item Retry mechanisms
    \item Error propagation
    \item Recovery strategies
\end{itemize}

\subsection{Scalability considerations}
\begin{itemize}
    \item Horizontal scaling
    \item Load balancing
    \item Service discovery
\end{itemize} 
\chapter{Triển khai thử nghiệm}

\section{Mô tả bài toán và yêu cầu}
\subsection{Hệ thống thử nghiệm}
\begin{itemize}
    \item E-commerce order processing system
    \item Microservices architecture
    \item Multiple communication patterns
\end{itemize}

\subsection{Yêu cầu hệ thống}
\begin{itemize}
    \item Order-Inventory management
    \item Payment processing
    \item Order notifications
    \item User activity logging
\end{itemize}

\section{Cài đặt và triển khai}
\subsection{Thiết kế kiến trúc}
\begin{itemize}
    \item Service boundaries
    \item Communication patterns
    \item Data flow
    \item Error handling
\end{itemize}

\subsection{Lựa chọn công nghệ}
\begin{itemize}
    \item Spring Boot for services
    \item RabbitMQ for message queue
    \item Kafka for pub/sub
    \item Docker for containerization
\end{itemize}

\subsection{Chi tiết triển khai}
\begin{itemize}
    \item REST API implementation
    \item Message Queue implementation
    \item Pub/Sub implementation
    \item Activity tracking system
\end{itemize}

\section{Kết quả triển khai}
\subsection{Hiệu suất hệ thống}
\begin{itemize}
    \item Latency metrics
    \item Throughput results
    \item Resource utilization
    \item Error rates
\end{itemize}

\subsection{Độ tin cậy}
\begin{itemize}
    \item Processing times
    \item Success rates
    \item Error handling
    \item Recovery times
\end{itemize}

\subsection{Khả năng mở rộng}
\begin{itemize}
    \item Broadcast performance
    \item Service failure impact
    \item System stability
    \item Resource efficiency
\end{itemize}

\subsection{Thiết lập hạ tầng}
\begin{itemize}
    \item Docker containers
    \item Service discovery
    \item Message brokers
    \item Monitoring tools
\end{itemize}

\section{Đánh giá hiệu năng}
\subsection{Phương pháp đánh giá}
\begin{itemize}
    \item Test scenarios
    \item Performance metrics
    \item Testing tools
    \item Data collection
\end{itemize}

\subsection{Phân tích so sánh}
\begin{itemize}
    \item Synchronous vs Asynchronous
    \item One-to-One vs One-to-Many
    \item Resource utilization
    \item Error handling
\end{itemize} 

% References
\begin{thebibliography}{99}
    \begin{bibsection}{Tài liệu tham khảo}
        \bibitem{ref1} Tài liệu tham khảo 1
        \bibitem{ref2} Tài liệu tham khảo 2
    \end{bibsection}
\end{thebibliography}

\end{document}