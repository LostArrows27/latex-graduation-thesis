\documentclass{uetgraduation}
\usepackage{multirow}

% Document metadata
\studentname{Nguyễn Hải Đan}
\title{Đánh giá các cơ chế giao tiếp trong kiến trúc microservice}
\documenttype{Khóa luận tốt nghiệp đại học hệ chính quy}
\major{Công nghệ thông tin}
\year{2025}
\supervisor{Your Supervisor's Name}
% \cosupervisor{Your Co-supervisor's Name} % Uncomment if you have a co-supervisor

% English metadata (for CLC students)
\englishtitle{Study of Communication Mechanisms in Microservice Architecture}
\englishmajor{Information Technology}
\englishsupervisor{Your Supervisor's Name in English}
% \englishcosupervisor{Your Co-supervisor's Name in English} % Uncomment if you have a co-supervisor

\begin{document}

% Cover pages
\makecovers

% Abstract
\begin{preamble}{Tóm tắt}
    \textbf{Tóm tắt:} Kiến trúc microservice đã trở thành một xu hướng quan trọng trong phát triển phần mềm hiện đại, cho phép xây dựng các hệ thống phức tạp từ các dịch vụ nhỏ, độc lập. Một trong những thách thức chính trong kiến trúc này là việc quản lý giao tiếp giữa các microservice. Khóa luận này tập trung nghiên cứu các cơ chế giao tiếp trong kiến trúc microservice, bao gồm các mô hình đồng bộ và bất đồng bộ, các giao thức và công nghệ được sử dụng, cũng như các thách thức và giải pháp trong việc triển khai. Nghiên cứu cũng đánh giá hiệu quả của các phương pháp giao tiếp khác nhau thông qua các trường hợp sử dụng thực tế và đề xuất các hướng tiếp cận tối ưu cho các tình huống cụ thể.

    \textit{\textbf{Từ khóa:} Microservice, Kiến trúc phần mềm, Giao tiếp dịch vụ, API, Message Queue.}
\end{preamble}

\begin{preamble}{Abstract}
    \textbf{Abstract:} Microservice architecture has become a significant trend in modern software development, enabling the construction of complex systems from small, independent services. One of the main challenges in this architecture is managing communication between microservices. This thesis focuses on studying communication mechanisms in microservice architecture, including synchronous and asynchronous models, protocols and technologies used, as well as challenges and solutions in implementation. The study also evaluates the effectiveness of different communication methods through real-world use cases and proposes optimal approaches for specific scenarios.

    \textit{\textbf{Keywords:} Microservice, Software Architecture, Service Communication, API, Message Queue.}
\end{preamble}

% Table of contents and lists
\begin{contentlisting}
    \tableofcontents
    \listoffigures
    \listoftables

    \begin{contentlistingsection}{Danh sách từ viết tắt}
        API: Application Programming Interface -- Giao diện lập trình ứng dụng

        RPC: Remote Procedure Call -- Gọi thủ tục từ xa

        MQ: Message Queue -- Hàng đợi tin nhắn

        REST: Representational State Transfer -- Chuyển giao trạng thái biểu diễn

        SOA: Service-Oriented Architecture -- Kiến trúc hướng dịch vụ
    \end{contentlistingsection}
\end{contentlisting}

% Declaration page
\newpage
\thispagestyle{empty}
\begin{center}
    \textbf{LỜI CAM ĐOAN}
\end{center}

Tôi xin cam đoan đây là công trình nghiên cứu của riêng tôi. Các số liệu, kết quả nêu trong khóa luận là trung thực và chưa từng được ai công bố trong bất kỳ công trình nào khác. Các tham khảo, trích dẫn trong khóa luận đều được chỉ rõ nguồn gốc. Nếu sai tôi xin hoàn toàn chịu trách nhiệm.

\begin{flushright}
    Hà Nội, ngày \ldots\ldots tháng \ldots\ldots năm 2025

    \vspace{2cm}
    \textit{Người cam đoan}

    \vspace{1cm}
    \textit{Nguyễn Hải Đan}
\end{flushright}

% Main content
\chapter{Mở đầu}

\section{Bối cảnh và sự cần thiết của đề tài}
Trong những năm gần đây, kiến trúc microservice đã trở thành một xu hướng quan trọng trong phát triển phần mềm hiện đại. Sự phát triển này đặt ra nhiều thách thức trong việc quản lý giao tiếp giữa các dịch vụ. Việc nghiên cứu và hiểu rõ các cơ chế giao tiếp trong kiến trúc microservice là cần thiết để:

\begin{itemize}
    \item Tối ưu hóa hiệu suất của hệ thống
    \item Đảm bảo độ tin cậy và khả năng mở rộng
    \item Giảm thiểu độ phức tạp trong phát triển và bảo trì
    \item Nâng cao khả năng chịu lỗi của hệ thống
\end{itemize}

\section{Mục tiêu nghiên cứu}
Khóa luận này nhằm đạt được các mục tiêu sau:

\begin{itemize}
    \item Phân tích và đánh giá các cơ chế giao tiếp trong kiến trúc microservice
    \item So sánh ưu nhược điểm của các phương pháp giao tiếp khác nhau
    \item Đề xuất các giải pháp tối ưu cho các tình huống cụ thể
    \item Thực nghiệm và đánh giá hiệu quả của các cơ chế giao tiếp
\end{itemize}

\section{Phạm vi nghiên cứu}
\begin{itemize}
    \item Tập trung vào các cơ chế giao tiếp phổ biến trong microservice
    \item Đánh giá trên các tiêu chí: hiệu suất, độ tin cậy, khả năng mở rộng
    \item Thực nghiệm trên các nền tảng và công nghệ phổ biến
\end{itemize}

\section{Phương pháp nghiên cứu}
\begin{itemize}
    \item \textbf{Phương pháp nghiên cứu lý thuyết:}
    \begin{itemize}
        \item Tổng hợp và phân tích tài liệu
        \item So sánh các phương pháp tiếp cận
        \item Đánh giá ưu nhược điểm
    \end{itemize}
    
    \item \textbf{Phương pháp nghiên cứu thực nghiệm:}
    \begin{itemize}
        \item Xây dựng môi trường thử nghiệm
        \item Triển khai các cơ chế giao tiếp
        \item Đo lường và đánh giá kết quả
    \end{itemize}
\end{itemize}

\section{Ý nghĩa khoa học và thực tiễn}
\begin{itemize}
    \item \textbf{Ý nghĩa khoa học:}
    \begin{itemize}
        \item Đóng góp vào việc nghiên cứu và phát triển các phương pháp giao tiếp trong kiến trúc microservice
        \item Cung cấp cơ sở lý thuyết cho việc lựa chọn và triển khai các cơ chế giao tiếp
        \item Đề xuất các hướng nghiên cứu mới trong lĩnh vực này
    \end{itemize}
    
    \item \textbf{Ý nghĩa thực tiễn:}
    \begin{itemize}
        \item Cung cấp hướng dẫn thực tế cho việc triển khai các hệ thống microservice
        \item Giúp các nhà phát triển đưa ra quyết định phù hợp về cơ chế giao tiếp
        \item Tối ưu hóa hiệu suất và độ tin cậy của hệ thống
    \end{itemize}
\end{itemize}

\section{Đối tượng và phạm vi nghiên cứu}
\subsection{Đối tượng nghiên cứu}
\begin{itemize}
    \item Các cơ chế giao tiếp trong kiến trúc microservice
    \item Các công nghệ và giao thức giao tiếp phổ biến
    \item Các mô hình triển khai và quản lý giao tiếp
\end{itemize}

\subsection{Phạm vi nghiên cứu}
\begin{itemize}
    \item Tập trung vào các cơ chế giao tiếp phổ biến trong microservice
    \item Đánh giá trên các tiêu chí: hiệu suất, độ tin cậy, khả năng mở rộng
    \item Thực nghiệm trên các nền tảng và công nghệ phổ biến
\end{itemize}

\section{Cấu trúc khóa luận}
Khóa luận được tổ chức thành 5 chương:

\begin{itemize}
    \item \textbf{Chương 1: Mở đầu} - Giới thiệu tổng quan về đề tài
    \item \textbf{Chương 2: Cơ sở lý thuyết} - Trình bày các khái niệm cơ bản
    \item \textbf{Chương 3: Phân tích các cơ chế giao tiếp} - Chi tiết về các phương pháp giao tiếp
    \item \textbf{Chương 4: Đánh giá và thực nghiệm} - Kết quả thực nghiệm và phân tích
    \item \textbf{Chương 5: Kết luận và hướng phát triển} - Tổng kết và đề xuất
\end{itemize} 
\chapter{Cơ sở lý thuyết}

\section{Tổng quan về Microservice Architecture}
\subsection{Khái niệm và đặc điểm}
Microservice Architecture là một kiến trúc phần mềm trong đó các ứng dụng được phát triển như một tập hợp các dịch vụ nhỏ, độc lập, mỗi dịch vụ chạy trong một quy trình riêng và giao tiếp với nhau thông qua các cơ chế nhẹ, thường là HTTP resource API.

\subsection{Lợi ích và thách thức}
\begin{itemize}
    \item Lợi ích:
    \begin{itemize}
        \item Khả năng mở rộng độc lập
        \item Dễ dàng triển khai và bảo trì
        \item Sử dụng công nghệ đa dạng
        \item Khả năng chịu lỗi cao
    \end{itemize}
    \item Thách thức:
    \begin{itemize}
        \item Quản lý giao tiếp giữa các dịch vụ
        \item Đảm bảo tính nhất quán dữ liệu
        \item Giám sát và debug phức tạp
        \item Quản lý phiên bản
    \end{itemize}
\end{itemize}

\section{Communication trong Microservices}
\subsection{Tầm quan trọng của giao tiếp}
Giao tiếp giữa các microservice là yếu tố quan trọng quyết định hiệu suất và độ tin cậy của toàn bộ hệ thống. Việc lựa chọn cơ chế giao tiếp phù hợp ảnh hưởng trực tiếp đến:
\begin{itemize}
    \item Hiệu suất của hệ thống
    \item Khả năng mở rộng
    \item Độ tin cậy
    \item Tính nhất quán dữ liệu
\end{itemize}

\subsection{Các yếu tố ảnh hưởng đến giao tiếp}
\begin{itemize}
    \item Yêu cầu về độ trễ
    \item Tính nhất quán dữ liệu
    \item Khối lượng giao tiếp
    \item Mô hình giao tiếp (đồng bộ/bất đồng bộ)
    \item Phạm vi giao tiếp (one-to-one/one-to-many)
\end{itemize} 
\chapter{Phân tích các Communication Patterns}

\section{Cách phân loại các pattern}
\subsection{Tiêu chí phân loại theo communication mode}
\begin{itemize}
    \item Synchronous Communication
    \begin{itemize}
        \item REST API
        \item gRPC
        \item GraphQL
    \end{itemize}
    \item Asynchronous Communication
    \begin{itemize}
        \item Message Queue
        \item Event Bus
        \item Pub/Sub
    \end{itemize}
\end{itemize}

\subsection{Tiêu chí phân loại theo communication scope}
\begin{itemize}
    \item One-to-One Communication
    \begin{itemize}
        \item Direct API calls
        \item Point-to-point messaging
    \end{itemize}
    \item One-to-Many Communication
    \begin{itemize}
        \item Event broadcasting
        \item Pub/Sub messaging
    \end{itemize}
\end{itemize}

\subsection{Các yếu tố ảnh hưởng đến việc lựa chọn pattern}
\begin{itemize}
    \item Performance requirements
    \item Data consistency needs
    \item System scalability
    \item Error handling requirements
    \item Development complexity
\end{itemize}

\section{Synchronous Communication Patterns}
\subsection{REST API Pattern}
\begin{itemize}
    \item Request-Response model
    \item HTTP methods (GET, POST, PUT, DELETE)
    \item Stateless communication
\end{itemize}

\begin{itemize}
    \item Order-Inventory check
    \item Payment processing
    \item Simple CRUD operations
\end{itemize}

\begin{itemize}
    \item Ưu điểm:
    \begin{itemize}
        \item Simple implementation
        \item Immediate feedback
        \item Standard protocol
    \end{itemize}
    \item Nhược điểm:
    \begin{itemize}
        \item High latency
        \item Resource blocking
        \item Tight coupling
    \end{itemize}
\end{itemize}

\section{Asynchronous Communication (one-to-one)}
\subsection{Message Queue Pattern}

\begin{itemize}
    \item Producer-Consumer model
    \item Message persistence
    \item Guaranteed delivery
\end{itemize}

\begin{itemize}
    \item Long-running payment processing
    \item Background tasks
    \item Batch processing
\end{itemize}

\begin{itemize}
    \item Ưu điểm:
    \begin{itemize}
        \item Better resource utilization
        \item Loose coupling
        \item Reliable delivery
    \end{itemize}
    \item Nhược điểm:
    \begin{itemize}
        \item Eventual consistency
        \item Complex workflow
        \item Message ordering
    \end{itemize}
\end{itemize}

\section{Asynchronous Communication (one-to-many)}
\subsection{Pub/Sub Pattern}
\begin{itemize}
    \item Publisher-Subscriber model
    \item Topic-based routing
    \item Event-driven architecture
\end{itemize}

\begin{itemize}
    \item Order notifications
    \item User activity logging
    \item Real-time updates
\end{itemize}

\begin{itemize}
    \item Ưu điểm:
    \begin{itemize}
        \item High scalability
        \item Decoupled services
        \item Efficient broadcasting
    \end{itemize}
    \item Nhược điểm:
    \begin{itemize}
        \item Message ordering
        \item Delivery guarantees
        \item Complex setup
    \end{itemize}
\end{itemize}

\section{So sánh và đánh giá các patterns}
\subsection{Performance comparison}
\begin{itemize}
    \item Latency metrics
    \item Throughput capabilities
    \item Resource utilization
\end{itemize}

\subsection{Error handling capabilities}
\begin{itemize}
    \item Retry mechanisms
    \item Error propagation
    \item Recovery strategies
\end{itemize}

\subsection{Scalability considerations}
\begin{itemize}
    \item Horizontal scaling
    \item Load balancing
    \item Service discovery
\end{itemize} 
\chapter{Triển khai thử nghiệm}

\section{Mô tả bài toán và yêu cầu}
\subsection{Hệ thống thử nghiệm}
\begin{itemize}
    \item E-commerce order processing system
    \item Microservices architecture
    \item Multiple communication patterns
\end{itemize}

\subsection{Yêu cầu hệ thống}
\begin{itemize}
    \item Order-Inventory management
    \item Payment processing
    \item Order notifications
    \item User activity logging
\end{itemize}

\section{Cài đặt và triển khai}
\subsection{Thiết kế kiến trúc}
\begin{itemize}
    \item Service boundaries
    \item Communication patterns
    \item Data flow
    \item Error handling
\end{itemize}

\subsection{Lựa chọn công nghệ}
\begin{itemize}
    \item Spring Boot for services
    \item RabbitMQ for message queue
    \item Kafka for pub/sub
    \item Docker for containerization
\end{itemize}

\subsection{Chi tiết triển khai}
\begin{itemize}
    \item REST API implementation
    \item Message Queue implementation
    \item Pub/Sub implementation
    \item Activity tracking system
\end{itemize}

\section{Kết quả triển khai}
\subsection{Hiệu suất hệ thống}
\begin{itemize}
    \item Latency metrics
    \item Throughput results
    \item Resource utilization
    \item Error rates
\end{itemize}

\subsection{Độ tin cậy}
\begin{itemize}
    \item Processing times
    \item Success rates
    \item Error handling
    \item Recovery times
\end{itemize}

\subsection{Khả năng mở rộng}
\begin{itemize}
    \item Broadcast performance
    \item Service failure impact
    \item System stability
    \item Resource efficiency
\end{itemize}

\subsection{Thiết lập hạ tầng}
\begin{itemize}
    \item Docker containers
    \item Service discovery
    \item Message brokers
    \item Monitoring tools
\end{itemize}

\section{Đánh giá hiệu năng}
\subsection{Phương pháp đánh giá}
\begin{itemize}
    \item Test scenarios
    \item Performance metrics
    \item Testing tools
    \item Data collection
\end{itemize}

\subsection{Phân tích so sánh}
\begin{itemize}
    \item Synchronous vs Asynchronous
    \item One-to-One vs One-to-Many
    \item Resource utilization
    \item Error handling
\end{itemize} 
\chapter{Đánh giá và thảo luận}

\section{Phương pháp và tiêu chí đánh giá}

\subsection{Tổng quan phương pháp đánh giá}
Phương pháp đánh giá trong dự án này được thiết kế để cung cấp một cái nhìn toàn diện về hiệu suất và độ tin cậy của các mẫu giao tiếp khác nhau trong kiến trúc microservice. Quá trình đánh giá tuân theo phương pháp luận khoa học nghiêm ngặt, bao gồm việc thiết lập các tiêu chí đánh giá rõ ràng, triển khai các kịch bản kiểm thử thực tế, và sử dụng các công cụ đo lường hiệu suất chuyên nghiệp.

Dự án xác định bốn kịch bản nghiệp vụ chính trong hệ thống thương mại điện tử: kiểm tra và cập nhật tồn kho, xử lý thanh toán, thông báo kết quả đơn hàng, và ghi nhận hoạt động người dùng. Mỗi kịch bản này đại diện cho một loại tương tác phổ biến trong các hệ thống microservice, với các yêu cầu về hiệu suất và độ tin cậy khác nhau. Việc chọn các kịch bản đa dạng giúp đảm bảo kết quả đánh giá có tính đại diện cao và áp dụng được cho nhiều ngữ cảnh khác nhau.

Dự án triển khai ba mẫu giao tiếp chính: giao tiếp đồng bộ sử dụng REST API, giao tiếp bất đồng bộ dạng một-một sử dụng RabbitMQ, và giao tiếp bất đồng bộ dạng một-nhiều sử dụng Kafka. Mỗi mẫu giao tiếp được triển khai trên cùng một nền tảng hạ tầng và được đánh giá trong cùng các kịch bản, đảm bảo tính công bằng và khách quan của quá trình so sánh.

\subsection{Tiêu chí đánh giá cụ thể}
Các tiêu chí đánh giá được xây dựng dựa trên các yêu cầu phi chức năng quan trọng của hệ thống microservice, phản ánh các khía cạnh chất lượng mà các kiến trúc microservice hiện đại cần đáp ứng. Các tiêu chí này được phân thành bốn nhóm chính: hiệu suất, tài nguyên hệ thống, độ tin cậy và khả năng chịu lỗi.

Về hiệu suất, dự án đánh giá thời gian phản hồi (thời gian từ khi gửi yêu cầu đến khi nhận được phản hồi ban đầu), thời gian xử lý end-to-end (tổng thời gian để hoàn thành toàn bộ quy trình), thông lượng (số lượng yêu cầu xử lý trong một đơn vị thời gian), và thời gian phản hồi phân vị thứ 95 (P95).

Về tài nguyên hệ thống, dự án đánh giá mức sử dụng CPU, bộ nhớ và lưu lượng mạng. Các chỉ số này phản ánh hiệu quả sử dụng tài nguyên của mỗi mẫu giao tiếp, yếu tố quan trọng ảnh hưởng đến chi phí vận hành hệ thống.

Về độ tin cậy, dự án đánh giá tỷ lệ thành công, tỷ lệ lỗi và tỷ lệ nhất quán dữ liệu. Đặc biệt, tỷ lệ nhất quán dữ liệu đo lường khả năng duy trì tính nhất quán giữa các service, yếu tố quan trọng trong các hệ thống phân tán.

Về khả năng chịu lỗi, dự án đánh giá tỷ lệ lan truyền lỗi, thời gian phục hồi và tỷ lệ thành công một phần. Các chỉ số này phản ánh khả năng của hệ thống trong việc duy trì hoạt động khi gặp sự cố.

\subsection{Công cụ và môi trường đánh giá}
Dự án sử dụng một bộ công cụ hiện đại để thực hiện đánh giá. Công cụ k6 được sử dụng để mô phỏng lưu lượng người dùng và đo lường các chỉ số hiệu suất. Prometheus được sử dụng để thu thập và lưu trữ các số liệu về hiệu suất hệ thống. Các công cụ này đều là mã nguồn mở và được sử dụng rộng rãi trong cộng đồng phát triển phần mềm, đảm bảo tính tin cậy và khả năng mở rộng của quá trình đánh giá.

Môi trường kiểm thử được thiết kế để mô phỏng các điều kiện thực tế mà một hệ thống thương mại điện tử có thể gặp phải. Các kịch bản kiểm thử bao gồm các mức tải khác nhau, từ tải nhẹ đến tải nặng, để đánh giá hiệu suất của các mẫu giao tiếp trong nhiều điều kiện khác nhau. Các service được triển khai trên cùng một cấu hình phần cứng, và các bài kiểm thử được thực hiện nhiều lần để đảm bảo tính đại diện thống kê của kết quả.

Kịch bản kiểm thử được thiết kế để tự động thực hiện các tác vụ như tạo đơn hàng, kiểm tra tồn kho, xử lý thanh toán và gửi thông báo. Mỗi kịch bản được thực hiện trên những mẫu giao tiếp thích hợp với kịch bản đấy, và các chỉ số hiệu suất được ghi lại để so sánh.

Việc sử dụng các công cụ và môi trường đánh giá hiện đại giúp đảm bảo tính khách quan và chính xác của kết quả đánh giá. Điều này cho phép dự án đưa ra những kết luận đáng tin cậy về hiệu suất và độ tin cậy của các mẫu giao tiếp trong kiến trúc microservice.

\section{Kết quả đánh giá}

\subsection{Mẫu giao tiếp trong kịch bản Order-Inventory}
Kết quả đánh giá hiệu suất giữa giao tiếp đồng bộ và bất đồng bộ trong kịch bản kiểm tra và cập nhật tồn kho cho thấy những đặc điểm đáng chú ý về cách các mẫu giao tiếp ảnh hưởng đến trải nghiệm người dùng và hiệu quả hệ thống. Giao tiếp bất đồng bộ thể hiện thời gian phản hồi ban đầu nhanh hơn đáng kể, giúp người dùng nhận được phản hồi tức thì khi thực hiện thao tác. Tuy nhiên, giao tiếp đồng bộ lại cho thời gian xử lý end-to-end nhỉnh hơn, phản ánh đặc tính xử lý trực tiếp, không qua trung gian của phương pháp này.

Về tính nhất quán dữ liệu, một phát hiện thú vị là giao tiếp bất đồng bộ đạt tỷ lệ nhất quán cao hơn, đặc biệt khi hệ thống chịu tải nặng. Điều này có vẻ trái ngược với quan niệm truyền thống rằng giao tiếp đồng bộ đảm bảo nhất quán dữ liệu tốt hơn. Nguyên nhân có thể do cơ chế hàng đợi giúp điều tiết luồng xử lý và tránh tình trạng quá tải, dẫn đến ít lỗi và nhất quán dữ liệu cao hơn.

Về hiệu quả sử dụng tài nguyên, giao tiếp bất đồng bộ thể hiện rõ ràng lợi thế với mức tiêu thụ CPU và bộ nhớ thấp hơn đáng kể. Điều này đặc biệt quan trọng trong môi trường đám mây, nơi chi phí vận hành thường tỷ lệ thuận với tài nguyên tiêu thụ.

\subsection{Mẫu giao tiếp trong kịch bản Order-Payment}
Trong kịch bản xử lý thanh toán, sự khác biệt về hiệu suất giữa giao tiếp đồng bộ và bất đồng bộ trở nên rõ rệt hơn. Giao tiếp bất đồng bộ thể hiện thời gian phản hồi ban đầu gần như tức thì, cải thiện đáng kể trải nghiệm người dùng so với phương pháp đồng bộ. Điều này đặc biệt quan trọng trong bối cảnh thanh toán, nơi người dùng mong đợi phản hồi nhanh chóng để biết yêu cầu của họ đã được tiếp nhận.

Thông lượng của giao tiếp bất đồng bộ vượt trội hơn hẳn so với giao tiếp đồng bộ, phản ánh khả năng tiếp nhận nhiều yêu cầu hơn trong cùng một khoảng thời gian. Điều này là do giao tiếp bất đồng bộ không bị chặn bởi thời gian xử lý của mỗi yêu cầu, cho phép hệ thống tiếp tục tiếp nhận yêu cầu mới trong khi các yêu cầu cũ đang được xử lý.

Đối với các trường hợp thanh toán kéo dài, giao tiếp bất đồng bộ vẫn duy trì thời gian phản hồi ban đầu thấp, trong khi giao tiếp đồng bộ buộc người dùng phải đợi đến khi toàn bộ quá trình hoàn tất. Điều này làm cho giao tiếp bất đồng bộ trở thành lựa chọn ưu việt hơn cho các tác vụ có thời gian xử lý dài như thanh toán.

\subsection{Mẫu giao tiếp trong kịch bản Order-Notification}
Kết quả đánh giá cho kịch bản thông báo kết quả đơn hàng thể hiện rõ ràng ưu điểm của mô hình Pub/Sub so với phương pháp gọi đồng bộ tuần tự. Mô hình Pub/Sub cung cấp thời gian broadcast và thời gian xử lý mỗi service nhanh hơn đáng kể, giúp thông báo được gửi đến tất cả các kênh nhanh chóng và hiệu quả.

Về khả năng chịu lỗi, mô hình Pub/Sub thể hiện thời gian phục hồi cực kỳ nhanh và tỷ lệ phục hồi thành công cao hơn so với phương pháp gọi đồng bộ tuần tự. Điều này phản ánh kiến trúc lỏng lẻo (loosely coupled) của mô hình Pub/Sub, nơi các subscriber hoạt động độc lập với nhau và với publisher, giúp hệ thống duy trì hoạt động ngay cả khi một số thành phần gặp sự cố.

Hiệu quả sử dụng tài nguyên của mô hình Pub/Sub cũng vượt trội hơn, với mức tiêu thụ CPU thấp hơn đáng kể. Điều này làm cho mô hình Pub/Sub trở thành lựa chọn tiết kiệm chi phí và hiệu quả hơn cho các kịch bản thông báo đa kênh.

\subsection{Mẫu giao tiếp trong kịch bản User Activity Logging}
Đối với kịch bản ghi nhận hoạt động người dùng, cả Kafka (mô hình một-nhiều) và RabbitMQ (mô hình một-một) đều thể hiện hiệu suất tương đương về thời gian phân phối và thông lượng. Tuy nhiên, Kafka sử dụng ít tài nguyên hơn và có thể xử lý nhiều consumer hơn một cách hiệu quả, làm cho nó phù hợp hơn cho các kịch bản có nhiều service cần truy cập cùng một dữ liệu.

Kafka cũng cung cấp khả năng lưu trữ dữ liệu dài hạn và phát lại các sự kiện, điều này đặc biệt hữu ích cho các kịch bản phân tích dữ liệu và học máy, nơi dữ liệu lịch sử có giá trị cao. RabbitMQ, mặt khác, phù hợp hơn cho các kịch bản yêu cầu điều hướng thông điệp phức tạp và định tuyến có điều kiện.

\subsection{Ảnh hưởng của tải hệ thống đến các mẫu giao tiếp}
Một khía cạnh quan trọng của dự án là đánh giá cách các mẫu giao tiếp hoạt động dưới các mức tải khác nhau. Kết quả cho thấy khi tải tăng, hiệu suất của cả giao tiếp đồng bộ và bất đồng bộ đều giảm, nhưng mức độ giảm khác nhau đáng kể.

Giao tiếp đồng bộ thể hiện sự suy giảm hiệu suất mạnh hơn khi tải tăng, với tỷ lệ nhất quán dữ liệu và thông lượng giảm nhanh. Điều này là do mô hình chặn (blocking model) của giao tiếp đồng bộ, nơi mỗi yêu cầu phải đợi đến khi yêu cầu trước đó hoàn thành. Khi số lượng yêu cầu tăng lên, điều này dẫn đến tình trạng nghẽn cổ chai và suy giảm hiệu suất.

Giao tiếp bất đồng bộ, mặt khác, duy trì hiệu suất ổn định hơn dưới tải nặng, nhờ vào khả năng đệm các yêu cầu và xử lý chúng theo tốc độ phù hợp với tài nguyên có sẵn. Đặc biệt, thời gian phản hồi ban đầu của giao tiếp bất đồng bộ vẫn duy trì ở mức thấp ngay cả khi tải tăng cao, giúp duy trì trải nghiệm người dùng tốt.

Mô hình Pub/Sub cũng thể hiện khả năng mở rộng tốt, với hiệu suất ổn định khi số lượng subscriber tăng lên. Điều này làm cho mô hình Pub/Sub trở thành lựa chọn phù hợp cho các hệ thống cần khả năng mở rộng theo chiều ngang.

\section{Thảo luận}

\subsection{Lựa chọn mẫu giao tiếp phù hợp cho từng kịch bản}
Dựa trên kết quả đánh giá toàn diện, có thể đưa ra những khuyến nghị về việc lựa chọn mẫu giao tiếp phù hợp cho từng kịch bản nghiệp vụ trong kiến trúc vi dịch vụ.

Đối với kịch bản kiểm tra và cập nhật tồn kho, giao tiếp bất đồng bộ sử dụng Message Queue (RabbitMQ) được khuyến nghị vì thời gian phản hồi ban đầu nhanh, tỷ lệ nhất quán dữ liệu cao và hiệu quả sử dụng tài nguyên tốt. Mặc dù có thời gian xử lý end-to-end dài hơn một chút, nhưng ưu điểm này được bù đắp bởi khả năng duy trì hiệu suất ổn định dưới tải cao và trải nghiệm người dùng tốt hơn. Độ trễ dữ liệu nhỏ (12-15ms) không đáng kể trong hầu hết các trường hợp sử dụng.

Đối với kịch bản xử lý thanh toán, giao tiếp bất đồng bộ cũng là lựa chọn ưu việt với thời gian phản hồi ban đầu nhanh và thông lượng cao. Trong bối cảnh thanh toán, việc phản hồi nhanh cho người dùng rằng yêu cầu đã được tiếp nhận là rất quan trọng, ngay cả khi quá trình xử lý thực sự có thể kéo dài. Giao tiếp bất đồng bộ cho phép xác nhận ngay lập tức, trong khi vẫn đảm bảo việc xử lý được hoàn thành đúng cách ở phía sau. Đặc biệt với các giao dịch thanh toán thời gian dài, mô hình này giúp giữ trải nghiệm người dùng mượt mà và phản hồi nhanh.

Đối với kịch bản thông báo kết quả đơn hàng, mô hình Pub/Sub sử dụng Kafka là lựa chọn vượt trội với thời gian broadcast nhanh, khả năng phục hồi tốt khi có service gặp lỗi, và hiệu quả sử dụng tài nguyên cao. Mô hình này đặc biệt phù hợp khi một sự kiện cần được xử lý bởi nhiều service độc lập, như gửi email, thông báo đẩy và cập nhật phân tích. Khả năng mở rộng dễ dàng bằng cách thêm các subscriber mới mà không ảnh hưởng đến publisher là một lợi thế lớn của mô hình này.

Đối với kịch bản ghi nhận hoạt động người dùng, Kafka được khuyến nghị nhẹ so với RabbitMQ vì hiệu quả sử dụng tài nguyên tốt hơn và mô hình một-nhiều phù hợp hơn cho việc phân phối dữ liệu hoạt động đến nhiều service phân tích khác nhau. Khả năng lưu trữ dữ liệu lâu dài và phát lại các sự kiện cũng là những ưu điểm quan trọng trong bối cảnh phân tích dữ liệu.

\subsection{Mô hình tích hợp các mẫu giao tiếp}
Trong thực tế, việc kết hợp các mẫu giao tiếp khác nhau thường mang lại kết quả tốt nhất cho hệ thống vi dịch vụ. Mỗi mẫu giao tiếp có những ưu điểm và nhược điểm riêng, phù hợp với những tình huống cụ thể. Một mô hình tích hợp hiệu quả có thể tận dụng ưu điểm của từng phương pháp để tối ưu hóa hiệu suất tổng thể của hệ thống.

Một chiến lược tích hợp hiệu quả là sử dụng giao tiếp đồng bộ cho các tác vụ yêu cầu phản hồi tức thời và có thời gian xử lý ngắn, như truy vấn thông tin đơn giản. Giao tiếp bất đồng bộ dạng một-một được sử dụng cho các tác vụ có thời gian xử lý dài nhưng cần phản hồi nhanh cho người dùng, như xử lý đơn hàng và thanh toán. Giao tiếp bất đồng bộ dạng một-nhiều được sử dụng cho các tác vụ cần phân phối thông tin đến nhiều service, như thông báo sự kiện và ghi nhận hoạt động.

Trong một hệ thống thương mại điện tử, mô hình tích hợp này có thể được triển khai như sau: REST API được sử dụng cho việc hiển thị thông tin sản phẩm và danh mục, RabbitMQ được sử dụng cho xử lý đơn hàng và thanh toán, và Kafka được sử dụng cho thông báo kết quả đơn hàng và phân tích dữ liệu.

Một yếu tố quan trọng cần xem xét khi thiết kế mô hình tích hợp là độ phức tạp của hệ thống. Việc sử dụng nhiều mẫu giao tiếp khác nhau có thể làm tăng độ phức tạp trong triển khai và bảo trì. Do đó, cần cân nhắc giữa lợi ích hiệu suất và chi phí phức tạp khi quyết định số lượng mẫu giao tiếp cần sử dụng.

\subsection{Tối ưu hóa hiệu suất trong thực tế}
Ngoài việc lựa chọn mẫu giao tiếp phù hợp, còn có nhiều chiến lược tối ưu hóa hiệu suất khác có thể áp dụng trong thực tế. Dựa trên kết quả đánh giá, có thể đưa ra một số khuyến nghị sau:

Đối với giao tiếp đồng bộ, việc triển khai Circuit Breaker pattern là rất quan trọng để ngăn chặn lỗi cascade và cải thiện khả năng chịu lỗi của hệ thống. Mẫu này giúp ngăn chặn các yêu cầu đến service không khả dụng, giảm thiểu tác động của lỗi dịch vụ đến toàn bộ hệ thống. Việc sử dụng timeouts hợp lý cũng giúp tránh tình trạng chờ đợi vô hạn khi service gặp sự cố.

Đối với giao tiếp bất đồng bộ, việc điều chỉnh kích thước hàng đợi và số lượng consumer có thể giúp cân bằng giữa thông lượng và độ trễ. Tăng số lượng consumer giúp cải thiện thông lượng, nhưng cũng làm tăng chi phí tài nguyên. Việc triển khai cơ chế retry với exponential backoff giúp xử lý các lỗi tạm thời, trong khi Dead Letter Queues giúp xử lý các thông điệp không thể xử lý.

Đối với mô hình Pub/Sub, việc phân vùng (partitioning) dữ liệu có thể giúp cải thiện khả năng mở rộng và hiệu suất. Việc chọn số lượng partition phù hợp với số lượng consumer giúp tối ưu hóa cân bằng tải và thông lượng. Việc duy trì kích thước thông điệp nhỏ và sử dụng định dạng nhị phân như Avro hoặc Protobuf thay vì JSON cũng giúp cải thiện hiệu suất.

\subsection{Xu hướng phát triển và nghiên cứu tiếp theo}
Dựa trên kết quả đánh giá và xu hướng hiện tại trong lĩnh vực, có thể thấy một số hướng phát triển và nghiên cứu tiếp theo đáng chú ý.

Một xu hướng đang phát triển là server-less communication, nơi các hàm serverless được sử dụng để xử lý các sự kiện và thông điệp mà không cần quan tâm đến cơ sở hạ tầng bên dưới. Mô hình này hứa hẹn giảm chi phí vận hành và cải thiện khả năng mở rộng, đặc biệt cho các hệ thống có tải không đều. Tuy nhiên, cần nghiên cứu thêm về hiệu suất và độ tin cậy của mô hình này trong các kịch bản khác nhau.

Một hướng nghiên cứu khác là Service Mesh, một lớp cơ sở hạ tầng chuyên dụng cho giao tiếp service-to-service, cung cấp các tính năng như service discovery, load balancing, encryption, observability, và authentication/authorization. Service Mesh có thể kết hợp các ưu điểm của cả giao tiếp đồng bộ và bất đồng bộ, đồng thời giải quyết một số thách thức về bảo mật và giám sát.

Việc kết hợp các mẫu giao tiếp với các công nghệ stream processing như Apache Flink hoặc Kafka Streams cũng là một hướng nghiên cứu hứa hẹn. Các công nghệ này cho phép xử lý dữ liệu liên tục theo thời gian thực, mở ra khả năng cho các ứng dụng phản ứng nhanh với các sự kiện và thay đổi dữ liệu.

Cuối cùng, việc áp dụng các kỹ thuật học máy và trí tuệ nhân tạo để tự động hóa việc lựa chọn và tối ưu hóa các mẫu giao tiếp dựa trên đặc điểm của tải và yêu cầu nghiệp vụ là một hướng nghiên cứu đầy tiềm năng. Các hệ thống tự thích ứng có thể điều chỉnh mẫu giao tiếp dựa trên các điều kiện hoạt động thực tế, tối ưu hóa hiệu suất và độ tin cậy một cách liên tục.

\subsection{Các hạn chế của dự án và hướng cải thiện}
Mặc dù dự án đã cung cấp những hiểu biết có giá trị về hiệu suất và độ tin cậy của các mẫu giao tiếp, vẫn có một số hạn chế cần được ghi nhận và cải thiện trong các nghiên cứu tương lai.

Trước hết, các kịch bản kiểm thử được thực hiện trong môi trường kiểm thử có kiểm soát, có thể không hoàn toàn phản ánh các điều kiện hoạt động thực tế. Các yếu tố như độ trễ mạng không đồng đều, sự cố phần cứng, và các vấn đề về bảo mật có thể ảnh hưởng đáng kể đến hiệu suất của các mẫu giao tiếp trong môi trường sản xuất thực tế.

Thứ hai, dự án tập trung chủ yếu vào hiệu suất và độ tin cậy, mà không đi sâu vào các khía cạnh khác như bảo mật, chi phí triển khai và vận hành, và khả năng tích hợp với các hệ thống hiện có. Những yếu tố này cũng là những cân nhắc quan trọng khi lựa chọn mẫu giao tiếp trong thực tế.

Thứ ba, việc đánh giá chỉ bao gồm ba mẫu giao tiếp chính, trong khi có nhiều mẫu và biến thể khác cũng được sử dụng trong thực tế, như gRPC, GraphQL, và WebSockets. Các mẫu này có thể có những ưu điểm đặc thù trong một số kịch bản cụ thể.

Cuối cùng, dự án không đánh giá hiệu suất của các mẫu giao tiếp trong thời gian dài, có thể làm thiếu sót các vấn đề như rò rỉ bộ nhớ, suy giảm hiệu suất theo thời gian, và khả năng phục hồi sau sự cố lớn.

Để cải thiện, các nghiên cứu tương lai có thể mở rộng phạm vi đánh giá để bao gồm nhiều mẫu giao tiếp hơn, các kịch bản đa dạng hơn, và thời gian kiểm thử dài hơn. Việc thực hiện các bài kiểm thử trong các môi trường đám mây thực tế với các yếu tố nhiễu như bị phát giả, tấn công DDoS, và sự cố phần cứng cũng sẽ cung cấp những hiểu biết có giá trị về độ tin cậy của các mẫu giao tiếp trong điều kiện khắc nghiệt.







% References
\begin{thebibliography}{9}
    \begin{bibsection}{Tiếng Việt}
    \end{bibsection}

    \begin{bibsection}{Tiếng Anh}
        \bibitem{idc2021}
        International Data Corporation (IDC),
        ``IDC FutureScape: Worldwide IT Industry 2021 Predictions'',
        \textit{IDC \#US46942020},
        October 2020.

        \bibitem{gartner2019}
        Gartner, Inc.,
        ``Gartner Identifies Key Trends in PaaS and Platform Architecture for Application Leaders'',
        \textit{Gartner Press Release},
        April 2019.

        \bibitem{jun2018}
        Jun Hong, X. et al.,
        ``Performance Analysis of RESTful API and RabbitMQ for Microservice Web Application'',
        \textit{IEEE ICTC},
        2018.

        \bibitem{richardson2019}
        Richardson, C.,
        \textit{Microservices Patterns},
        Manning Publications, 2019.

        \bibitem{newman2015}
        Newman, S.,
        \textit{Building Microservices},
        O'Reilly Media, 2015.

        \bibitem{aksakalli2021}
        Karabey Aksakalli, I., Çelik, T., Can, A. B., \& Tekinerdoğan, B.,
        ``Deployment and communication patterns in microservice architectures: A systematic literature review'',
        \textit{Journal of Systems and Software},
        Vol. 180, 2021, pp. 111014.

        \bibitem{wolff2016}
        Wolff, E.,
        \textit{Microservices: Flexible Software Architecture},
        Addison-Wesley Professional, 2016.

        \bibitem{fowler2014}
        Fowler, M.,
        ``Microservices'',
        \textit{https://martinfowler.com/articles/microservices.html},
        2014.

        \bibitem{hohpe2004}
        Hohpe, G., \& Woolf, B.,
        \textit{Enterprise Integration Patterns},
        Addison-Wesley, 2004.
    \end{bibsection}
\end{thebibliography}

\end{document}